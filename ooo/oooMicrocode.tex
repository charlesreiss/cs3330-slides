\usetikzlibrary{matrix}

\begin{frame}[fragile,label=pushqMicrocode]{renaming with microcode}
\lstset{language=myasm,style=small,morekeywords=iaddq}
\newcommand{\xAfter}[2]{%
    \alt<#1->{\myemph<#1>{\sout{#2}}}{#2}%
}
\newcommand{\showAfter}[2]{%
    \alt<#1->{\myemph<#1>{#2}}{}%
}
\begin{tikzpicture}
\tikzset{
    appear on/.style={
        alt=<#1->{}{invisible},
        alt=<#1>{draw=red,very thick},
    },
}
\matrix[tight matrix no line,
    column 1/.style={nodes={text width=5cm}},
    column 2/.style={nodes={text width=8cm,text depth=.1cm}},
] (instrs) {
    |[align=center]| original \& |[align=center]| renamed \\
    \lstinline|pushq %r8| \& |[appear on=2]| \lstinline|iaddq $8, %x11| $\rightarrow$ \lstinline|%x18| \\
    ~ \& |[appear on=2]| \lstinline|rmmovq %x13, 0(%x18)| $\rightarrow$ (memory) \\
    ~ \& ~ \\
};
\tikzset{
    box/.style={
        draw=red,ultra thick,
        anchor=west,
        align=left,
        font=\small,
        fill=white,
        at={([xshift=-2cm,yshift=2cm]free.east)},
    }
}
\matrix[tight matrix,label={[align=center]north:arch $\rightarrow$ phys \\ register map},
    nodes={font=\tt\small,text width=1.5cm,minimum height=0.45cm},
    column 2/.style={nodes={text width=6cm}},
    anchor=north west
    ] (rmap) at ([yshift=-1.5cm]instrs.south west) {
\%rax \& \%x04 \\
\%rcx \& \%x09 \\
\ldots \& \ldots \\
\%rsp \& \xAfter{2}{\%x11}\showAfter{2}{\%x18} \\
\ldots \& \ldots \\
\%r8 \& \%x13 \\
\%r9 \& \%x17 \\
\%r10 \& \%x19 \\
\ldots \& \ldots \\
};
    \matrix[tight matrix,label={[align=center]free\\regs},anchor=west,nodes={font=\tt\small}] (free) at ([xshift=1cm]rmap.east) {
\xAfter{2}{\%x18} \\
\%x20 \\
\%x21 \\
\%x23 \\
\%x24 \\
\ldots \\
};
\begin{visibleenv}<2>
\node[box] {
    pushq is really complicated \\
    one implementation option: \\
    split into simpler ``microinstructions'' \\
    also exposes \%rsp to register renmaing
};
\end{visibleenv}
\end{tikzpicture}
\end{frame}


\begin{frame}[fragile,label=renameCC]{renaming and condition codes}
\lstset{language=myasm,style=small}
\newcommand{\xAfter}[2]{%
    \alt<#1->{\myemph<#1>{\sout{#2}}}{#2}%
}
\newcommand{\showAfter}[2]{%
    \alt<#1->{\myemph<#1>{#2}}{}%
}
\begin{tikzpicture}
\tikzset{
    appear on/.style={
        alt=<#1->{}{invisible},
        alt=<#1>{draw=red,very thick},
    },
}
\matrix[tight matrix no line,
    column 1/.style={nodes={text width=5cm}},
    column 2/.style={nodes={text width=8cm,text depth=.1cm}},
] (instrs) {
    |[align=center]| original \& |[align=center]| renamed \\
    \lstinline|cmpq %r8, %r9| \& |[appear on=2]| \lstinline|cmpq %x13, %x17| $\rightarrow$ \lstinline|%x18.cc| \\
    \lstinline|jle D| \& |[appear on=2]| \lstinline|jle %x18.cc, D| \\
};
\tikzset{
    box/.style={
        draw=red,ultra thick,
        anchor=west,
        align=left,
        font=\small,
        fill=white,
        at={([xshift=-2cm,yshift=1.5cm]free.north east)},
    }
}
\matrix[tight matrix,label={[align=center]north:arch $\rightarrow$ phys \\ register map},
    nodes={font=\tt\small,text width=1.5cm,minimum height=0.45cm},
    column 2/.style={nodes={text width=4.5cm}},
    anchor=north west
    ] (rmap) at ([yshift=-1.5cm]instrs.south west) {
\%rax \& \%x04 \\
\%rcx \& \%x09 \\
\ldots \& \ldots \\
\%rsp \& \%x18 \\
\ldots \& \ldots \\
\%r8 \& \%x13 \\
\%r9 \& \%x17 \\ % \xAfter{2}{\%x17}\showAfter{2}{\%x18} \\
\%r10 \& \%x19 \\
\ldots \& \ldots \\
|[alt=<1-3>{fill=red!10}]| CC\& |[alt=<1-3>{fill=red!10}]| \xAfter{2}{\%x04.cc}\showAfter{2}{\%x18.cc} \\
};
    \matrix[tight matrix,label={[align=center]free\\regs},anchor=south west,nodes={font=\tt\small}] (free) at ([xshift=1cm]rmap.south east) {
\xAfter{1}{\%x18} \\
\%x20 \\
\%x21 \\
\%x23 \\
\%x24 \\
\ldots \\
};
\begin{visibleenv}<2>
\node[box] {
    one option for condition codes: \\
    map to physical registers and track with renaming \\
    not sure if real CPUs do this option \\
    (complicates the commit stage? \\
    more area for regs than alternative?)
    ~ \\
    alternative 1: entirely separate cond. code registers \\
    (with separate free register list) \\
    alternative 2: handle in `in-order' part of pipeline?
};
\end{visibleenv}
\end{tikzpicture}
\end{frame}

\begin{frame}[fragile,label=renamingTrickiness1]{renaming trickiness (1)}
    \begin{itemize}
    \item need to expose input + outputs
    \item hidden dependencies on stack pointer, condition codes, memory, etc.
    \item stack pointer + condition codes
        \begin{itemize}
        \item turn into visible register somehow
        \item alternative: force to execute in-order
        \end{itemize}
    \item memory: complex techniques we won't discuss
    \end{itemize}
\end{frame}

\begin{frame}[fragile,label=renamingTrickiness2]{renaming trickiness (2)}
    \begin{itemize}
    \item opportunity to translate complex instructions into simpler
    \item commonly used for memory operands, probably some stack instructions
        \begin{itemize}
        \item \lstinline|popq %rcx| $\rightarrow$ addq, store
        \item \lstinline|addq %rax, (%rbx)| $\rightarrow$ load, addq, store
        \item \ldots
        \end{itemize}
    \end{itemize}
\end{frame}

