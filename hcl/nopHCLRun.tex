\begin{frame}{nop CPU: running}
    \begin{itemize}
    \item need a program in memory
        \begin{itemize}
        \item .yo file
        \end{itemize}
    \item {\tt tools/yas} --- convert {\tt .ys}  to {\tt .yo}
    \item {\tt tools/yis} --- reference interpreter for {\tt .yo} files
        \begin{itemize}
        \item if your processor doesn't do the same thing\ldots
        \end{itemize}
    \item can build tools by running {\tt make}
    \end{itemize}
\end{frame}

\begin{frame}[fragile,label=makeProgram]{nop CPU: creating a program}
    \begin{itemize}
    \item create assemby file: nops.ys:
\begin{Verbatim}
    nop
    nop
    nop
    nop
    nop
\end{Verbatim}
    \item assemble using {\tt tools/yas nops.ys} or {\tt make nops.yo}
    \end{itemize}
\end{frame}

\begin{frame}[fragile,label=nopYo]{nop.yo}
    \begin{itemize}
    \item more readable/simpler than normal executables:
\begin{Verbatim}
0x000: 10                   | nop
0x001: 10                   | nop
0x002: 10                   | nop
0x003: 10                   | nop
0x004: 10                   | nop
                            | 
\end{Verbatim}
    \item loaded into data and program memory
    \item parts left of {\tt |} just comments
    \end{itemize}
\end{frame}

\begin{frame}[fragile,label=runSim1]{running a simulator (1)}
    \begin{itemize}
    \vspace{.25cm}
\begin{Verbatim}[fontsize=\fontsize{8}{9}\selectfont]
Usage: ./hclrs [options] HCL-FILE [YO-FILE [TIMEOUT]]
Runs HCL_FILE on YO-FILE. If --check is specified, no YO-FILE may be supplied.
Default timeout is 9999 cycles.

Options:
    -c, --check         check syntax only
    -d, --debug         output wire values after each cycle and other debug
                        output
    -q, --quiet         only output state at the end
    -t, --testing       do not output custom register banks (for autograding)
    -h, --help          print this help menu
    -i, --interactive   prompt after each cycle
        --trace-assignments 
                        show assignments in the order they are simulated
        --version       print version number
\end{Verbatim}
    \end{itemize}
\end{frame}

\begin{frame}[fragile,label=runSim2]{running a simulator (2)}
\begin{Verbatim}[fontsize=\fontsize{8}{9}\selectfont,commandchars=\\\{\}]
$ ./hclrs nop_cpu.hcl nops.yo
+------------------- between cycles    0 and    1 ----------------------+
| RAX:                0   RCX:                0   RDX:                0 |
| RBX:                0   RSP:                0   RBP:                0 |
| RSI:                0   RDI:                0   R8:                 0 |
| R9:                 0   R10:                0   R11:                0 |
| R12:                0   R13:                0   R14:                0 |
| register \vemphA{pF(N) { thePc=0000000000000000 }}                             |
| used memory:   _0 _1 _2 _3  _4 _5 _6 _7   _8 _9 _a _b  _c _d _e _f    |
|  0x0000000_:   \vemphB{10 10 10 10  10}                                        |
+-----------------------------------------------------------------------+
\vemphC{pc = 0x0; loaded [10 : nop]}
....
+------------ timed out after  9999 cycles in state: -------------------+
| RAX:                0   RCX:                0   RDX:                0 |
| RBX:                0   RSP:                0   RBP:                0 |
| RSI:                0   RDI:                0   R8:                 0 |
| R9:                 0   R10:                0   R11:                0 |
| R12:                0   R13:                0   R14:                0 |
| register pF(N) { thePc=000000000000270f }                             |
| used memory:   _0 _1 _2 _3  _4 _5 _6 _7   _8 _9 _a _b  _c _d _e _f    |
|  0x0000000_:   10 10 10 10  10                                        |
+-----------------------------------------------------------------------+
\end{Verbatim}
\end{frame}

