\usetikzlibrary{arrows.meta,calc,positioning,matrix,patterns}

\begin{frame}<1-2>[label=timeMultiReally]{time multiplexing really}
\begin{tikzpicture}
\tikzset{
    prog1/.style={draw,fill=cyan},
    prog2/.style={draw,fill=green},
    prog3/.style={draw,fill=violet!30},
    proglabel/.style={font=\tt\scriptsize},
    labelprog1/.style={execute at begin node={\strut loop.exe}},
    labelprog2/.style={execute at begin node={\strut ssh.exe}},
    labelprog3/.style={execute at begin node={\strut firefox.exe}},
}

\begin{scope}[xscale=1.5,yscale=1]
\foreach \s/\e/\p [count=\x] in {0/2/1,2/3/2,3/5/3,5/6/1,6/7/2}{
    \coordinate (s-\x) at (\s, 0);
    \coordinate (e-\x) at (\e, 0);
    \draw[prog\p] (\s, 0) rectangle (\e, 1) coordinate[midway] (mid-\x);
    \node[anchor=center,proglabel,labelprog\p] at (mid-\x) {};
    \begin{pgfonlayer}{fg}
    \draw[fill=white] ([xshift=-.05cm]e-\x) rectangle ([xshift=.05cm,yshift=1cm]e-\x);
    \draw[pattern=north west lines] ([xshift=-.05cm]e-\x) rectangle ([xshift=.05cm,yshift=1cm]e-\x);
   \end{pgfonlayer}
}
\end{scope}
\begin{scope}[xshift=5cm]
\draw[fill=white,pattern=north west lines] (0, -1) rectangle (1, -2);
\node[anchor=west] at (1, -1.5) {\strut = operating system};
\end{scope}
\begin{visibleenv}<2->
    \draw[red,very thick,Latex-] ([xshift=-.05cm]e-1) -- (1, -4) node[fill=white,draw] {exception happens};
    \draw[red,very thick,Latex-] ([xshift=+.05cm]e-1) -- (7, -4) node[fill=white,draw] {return from exception};
\end{visibleenv}
\end{tikzpicture}
\end{frame}

\begin{frame}{OS and time multiplexing}
% FIXME: picture showing zoom of timeline
\begin{itemize}
\item starts running instead of normal program
    \begin{itemize}
    \item mechanism for this: \myemph{exceptions} (later)
    \end{itemize}
\item saves old program counter, registers somewhere
\item sets new registers, jumps to new program counter
\item called \myemph{context switch}
    \begin{itemize}
    \item saved information called \myemph{context}
    \end{itemize}
\end{itemize}
\end{frame}

\begin{frame}<1>[fragile,label=context]{context}
\begin{itemize}
\item all registers values
    \begin{itemize}
        \item \lstinline|%rax| \lstinline|%rbx|, \ldots, \myemph{\tt \%rsp}, \ldots
    \end{itemize}
\item condition codes
\item program counter
\item \sout<2->{i.e. all visible state in your CPU except memory}
\item<2-> \myemph{address space}: map from program to real addresses
\end{itemize}
\end{frame}

\begin{frame}[fragile,label=ctxtSwitchPseudo]{context switch pseudocode}
\lstset{
    style=small,
    language=myasm,
    morekeywords={copy_preexception_pc},
}
\begin{lstlisting}
context_switch(last, next):
  copy_preexception_pc last->pc
  mov rax,last->rax 
  mov rcx, last->rcx 
  mov rdx, last->rdx
  ...
  mov next->rdx, rdx
  mov next->rcx, rcx
  mov next->rax, rax
  jmp next->pc
\end{lstlisting}
\end{frame}

\tikzset{a/.style={fill=blue!30},b/.style={fill=green!30},>=Latex}
\newsavebox{\aContext}
\savebox{\aContext}{
\begin{tikzpicture}
\matrix[tight matrix,nodes={font=\small\tt,a}] (cpuState) {
    \%rax \& SF \\
    \%rbx \& ZF \\
    \%rcx \& PC \\
    \ldots \& \ldots \\
};
\end{tikzpicture}
}
\newsavebox{\bContext}
\savebox{\bContext}{
\begin{tikzpicture}
\matrix[tight matrix,nodes={font=\small\tt,b}] (cpuState) {
    \%rax \& SF \\
    \%rbx \& ZF \\
    \%rcx \& PC \\
    \ldots \& \ldots \\
};
\end{tikzpicture}
}


\begin{frame}{contexts (A running)}
\begin{tikzpicture}
\matrix[tight matrix,label={north:in CPU},nodes={font=\small\tt,a}] (cpuState) {
    \%rax \\
    \%rbx \\
    \%rcx \\
    \%rsp \\
    \ldots \\
    SF \\
    ZF \\
    PC \\
};

\matrix[tight matrix,right=3cm of cpuState, nodes={minimum height=2cm,text width=5cm,align=left},label={in Memory}] (memState) {
    |[a]| {Process A memory: \\ code, stack, etc.} \\
    |[b]| {Process B memory: \\ code, stack, etc.} \\
    |[fill=black!10]|{OS memory: \\ \usebox{\bContext} }\\
};
\draw[thick,dashed,->] (cpuState-4-1.east) -- (memState-1-1.west);
\draw[thick,dashed,->] (cpuState-8-1.east) -- (memState-1-1.west);
\end{tikzpicture}
\end{frame}

\begin{frame}{contexts (B running)}
\begin{tikzpicture}
\tikzset{a/.style={fill=blue!30},b/.style={fill=green!30}}
\matrix[tight matrix,label={north:in CPU},nodes={font=\small\tt,b}] (cpuState) {
    \%rax \\
    \%rbx \\
    \%rcx \\
    \%rsp \\
    \ldots \\
    SF \\
    ZF \\
    PC \\
};
\matrix[tight matrix,right=3cm of cpuState, nodes={minimum height=2cm,text width=5cm},label={north:in Memory}]
    (memState) {
    |[a]| {Process A memory: \\ code, stack, etc.} \\
    |[b]| {Process B memory: \\ code, stack, etc.} \\
    |[fill=black!10]| {OS memory: \\ \usebox{\aContext}} \\
};
\draw[thick,dashed,->] (cpuState-4-1.east) -- (memState-2-1.west);
\draw[thick,dashed,->] (cpuState-8-1.east) -- (memState-2-1.west);
\end{tikzpicture}
\end{frame}


