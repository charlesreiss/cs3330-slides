\usepgflibrary{shapes.gates.logic.mux}
\usetikzlibrary{calc,chains,shapes.misc,shapes.callouts,shapes.geometric,shapes.gates.logic.US,circuits.logic.US}

\begin{frame}[fragile,label=buildPCUpdate]{building the PC update (one possibility)}
\begin{tikzpicture}[circuit logic]
    \tikzset{
        note/.style={align=left,inner sep=0.5mm},
        note border/.style={draw=red, ultra thick},
        hiliteFill/.style={fill=red!15},
        hiliteCircuit/.style={draw=red,very thick},
    }
    \node[hReg={\fontsize{9}{10}\selectfont PC},minimum height=1.5cm,fill=yellow] (pc) {};
    \node[anchor=west,font=\small] (to instr mem) at ([xshift=7cm]pc.east) { to instr. mem };
    \begin{visibleenv}<1-3>
        \draw[a] (pc.east) -- (to instr mem);
    \end{visibleenv}
    \node[anchor=north west,logicFill,font=\small,inner sep=1mm] (plus instr len) at ([xshift=2.5cm,yshift=-1.2cm]pc.south west) {
        + instr. length
    };
    \begin{visibleenv}<1-3>
    \draw[a] (pc.east) -| (plus instr len.north);
    \end{visibleenv}
    \begin{visibleenv}<1>
        \draw[a,alt=<1>{red}] (plus instr len.west) -| ([xshift=-1cm]pc.west)  -- (pc.west);
    \end{visibleenv}

    \begin{visibleenv}<1->
        \node[note,anchor=south west,alt=<1>{hiliteFill}] (note1) at ([xshift=-3cm,yshift=4.5cm]pc.north west) {
            (1) normal case: PC $\leftarrow$ PC + instr len
        };
    \end{visibleenv}
    \begin{visibleenv}<2->
        \node[note,anchor=north west,alt=<2>{hiliteFill}] (note2) at (note1.south west) {
            (2) immediate: call/jmp, and \textit{prediction} for cond. jumps
        };
    \end{visibleenv}
    \begin{visibleenv}<3->
        \node[note,anchor=north west,alt=<3>{hiliteFill}] (note3) at (note2.south west) {
            (3) repeat previous PC for stalls (load/use hazard, halt, ret?)
        };
    \end{visibleenv}
    \begin{visibleenv}<4->
        \node[note,anchor=north west,alt=<4>{hiliteFill}] (note4) at (note3.south west) {
            (4) correct for misprediction of conditional jump
        };
    \end{visibleenv}
    \begin{visibleenv}<5->
        \node[note,anchor=north west,alt=<5>{hiliteFill}] (note5) at (note4.south west) {
            (5) correct for missing return address for \texttt{ret}
        };
    \end{visibleenv}
    \begin{visibleenv}<2->
        \node[anchor=south west,align=right,font=\small] (immed) at ([yshift=1.9cm]to instr mem.north west) {
            immediate value
        };
        \node[alt=<2>{hiliteCircuit},anchor=output,mux,inputs={nnn},minimum height=2cm,
              logicFill,font=\small] (beforePcMux) at ([xshift=-1cm]pc.east) {MUX};
        \draw[a,alt=<2>{hiliteCircuit}] (immed) -| ([xshift=-1cm]beforePcMux.input 1) -- (beforePcMux.input 1);
        \draw[a] (plus instr len.west) -| ([xshift=-1cm]beforePcMux.input 2) -- (beforePcMux.input 2);
        \draw[a,alt=<2>{hiliteCircuit}] (beforePcMux.output) -- (pc.west);
        \node[inner sep=0mm,
              fit={
                  ([yshift=-2.25mm,xshift=-3mm]pc.south)
                  ([yshift=-3.5mm,xshift=3mm]pc.south)
              },cross out,alt=<2>{draw=red}{draw=black},ultra thick] {};
        \node[font=\fontsize{9}{10}\selectfont,anchor=north] at ([yshift=-3.5mm]pc.south) {predicted PC};
    \end{visibleenv}
    \begin{visibleenv}<2>
        \draw[a,hiliteCircuit] (beforePcMux.input 1) -- (beforePcMux.output);
    \end{visibleenv}
    \begin{visibleenv}<3>
        \draw[a,alt=<3>{hiliteCircuit}] (pc.east) -- ++(1.5cm, 0cm) |- ([yshift=-1cm]pc.south) 
            -| ([xshift=-.5cm]beforePcMux.input 3) -- (beforePcMux.input 3);
    \end{visibleenv}
    \begin{visibleenv}<3>
        \draw[a,hiliteCircuit] (beforePcMux.input 3) -- (beforePcMux.output) -- (pc.west);
    \end{visibleenv}
    \begin{visibleenv}<4->
        \node[alt=<4>{hiliteCircuit},anchor=input 3,mux,inputs={nnn},minimum height=1.5cm,
            logicFill,font=\scriptsize] (afterPcMux) at([xshift=1cm]pc.east) {MUX};
        \draw[a] (pc.east) -- (afterPcMux.input 3);
        \draw[a] (afterPcMux.output) -| (plus instr len.north);
        \draw[a] (afterPcMux.output) -| (to instr mem.west -| plus instr len.north) -- (to instr mem.west);
        \draw[a] (afterPcMux.output) -- ++(.5cm, 0cm) |- ([yshift=-1cm]pc.south) 
            -| ([xshift=-.5cm]beforePcMux.input 3) -- (beforePcMux.input 3);
        \node[anchor=south west,align=right,font=\small] (cond jump correct) at ([yshift=0.6cm]to instr mem.north west) {
            next PC from cond. jump
        };
        \draw[a,alt=<4>{hiliteCircuit}] (cond jump correct.west) -| ([xshift=-.4cm]afterPcMux.input 1) --
            (afterPcMux.input 1);
    \end{visibleenv}
    \begin{visibleenv}<4>
        \draw[a,hiliteCircuit] (afterPcMux.input 1) -- (afterPcMux.output);
    \end{visibleenv}
    \begin{visibleenv}<5->
        \node[anchor=south west,align=right,font=\small] (ret correct) at ([yshift=1.2cm]to instr mem.north west) {
            return address from ret
        };
        \draw[a,alt=<5>{hiliteCircuit}] (ret correct.west) -| ([xshift=-.8cm]afterPcMux.input 2) --
            (afterPcMux.input 2);
    \end{visibleenv}
    \begin{visibleenv}<5>
        \draw[a,hiliteCircuit] (afterPcMux.input 2) -- (afterPcMux.output);
    \end{visibleenv}
    \begin{visibleenv}<6>
        \tikzset{
            from mark/.style={fill={#1},draw=black,thick,inner sep=.5mm,font=\small}
        }
        \node[from mark=yellow,anchor=east] at ([xshift=-.5cm]immed.west) {via instr. mem};
        \node[from mark={rgb:green,3;blue,3;white,5},anchor=east] at ([xshift=-.5cm]cond jump correct.west) {via execute/memory pipeline regs};
        \node[from mark={rgb:blue,3;violet,3;white,5},anchor=east] at ([xshift=-.5cm]ret correct.west) {via memory/writeback pipeline regs};
    \end{visibleenv}
\end{tikzpicture}
\end{frame}

\begin{frame}{PC update overview}
    \begin{itemize}
    \item predict based on instruction length + immediate
    \item override prediction with stalling sometimes
    \item correct when prediction is wrong just before fetching
        \begin{itemize}
        \item retrieve corrections from pipeline register outputs for jCC/ret instruction
        \end{itemize}
    \vspace{.5cm}
    \item above is what textbook does
    \item alternative: could instead correct prediction just before setting PC register
        \begin{itemize}
        \item retrieve corrections into PC cycle before corrections used
        \item moves logic from beginning-of-fetch to end-of-previous-fetch
        \end{itemize}
    \item {\small I think this is more intuitive, but consistency with textbook is less confusing\ldots}
    \end{itemize}
\end{frame}
