\begin{frame}{on virtual address sizes}
    \begin{itemize}
    \item virtual address size = size of pointer?
    \vspace{.5cm}
    \item often, but --- sometimes part of pointer not used
    \item example: typical x86-64 only use 48 bits
        \begin{itemize}
        \item rest of bits have fixed value
        \end{itemize}
    \item virtual address size is amount used for mapping
    \end{itemize}
\end{frame}

\begin{frame}{address space sizes}
\begin{itemize}
    \item amount of stuff that can be addressed = address space size
        \begin{itemize}
        \item based on number of unique addresses
        \end{itemize}
    \item e.g. 32-bit virtual address = $2^{32}$ byte virtual address space
    \item e.g. 20-bit physical addresss = $2^{20}$ byte physical address space
    \vspace{.5cm}
    \item<2-> what if my machine has 3GB of memory (not power of two)?
        \begin{itemize}
        \item not all addresses in physical address space are useful
        \item most common situation (since CPUs support having a lot of memory)
        \end{itemize}
\end{itemize}
\end{frame}
