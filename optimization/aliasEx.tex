\begin{frame}[fragile,label=aliasingEx]{aliasing exercise}
\begin{lstlisting}[language=C++,style=smaller]
void add(int *s1, int *s2, int *d) {
    for (int i = 0; i < 1000; ++i)
        d[i] = s1[i] + s2[i];
}
\end{lstlisting}
\hrule
The compiler \textbf{cannot} generate code equivalent to this:
\begin{lstlisting}[language=C++,style=smaller]
void add(int *s1, int *s2, int *d) {
    for (int i = 0; i < 1000; i += 2) {
        int temp1 = s1[i] + s2[i];
        int temp2 = s1[i+1] + s2[i+1];
        d[i] = temp1; d[i+1] = temp2;
    }
}
\end{lstlisting}
Which is an example of a call where the results could disagree:
{\small
\begin{tabular}{lll}
A. \texttt{add(\&A[0], \&A[1], \&B[0])} & B. \texttt{add(\&A[0], \&A[0], \&A[1])} \\
C. \texttt{add(\&B[0], \&A[10], \&A[0])} & D. \texttt{add(\&A[1000], \&A[1001], \&A[0])} \\
\end{tabular}
}
{\small (assume A, B are distinct, large arrays)}
\end{frame}

\begin{frame}<0>[fragile,label=aliasingExExplain]{aliasing exercise}
recall: s1=s2=A+0; d=A+1
\begin{lstlisting}[language=C++,style=smaller]
for (int i = 0; i < 1000; ++i)
    d[i] = s1[i] + s2[i];

/* i = 0: */ A[1] = A[0] + A[0];
/* i = 1: */ A[2] = A[1] + A[1];
\end{lstlisting}
\hrule
\begin{lstlisting}[language=C++,style=smaller]
for (int i = 0; i < 1000; i += 2) {
    temp1 = s1[i] + s2[i];
    temp2 = s1[i] + s2[i];
    d[i] = temp1;
    d[i] = temp2;

/* i = 0: */ temp1 = A[0] + A[0];
             temp2 = A[1] + A[1];
             A[1] = temp1;
             A[2] = temp2;
\end{lstlisting}
\end{frame}

\iftoggle{heldback}{}{\againframe<1>{aliasingExExplain}}
