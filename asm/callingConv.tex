\begin{frame}{x86-64 calling convention}
    \begin{itemize}
    \item \myemph{registers} for first 6 arguments:
    \begin{itemize}
    \item {\tt \%rdi} (or {\tt \%edi} or {\tt \%di}, etc.), then
    \item {\tt \%rsi} (or {\tt \%esi} or {\tt \%si}, etc.), then
    \item {\tt \%rdx} (or {\tt \%edx} or {\tt \%dx}, etc.), then
    \item {\tt \%rcx} (or {\tt \%ecx} or {\tt \%cx}, etc.), then
    \item {\tt \%r8} (or {\tt \%r8d} or {\tt \%r8w}, etc.), then
    \item {\tt \%r9} (or {\tt \%r9d} or {\tt \%r9w}, etc.)
    \end{itemize}
    \item rest on stack
    \item return value in {\tt \%rax}
    \item don't memorize: Figure 3.28 in book
    \end{itemize}
\end{frame}

% FIXME: %rip?

\begin{frame}[fragile,label=x8664CCExample]{x86-64 calling convention example}
\begin{ccodeNL*}{fontsize=\small}
int foo(int x, int y, int z) { return 42; }
...
    foo(1, 2, 3);
...
\end{ccodeNL*}
\hrule
\begin{asmcodeS}
...
    // foo(1, 2, 3)
    movl $1, %edi
    movl $2, %esi
    movl $3, %edx
    call foo  // call pushes address of next instruction
              // then jumps to foo
...
foo: 
    movl $42, %eax
\end{asmcodeS}
\end{frame}

\begin{frame}[fragile,label=stackFrame]{call/ret}
\begin{itemize}
\item call:
    \begin{itemize}
    \item push address of \myemph{next instruction} on the stack
    \end{itemize}
\item ret:
    \begin{itemize}
    \item pop address from stack; jump
    \end{itemize}
\end{itemize}
\end{frame}

\begin{frame}[fragile,label=calleeSaved]{callee-saved registers}
    \begin{itemize}
    \item functions \myemph{must preserve} these
    \vspace{.5cm}
    \item {\tt \%rsp} (stack pointer), {\tt \%rbx}, {\tt \%rbp} (frame pointer, maybe)
    \item {\tt \%r12-\%r15}
    \end{itemize}
\end{frame}

\begin{frame}[fragile,label=callerCallee]{caller/callee-saved}
\lstset{style=small}
\begin{lstlisting}
foo:
    pushq %r12 // r12 is caller-saved
    ... use r12 ...
    popq %r12
    ret

...
other_function:
    ...
    pushq %r11 // r11 is caller-saved
    callq foo
    popq %r11
\end{lstlisting}
\end{frame}

