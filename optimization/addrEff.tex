\begin{frame}[fragile,label=addrEff]{addressing efficiency}
    \lstset{language=C,style=smaller,
    moredelim=**[is][\btHL<1>]{~2~}{~end~},
}
    \vspace{-.25cm}
\begin{lstlisting}
for (int kk = 0; kk < N; kk += 2) {
  for (int i = 0; i < N; ++i) {
    for (int j = 0; j < N; ++j) {
      float Cij = C[i * N + j];
      for (int k = kk; k < kk + 2; ++k) {
        Cij += A[~2~i * N~end~ + k] * B[~2~k * N~end~ + j];
      }
      C[i * N + j] = Cij;
    }
  }
}
\end{lstlisting}
\begin{itemize}
    \item tons of multiplies by N??
    \item isn't that slow?
\end{itemize}
\end{frame}

\begin{frame}[fragile,label=addrXform]{addressing transformation}
    \lstset{language=C,style=smaller,
    moredelim=**[is][\btHL<2>]{~2~}{~end~},
}
    \vspace{-.25cm}
\begin{lstlisting}
for (int kk = 0; k < N; kk += 2)
  for (int i = 0; i < N; ++i) {
    for (int j = 0; j < N; ++j) {
      float Cij = C[i * N + j];
      float *Bkj_pointer = &B[kk * N + j];
      for (int k = kk; k < kk + 2; ++k) {
        // Bij += A[i * N + k] * A[k * N + j~];
        Bij += A[i * N + k] * Bkj_pointer;
        Bkj_pointer += N;
      }
      C[i * N + j] = ~2~Bij~end~;
    }
  }
\end{lstlisting}
\begin{itemize}
    \item transforms loop to \myemph{iterate with pointer}
    \item \myemph{compiler} will often do this 
    \item increment/decrement by N ($\times$ sizeof(float))
\end{itemize}
\end{frame}

\begin{frame}[fragile,label=addressEff]{addressing efficiency}
    \begin{itemize}
        \item compiler will \myemph{usually} eliminate slow multiplies
            \begin{itemize}
            \item doing transformation yourself often slower if so
            \end{itemize}
        \item \lstinline|i * N; ++i| into \lstinline|i_times_N; i_times_N += N|
        \item way to check: see if assembly uses lots multiplies in loop
        \item if it doesn't --- do it yourself
    \end{itemize}
\end{frame}

\begin{frame}[fragile,label=addrXform2]{another addressing transformation}
\lstset{language=C,style=smaller,
    moredelim=**[is][\btHL<2>]{~1~}{~end~},
    moredelim=**[is][\btHL<2>]{~2~}{~end~},
}
    \vspace{-.25cm}
\begin{lstlisting}
    for (int i = 0; i < n; i += 4) {
        C[(i+0) * n + j] += ~1~A[(i+0) * n + k]~end~ * B[k * n + j];
        C[(i+1) * n + j] += ~1~A[(i+1) * n + k]~end~ * B[k * n + j];
        // ...
\end{lstlisting}
\hrule
\begin{lstlisting}
    int offset = 0;
    float *Ai0_base = &A[k];
    float *Ai1_base = Ai0_base + n;
    float *Ai2_base = Ai1_base + n;
    // ...
    for (int i = 0; i < n; i += 4) {
        C[(i+0) * n + j] += ~1~Ai0_base[offset]~end~ * B[k * n + j];
        C[(i+1) * n + j] += ~1~Ai1_base[offset]~end~ * B[k * n + j];
        // ...
        offset += n;
\end{lstlisting}
    \begin{itemize}
    \item compiler will sometimes do this, too
    \end{itemize}
\end{frame}

\begin{frame}[fragile,label=addrXform2Backfire]{another addressing transformation}
\lstset{language=C,style=smaller,
    moredelim=**[is][\btHL<2>]{~2~}{~end~},
}
    \vspace{-.25cm}
\begin{lstlisting}
    for (int i = 0; i < n; i += ~2~20~end~) {
        C[(i+0) * n + j] += A[(i+0) * n + k] * B[k * n + j];
        C[(i+1) * n + j] += A[(i+1) * n + k] * B[k * n + j];
        // ...
\end{lstlisting}
\hrule
\begin{lstlisting}
    int offset = 0;
    float *Ai0_base = &A[0*n+k];
    float *Ai1_base = Ai0_base + n;
    float *Ai2_base = Ai1_base + n;
    // ...
    for (int i = 0; i < n; i += ~2~20~end~) {
        C[(i+0) * n + j] += Ai0_base[i*n] * B[k * n + j];
        C[(i+1) * n + j] += Ai1_base[i*n] * B[k * n + j];
        // ...
        offset += n;
\end{lstlisting}
    \begin{itemize}
    \item storing 20 \lstinline|AiX_base|? --- need the stack
    \item maybe faster (quicker address computation)
    \item maybe slower (can't do enough loads)
    \end{itemize}
\end{frame}

\begin{frame}[fragile,label=addrXform2BackfireAlt]{alternative addressing transformation}
\lstset{language=C,style=smaller,
    moredelim=**[is][\btHL<2>]{~2~}{~end~},
}
instead of:
\begin{lstlisting}
    float *Ai0_base = &A[0*n+k];
    float *Ai1_base = Ai0_base + n;
    // ...
    for (int i = 0; i < n; i += ~2~20~end~) {
        C[(i+0) * n + j] += Ai0_base[i*n] * B[k * n + j];
        C[(i+1) * n + j] += Ai1_base[i*n] * B[k * n + j];
        // ...
\end{lstlisting}
\hrule
could do:
\begin{lstlisting}
    float *Ai0_base = &A[k];
    for (int i = 0; i < n; i += ~2~20~end~) {
        float *A_ptr = &Ai0_base[i*n];
        C[(i+0) * n + j] += *A_ptr * A[k * n + j];
        A_ptr += n;
        C[(i+1) * n + j] += *A_ptr * B[k * n + j];
        // ...
\end{lstlisting}
    \begin{itemize}
    \item avoids spilling on the stack, but more dependencies
    \end{itemize}
\end{frame}

\begin{frame}{addressing efficiency generally}
    \begin{itemize}
    \item mostly: compiler does very good job itself
        \begin{itemize}
        \item eliminates multiplications, use pointer arithmetic
        \item often will do better job than if how typically programming would do it manually
        \end{itemize}
    \vspace{.5cm}
    \item sometimes compiler won't take the best option
        \begin{itemize}
        \item if spilling to the stack: can cause weird performance anomalies
        \item if indexing gets too complicated --- might not remove multiply
        \end{itemize}
    \item if compiler doesn't, you can always make addressing simple yourself
        \begin{itemize}
        \item convert to pointer arith. without multiplies
        \end{itemize}
    \end{itemize}
\end{frame}

