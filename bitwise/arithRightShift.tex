\usetikzlibrary{arrows.meta,calc,decorations.pathreplacing}

\tikzset{
    simple wire/.style={very thick,>=Latex},
    wire/.style={line width=1.5pt,>=Latex},
    wire small/.style={line width=1pt,>=Latex},
    binLabel/.style={font=\tt},
    hiBox/.style={red, very thick, draw},
    >=Latex,
}

\begin{frame}{arithmetic right shift}
\begin{itemize}
\item x86 instruction: {\keywordstyle sar} --- arithmetic shift right
\item {{\keywordstyle sar} \tt\$\textit{amount}, \%reg} (or variable: {\tt{\keywordstyle sar} \%cl, \%reg})
\end{itemize}
\begin{tikzpicture}
\begin{visibleenv}<1-2>
\draw[blue,decorate,decoration=brace,ultra thick] (-.2, .7) -- (13., .7) node[midway, above] {\%reg (initial value)};
\draw[blue,decorate,decoration={brace,mirror},ultra thick] (-.2, -3.1) -- (13., -3.1) node[midway, below] {\%reg (final value)};
\end{visibleenv}
\begin{visibleenv}<3>
\draw[blue,decorate,decoration=brace,ultra thick] (-.2, 1.2) -- (13., 1.2) node[midway, above] {\%reg (initial value)};
\draw[blue,decorate,decoration={brace,mirror},ultra thick] (-.2, -3.6) -- (13., -3.6) node[midway, below] {\%reg (final value)};
\end{visibleenv}
\foreach \x/\v in {28/1,29/0,30/1,31/1} {
    \node[anchor=south] at ($(0,0) + \x*(0.4, 0)$) {\tt \v};
    \begin{visibleenv}<3>
    \node[anchor=south] at ($(0,0.4) + \x*(0.4, 0)$) {\tt \v};
    \end{visibleenv}
}
\foreach \x/\v in {1/0,2/1,3/\strut\ldots,23/\strut\ldots,24/0,25/1,26/0,27/0} {
    \node[anchor=south] at ($(0,0) + \x*(0.4, 0)$) {\tt \v};
    \node[anchor=north] at ($(0, -2.5) + 4*(0.4,0) + \x*(0.4, 0)$) {\tt \v};
    \begin{visibleenv}<3>
    \node[anchor=south] at ($(0,0.4) + \x*(0.4, 0)$) {\tt \v};
    \node[anchor=north] at ($(0, -2.9) + 4*(0.4,0) + \x*(0.4, 0)$) {\tt \v};
    \end{visibleenv}
}
\foreach \x in {0,1,2,3,...,27} {
    \draw[simple wire,->] ($(0,0) + \x*(0.4, 0)$) -- ($(0,-2.5) + 4*(0.4, 0) + \x*(0.4, 0)$);
}
\begin{visibleenv}<1>
\foreach \x/\v in {0/0} {
    \node[blue,anchor=south] at ($(0,0) + \x*(0.4, 0)$) {\tt \v};
    \node[blue,anchor=north] at ($(0, -2.5) + 4*(0.4,0) + \x*(0.4, 0)$) {\tt \v};
}
\foreach \x in {0,1,2,3} {
    \draw[simple wire,->,blue] (0,0) -- ($(0,-2.5) + \x*(0.4, 0)$);
    \node[blue,anchor=north] at ($(0,-2.5) + \x*(0.4, 0)$) {\tt\bfseries 0};
}
\end{visibleenv}

\begin{visibleenv}<2>
\foreach \x/\v in {0/1} {
    \node[blue,anchor=south] at ($(0,0) + \x*(0.4, 0)$) {\tt \v};
    \node[blue,anchor=north] at ($(0, -2.5) + 4*(0.4,0) + \x*(0.4, 0)$) {\tt \v};
}
\foreach \x in {0,1,2,3} {
    \draw[simple wire,->,blue] (0,0) -- ($(0,-2.5) + \x*(0.4, 0)$);
    \node[blue,anchor=north] at ($(0,-2.5) + \x*(0.4, 0)$) {\tt\bfseries 1};
}
\end{visibleenv}

\begin{visibleenv}<3>
\foreach \x/\v in {0/0} {
    \node[blue,anchor=south] at ($(0,0.4) + \x*(0.4, 0)$) {\tt \v};
    \node[blue,anchor=north] at ($(0, -2.5) + 4*(0.4,0) + \x*(0.4, 0)$) {\tt \v};
}
\foreach \x/\v in {0/1} {
    \node[blue,anchor=south] at ($(0, 0.0) + \x*(0.4, 0)$) {\tt \v};
    \node[blue,anchor=north] at ($(0,-2.9) + 4*(0.4,0) + \x*(0.4, 0)$) {\tt \v};
}
\foreach \x in {0,1,2,3} {
    \draw[simple wire,->,blue] (0,0) -- ($(0,-2.5) + \x*(0.4, 0)$);
    \node[blue,anchor=north] at ($(0,-2.5) + \x*(0.4, 0)$) {\tt\bfseries 0};
    \node[blue,anchor=north] at ($(0,-2.9) + \x*(0.4, 0)$) {\tt\bfseries 1};
}
\end{visibleenv}
\end{tikzpicture}
\end{frame}


\begin{frame}[fragile,label=rightShiftC]{right shift in C}
\begin{ccodeNL}
int shift_signed(int x) {
    return x >> 5;
}
unsigned shift_unsigned(unsigned x) {
    return x >> 5;
}
\end{ccodeNL}
\hrule
\begin{tabular}{ll}
\begin{asmcodeNL}
shift_signed:
    movl %edi, %eax
    sarl $5, %eax
    ret
\end{asmcodeNL}
&
\begin{asmcodeNL}
shift_unsigned:
    movl %edi, %eax
    shrl $5, eax
    ret
\end{asmcodeNL}
\end{tabular}
\end{frame}

\begin{frame}{standards and shifts in C}
    \begin{itemize}
    \item signed right shift is \myemph{implementation-defined}
        \begin{itemize}
        \item standard lets compilers choose which type of shift to do
        \item all x86 compilers I know of --- arithmetic
        \end{itemize}
    \item \myemph<2>{we'll assume compiler decides arithmetic in this class}
    \vspace{.5cm}
    \item shift amount $\ge$ width of type: undefined
        \begin{itemize}
        \item x86 assembly: only uses lower bits of shift amount
        \end{itemize}
    \vspace{.5cm}
    \end{itemize}
\end{frame}

