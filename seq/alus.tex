\begin{frame}[fragile,label=ALUs]{ALUs}
    \tikzset{alu/.style={trapezium,
            trapezium angle=30,
            shape border rotate=270,
            minimum width=4cm,
            minimum height=3cm,
            trapezium stretches=true,
            append after command={%
                    \pgfextra
                        \draw (\tikzlastnode.top left corner) --
                           (\tikzlastnode.top right corner) -- 
                           (\tikzlastnode.bottom right corner) -- 
                           ($(\tikzlastnode.bottom right corner)!.666!(\tikzlastnode.bottom side)$)--
                           ([xshift=8mm]\tikzlastnode.bottom side)--
                           ($(\tikzlastnode.bottom side)!.334!(\tikzlastnode.bottom left corner)$)--
                           (\tikzlastnode.bottom left corner)--
                           (\tikzlastnode.top left corner);
                    \endpgfextra}}}
    \begin{tikzpicture}
        \node[alu] (alu) {ALU};
        \draw[a] (alu.east) -- ++(0:5mm) node[right] (outLabel) {{\color{blue}A} {\it OP} {\color{green}B}};
        \draw[a,latex-] (alu.130) -- ++(180:5mm) node[left] {\color{blue} A};
        \draw[a,latex-] (alu.230) -- ++(180:5mm) node[left] {\color{green} B};
        \draw[b,latex-] (alu.south) -- ++(-90:2cm) node[below] {operation select};
        \node[draw, rectangle,below right=1cm and .2cm of outLabel, align=left] {
            Operations needed: \\
            add --- \addq, addresses \\
            sub --- \subq \\
            xor --- \xorq \\
            and --- \andq \\
            more?
        };
    \end{tikzpicture}
\end{frame}

\begin{frame}{ALUs not for PC increment}
    \begin{itemize}
    \item our processor will have one ALU
    \vspace{.5cm}
    \item not used for PC increment (computing next instruction address)
        \begin{itemize}
        \item need to do other computation in same cycle
        \item don't need a general circuit for it
        \end{itemize}
    \end{itemize}
\end{frame}
