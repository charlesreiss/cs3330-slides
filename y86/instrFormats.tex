\usetikzlibrary{fit}

\begin{frame}[fragile,label=Y86format]{Y86-64 instruction formats}
\begin{tikzpicture}
\instrEncodingTable
\end{tikzpicture}
\end{frame}

\begin{frame}[fragile,label=Y86cc]{Secondary opcodes: {\tt cmov{\it cc}}/{\tt j{\it cc}}}
\begin{tikzpicture}
\tikzset{extra box/.style={opacity=0.5},
         extra box cc/.style={opacity=1.0}}
\instrEncodingTable
\tikzset{
    hilite/.style={draw,rectangle,rounded corners,line width=2pt,red!80!black}
}
\node[fit=(table-9-3),hilite] (jccHi) {};
\node[fit=(table-4-3),hilite] (movccHi) {};
\node[yshift=1cm,right=0cm of table-4-5.north east,anchor=north west,opacity=.95,fill=white,draw,rectangle,draw=blue!30!black,line width=2pt] (ccTable) {
    \def\arraystretch{1.4}
    \begin{tabular}{ll}
    \ccify{0} & {\it always} \small({\tt jmp}/{\tt rrmovq})\\
    \ccify{1} & {\tt le} \\
    \ccify{2} & {\tt l} \\ 
    \ccify{3} & {\tt e} \\ 
    \ccify{4} & {\tt ne} \\
    \ccify{5} & {\tt ge} \\
    \ccify{6} & {\tt g} \\
    \end{tabular}
};
\end{tikzpicture}
\end{frame}

\begin{frame}[fragile,label=Y86fn]{Secondary opcodes: {\keywordstyle {\it OP}q}}
\begin{tikzpicture}
\tikzset{extra box/.style={opacity=0.5},
         extra box fn/.style={opacity=1.0}}
\instrEncodingTable
\tikzset{
    hilite/.style={draw,rectangle,rounded corners,line width=2pt,red!80!black}
}
\node[fit=(table-8-3),hilite] (fnHi) {};
\node[yshift=1.5cm,right=0cm of table-8-5.north east,anchor=north west,opacity=.95,fill=white,draw,rectangle,draw=blue!30!black,line width=2pt] (fnTable) {
    \def\arraystretch{1.4}
    \begin{tabular}{ll}
    \fnify{0} & {\keywordstyle add} \\
    \fnify{1} & {\keywordstyle sub} \\
    \fnify{2} & {\keywordstyle and} \\ 
    \fnify{3} & {\keywordstyle xor} \\ 
    \end{tabular}
};
\end{tikzpicture}
\end{frame}

\begin{frame}[fragile,label=Y86reg]{Registers: \rA, \rB}
\begin{tikzpicture}
\tikzset{extra box/.style={opacity=0.6},
         extra box register/.style={opacity=1.0}}
\instrEncodingTable
\tikzset{
    hilite/.style={draw,rectangle,rounded corners,line width=2pt,red!80!black}
}
\node[hilite,fit=(table-4-4) (table-4-5) (table-8-4) (table-8-5)] {};
\node[hilite,fit=(table-12-4) (table-12-5) (table-13-4) (table-13-5)] {};
\node[yshift=1cm,right=1.25cm of table-2-3.north east,anchor=north west,opacity=.95,fill=white,draw,rectangle,draw=blue!30!black,line width=2pt] (fnTable) {
    \def\arraystretch{1.4}
    \tt
    \begin{tabular}{ll@{\hspace{1cm}}ll}
    \rnify{0} & \%rax & \rnify{8} & \%r8 \\
    \rnify{1} & \%rcx & \rnify{9} & \%r9 \\
    \rnify{2} & \%rdx & \rnify{A} & \%r10 \\ 
    \rnify{3} & \%rbx & \rnify{B} & \%r11 \\
    \rnify{4} & \%rsp & \rnify{C} & \%r12 \\
    \rnify{5} & \%rbp & \rnify{D} & \%r13 \\
    \rnify{6} & \%rsi & \rnify{E} & \%r14 \\
    \rnify{7} & \%rdi & \literalify{F} & \normalfont\it none\\
    \end{tabular}
};
\end{tikzpicture}
% FIXME: Why F in blank spots
% FIXME: Why rB not rA for irmovq
\end{frame}

\begin{frame}[fragile,label=Y86immed]{Immediates: \V, \D, \Dest}
\begin{tikzpicture}
\tikzset{
    extra box/.style={opacity=0.5},
    extra box immediate/.style={opacity=1.0},
    hilite/.style={draw,rectangle,rounded corners,line width=2pt,red!80!black},
}
\instrEncodingTable
\node[visible on=<1>,hilite,fit=(V-5) (D-6) (D-7)] {};
\node[visible on=<2>,hilite,fit=(Dest-9) (Dest-10)] {};
\end{tikzpicture}
\end{frame}

