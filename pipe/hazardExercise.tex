\begin{frame}[fragile,label=hazardExercise]{data hazard exericse}
\begin{tikzpicture}[overlay,remember picture]
\node[anchor=north east] at ([xshift=-.5cm,yshift=-1cm]current page.north east)
    {\resizebox{!}{0.25\textwidth}{\usebox{\pipeAddQBox}}};
\end{tikzpicture}
\vspace{-1cm}
\lstset{
    style=small,
    moredelim=**[is][\btHL<2|handout:0>]{@2@}{@},
    moredelim=**[is][\btHL<3|handout:0>]{@3@}{@},
    moredelim=**[is][\btHL<4|handout:0>]{@4@}{@},
    moredelim=**[is][\btHL<5|handout:0>]{@5@}{@},
}
\begin{lstlisting}
addq %r8, %r9 
addq %r10, %r11
addq %r9, %r8
addq %r11, %r10
\end{lstlisting}
    \vspace{2cm}
    \begin{itemize}
    \item to resolve data hazards with stalling, how many stalls are needed?
    \item hint: complete timeline
    \end{itemize}
\begin{tabular}{l|rrrrrrrr}
instr & cycle: 0 & 1 & 2 & 3 & 4 & 5 & 6 & 7 \\ \hline
addq R8, R9 & F & D & E & W \\
addq R10, R11 & ~ & F \\
addq R9, R8 \\
addq R11, R10 \\
\end{tabular}
\end{frame}

\iftoggle{heldback}{
\includecomment{forheldback}
\excludecomment{isheldback}
}{
\excludecomment{forheldback}
\includecomment{isheldback}
}

\begin{forheldback}
\begin{frame}[fragile,label=hazardExerciseNoSln]{data hazard exericse solution}
    (not on this version of the slides)
\end{frame}
\end{forheldback}
\begin{isheldback}
\begin{frame}[fragile,label=hazardExerciseSln]{data hazard exericse solution}
\begin{tikzpicture}[overlay,remember picture]
\node[anchor=north east] at ([xshift=-.5cm,yshift=-1cm]current page.north east)
    {\resizebox{!}{0.25\textwidth}{\usebox{\pipeAddQBox}}};
\end{tikzpicture}
\lstset{
    style=small,
    moredelim=**[is][\btHL<2|handout:0>]{@2@}{@},
    moredelim=**[is][\btHL<3|handout:0>]{@3@}{@},
    moredelim=**[is][\btHL<4|handout:0>]{@4@}{@},
    moredelim=**[is][\btHL<5|handout:0>]{@5@}{@},
}
\vspace{3cm}
~

\begin{tikzpicture}
    \tikzset{fetch/.style={fill=yellow!15},decode/.style={fill=orange!15},execute/.style={fill=green!15},writeback/.style={fill=blue!15}}
\matrix[tight matrix no line,
        nodes={text width=.6cm,font=\tt,minimum height=.7cm,align=center},
        column 1/.style={nodes={text width=5cm,align=left}}] (tbl) {
|[align=right]| \normalfont\textit{cycle \#} \hspace{.5cm}
                            \& 0 \& 1 \& 2 \& 3 \& 4 \& 5 \& 6 \& 7 \& 8 \\
    \lstinline|addq %r8, %r9|   \& |[fetch]| F \& |[decode]| D \& |[execute]| E \& |[writeback]| W \& ~ \& ~ \& ~ \& ~ \& ~ \\
    \lstinline|addq %r10, %r11|  \& ~ \& |[fetch]| F \& |[decode]| D \& |[execute]| E \& |[writeback]| W \& ~ \& ~ \\
    \lstinline|addq %r9, %r8|   \& ~ \& ~ \& ~ \& |[fetch]| F \& |[decode]| D \& |[execute]| E \& |[writeback]| W \\
    \lstinline|addq %r11, %r10| \& ~ \& ~ \& ~ \& ~ \& |[fetch]| F \& |[decode]| D \& |[execute]| E \& |[writeback]| W \& ~\\
};
\newcommand{\fwd}[4]{
    \pgfmathtruncatemacro{\idx}{#1+2}
    \draw[forwardLine,hiForwardOn=#4] ([xshift=-.05cm]tbl-#2-\idx.west) -- ([xshift=+.05cm]tbl-#3-\idx.west);
}
\tikzset{
    forwardLine/.style={red!90!black,thick,-Latex},
    hiForwardOn/.style={alt=#1{blue}{}},
    forwardLineB/.style={red,dashed,thick,-Latex},
    sepLine/.style={densely dotted,thick},
};

\foreach \x in {1,...,10} {
    \draw[sepLine] (tbl-1-\x.north east) -- (tbl-5-\x.south east);
}
\fwd{4}{2}{4}{<0|handout:0>}
\fwd{5}{3}{5}{<0|handout:0>}
    \node[draw,red,cross out,very thick] at (tbl-4-4) {~};
\end{tikzpicture}
\end{frame}
\end{isheldback}
