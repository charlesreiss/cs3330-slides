% FIXME:
% missing pieces
    % single-cycle memories
    % supporting multiple programs/operating systems
    % instruction scheduling?

\begin{frame}<1>[label=missingPieces]{missing pieces}
    \begin{itemize}
    \item \myemph<2>{multi-cycle memories}
    \item beyond pipelining: out-of-order, multiple issue
    \end{itemize}
\end{frame}

\againframe<2>{missingPieces}

\begin{frame}{multi-cycle memories}
    \begin{itemize}
    \item ideal case for memories: single-cycle
    \item achieved with \myemph{caches} (next topic)
        \begin{itemize}
        \item fast access to small number of things
        \end{itemize}
    \vspace{.5cm}
    \item typical performance:
        \begin{itemize}
        \item 90+\% of the time: single-cycle
        \end{itemize}
    \item sometimes many cycles (3--400+)
    \end{itemize}
\end{frame}

\begin{frame}[fragile,label=varSpeedMem]{variable speed memories}
\begin{tikzpicture}
\tikzset{
    forwardLine/.style={blue!60!black,thick,-Latex},
    hiForwardOn/.style={alt=#1{blue}{}},
    forwardLineB/.style={red,dashed,thick,-Latex},
    sepLine/.style={black!40,densely dotted},
};
    \tikzset{fetch/.style={fill=yellow!15},decode/.style={fill=orange!15},execute/.style={fill=green!15},
    memory/.style={fill=blue!15},writeback/.style={fill=violet!15}}
\newcommand{\fdemw}{ \& |[fetch]| F \& |[decode]| D \& |[execute]| E \& |[memory]| M \& |[writeback]| W}
\newcommand{\fdem}{ \& |[fetch]| F \& |[decode]| D \& |[execute]| E \& |[memory]| M }
\newcommand{\m}{ \& |[memory]| M }
\newcommand{\w}{ \& |[writeback]| W }
\newcommand{\dec}{ \& |[decode]| D } % conflicts with existing \d macro
\newcommand{\e}{ \& |[execute]| E }
\newcommand{\f}{ \& |[fetch]| F }
\matrix[tight matrix noline,
        nodes={text width=.75cm,font=\tt,minimum height=.7cm,align=center},
        column 1/.style={nodes={text width=5.8cm,align=left}},
        ] (tbl) {
|[align=right]| \normalfont\textit{cycle \#} \hspace{.5cm}
                                 \& 0 \& 1 \& 2 \& 3 \& 4 \& 5 \& 6 \& 7 \& 8 \\
~ \\
\lstinline|mrmovq 0(%rbx), %r8|       \fdemw \& ~ \& ~ \& ~ \\
\lstinline|mrmovq 0(%rcx), %r9|       \& ~ \fdemw \& ~ \& ~ \& ~ \\
    \lstinline|addq %r8, %r9|             \& ~ \& ~ \& |[fetch]| F \& |[decode]| D \& |[decode]| D \& |[execute]| E \& |[memory]| M \& |[writeback]| W \& ~ \& ~ \\
~       \\
~ \\
    \lstinline|mrmovq 0(%rbx), %r8|       \fdem \m \m \m \m \w \\
\lstinline|mrmovq 0(%rcx), %r9|       \& ~   \f   \dec  \e   \e   \e   \e   \e   \m   \m   \m   \m   \m \\
\lstinline|addq %r8, %r9|             \& ~ \& ~   \f   \dec  \dec  \dec  \dec  \dec  \dec  \dec  \dec  \dec  \dec\\
};
    \node[anchor=west,font=\it] at ([xshift=-.25cm]tbl-2-1.west) { memory is fast: (cache ``hit''; recently accessed?) };
    \node[anchor=west,font=\it] at ([xshift=-.25cm]tbl-7-1.west) { memory is slow: (cache ``miss'') };
\newcommand{\fwd}[4]{
    \pgfmathtruncatemacro{\idx}{#1+2}
    \draw[forwardLine,hiForwardOn=#4] ([xshift=-.15cm]tbl-#2-\idx.west) -- ([xshift=-.05cm]tbl-#3-\idx.west);
}
\newcommand{\fwdA}[4]{
    \pgfmathtruncatemacro{\idx}{#1+2}
    \draw[forwardLineB] ([xshift=-.15cm]tbl-#2-\idx.east) -- ([xshift=-.05cm]tbl-#3-\idx.east);
}

\newcommand{\fwdNo}[3]{
    \pgfmathtruncatemacro{\idx}{#1+2}
    \draw[forwardLineB] ([xshift=-.15cm]tbl-#2-\idx.west) -- ([xshift=-.05cm]tbl-#3-\idx.west);
}
\fwd{5}{3}{5}{<0>}
\fwd{5}{4}{5}{<0>}
\end{tikzpicture}
\end{frame}

\againframe<3>{missingPieces}

\begin{frame}[fragile,label=multiIssueIntro]{beyond pipelining: multiple issue}
    \begin{itemize}
    \item start \myemph{more than one instruction/cycle}
    \item multiple parallel pipelines; many-input/output register file
    \item \myemph{hazard handling much more complex}
    \end{itemize}
\begin{tikzpicture}
\tikzset{
    forwardLine/.style={blue!60!black,thick,-Latex},
    hiForwardOn/.style={alt=#1{blue}{}},
    forwardLineB/.style={red,dashed,thick,-Latex},
    sepLine/.style={black!40,densely dotted},
};
    \tikzset{fetch/.style={fill=yellow!15},decode/.style={fill=orange!15},execute/.style={fill=green!15},
    memory/.style={fill=blue!15},writeback/.style={fill=violet!15}}
\newcommand{\fdemw}{ \& |[fetch]| F \& |[decode]| D \& |[execute]| E \& |[memory]| M \& |[writeback]| W}
\newcommand{\fdem}{ \& |[fetch]| F \& |[decode]| D \& |[execute]| E \& |[memory]| M }
\newcommand{\m}{ \& |[memory]| M }
\newcommand{\w}{ \& |[writeback]| W }
\newcommand{\dec}{ \& |[decode]| D } % conflicts with existing \d macro
\newcommand{\e}{ \& |[execute]| E }
\newcommand{\f}{ \& |[fetch]| F }
\matrix[tight matrix noline,
        nodes={text width=.75cm,font=\tt,minimum height=.7cm,align=center},
        column 1/.style={nodes={text width=5.2cm,align=left}},
        ] (tbl) {
|[align=right]| \normalfont\textit{cycle \#} \hspace{.5cm}
                                 \& 0 \& 1 \& 2 \& 3 \& 4 \& 5 \& 6 \& 7 \& 8 \\
\lstinline|addq %r8, %r9|       \fdemw \& ~ \& ~ \& ~ \\
\lstinline|subq %r10, %r11|     \fdemw \& ~ \& ~ \& ~ \\
\lstinline|xorq %r9, %r11|      \& ~  \fdemw \& ~ \& ~ \& ~ \\
\lstinline|subq %r10, %rbx|     \& ~  \fdemw \& ~ \& ~ \& ~ \\
\ldots \\
};
\newcommand{\fwd}[4]{
    \pgfmathtruncatemacro{\idx}{#1+2}
    \draw[forwardLine,hiForwardOn=#4] ([xshift=-.15cm]tbl-#2-\idx.west) -- ([xshift=-.05cm]tbl-#3-\idx.west);
}
\newcommand{\fwdO}[4]{
    \pgfmathtruncatemacro{\idx}{#1+2}
    \draw[forwardLine,hiForwardOn=#4] ([xshift=-.05cm]tbl-#2-\idx.west) -- ([xshift=+.05cm]tbl-#3-\idx.west);
}
\newcommand{\fwdA}[4]{
    \pgfmathtruncatemacro{\idx}{#1+2}
    \draw[forwardLineB] ([xshift=-.15cm]tbl-#2-\idx.east) -- ([xshift=-.05cm]tbl-#3-\idx.east);
}

\newcommand{\fwdNo}[3]{
    \pgfmathtruncatemacro{\idx}{#1+2}
    \draw[forwardLineB] ([xshift=-.15cm]tbl-#2-\idx.west) -- ([xshift=-.05cm]tbl-#3-\idx.west);
}
\fwdO{3}{2}{4}{<0>}
\fwd{3}{3}{5}{<0>}
\fwd{3}{3}{4}{<0>}
\end{tikzpicture}
\end{frame}

\begin{frame}[fragile,label=oooIntro]{beyond pipelining: out-of-order}
    \begin{itemize}
    \item find \myemph{later instructions to do} instead of stalling
    \item lists of available instructions in pipeline registers
        \begin{itemize}
        \item take any instruction with available values
        \end{itemize}
    \item provide \myemph{illusion that work is still done in order}
        \begin{itemize}
        \item much more complicated hazard handling logic
        \end{itemize}
    \end{itemize}
\begin{tikzpicture}
\tikzset{
    forwardLine/.style={blue!60!black,thick,-Latex},
    hiForwardOn/.style={alt=#1{blue}{}},
    forwardLineB/.style={red,dashed,thick,-Latex},
    sepLine/.style={black!40,densely dotted},
};
    \tikzset{fetch/.style={fill=yellow!15},decode/.style={fill=orange!15},execute/.style={fill=green!15},
    memory/.style={fill=blue!15},writeback/.style={fill=violet!15}}
\newcommand{\fdemw}{ \& |[fetch]| F \& |[decode]| D \& |[execute]| E \& |[memory]| M \& |[writeback]| W}
\newcommand{\fdem}{ \& |[fetch]| F \& |[decode]| D \& |[execute]| E \& |[memory]| M }
\newcommand{\m}{ \& |[memory]| M }
\newcommand{\w}{ \& |[writeback]| W }
\newcommand{\dec}{ \& |[decode]| D } % conflicts with existing \d macro
\newcommand{\e}{ \& |[execute]| E }
\newcommand{\f}{ \& |[fetch]| F }
\matrix[tight matrix noline,
        nodes={text width=.75cm,font=\tt,minimum height=.7cm,align=center},
        column 1/.style={nodes={text width=5.8cm,align=left}},
        ] (tbl) {
|[align=right]| \normalfont\textit{cycle \#} \hspace{.5cm}
                                 \& 0 \& 1 \& 2 \& 3 \& 4 \& 5 \& 6 \& 7 \& 8 \\
\lstinline|mrmovq 0(%rbx), %r8| \fdem                  \m   \m   \w \\
\lstinline|subq %r8, %r9|       \& ~   \f \& ~ \& ~ \& ~  \& ~  \dec \e \w \\
    \lstinline|addq %r10, %r11|     \& ~ \& ~    \f   \dec  \e  \&~  \& ~ \& ~ \& ~ \w \\
    \lstinline|xorq %r12, %r13|     \& ~ \& ~ \& ~     \f   \dec \e \& ~ \& ~ \& ~ \& ~ \& \w \\
\ldots \\
};
\newcommand{\fwd}[4]{
    \pgfmathtruncatemacro{\idx}{#1+2}
    \draw[forwardLine,hiForwardOn=#4] ([xshift=-.15cm]tbl-#2-\idx.west) -- ([xshift=-.05cm]tbl-#3-\idx.west);
}
\newcommand{\fwdO}[4]{
    \pgfmathtruncatemacro{\idx}{#1+2}
    \draw[forwardLine,hiForwardOn=#4] ([xshift=-.05cm]tbl-#2-\idx.west) -- ([xshift=+.05cm]tbl-#3-\idx.west);
}
\newcommand{\fwdA}[4]{
    \pgfmathtruncatemacro{\idx}{#1+2}
    \draw[forwardLineB] ([xshift=-.15cm]tbl-#2-\idx.east) -- ([xshift=-.05cm]tbl-#3-\idx.east);
}

\newcommand{\fwdNo}[3]{
    \pgfmathtruncatemacro{\idx}{#1+2}
    \draw[forwardLineB] ([xshift=-.15cm]tbl-#2-\idx.west) -- ([xshift=-.05cm]tbl-#3-\idx.west);
}
\end{tikzpicture}
\end{frame}
