\begin{frame}[fragile,label=perfAssgn]{performance HWs}
    \begin{itemize}
        \item assignment 1: rotate an image
        \item assignment 2: smooth (blur) an image
            \begin{itemize}
            \item two parts
            \item part 1: due with rotate --- optimizations we've mostly talked about
            \item part 2: due later --- part with vector instructions
            \end{itemize}
    \end{itemize}
\end{frame}

\begin{frame}[fragile,label=imageRep]{image representation}
\begin{lstlisting}[language=C,style=small]
typedef struct { 
    unsigned char red, green, blue, alpha;
} pixel;
pixel *image = malloc(dim * dim * sizeof(pixel));

image[0]            // at (x=0, y=0)
image[4 * dim + 5]  // at (x=5, y=4)
...
\end{lstlisting}
\end{frame}

\begin{frame}[fragile,label=rotate]{rotate assignment}
\lstset{
    language=C,style=smaller,
}
    \vspace{-.25cm}
\begin{lstlisting}
void rotate(pixel *src, pixel *dst, int dim) {
    int i, j;
    for (i = 0; i < dim; i++)
        for (j = 0; j < dim; j++)
            dst[RIDX(dim - 1 - j, i, dim)] =
                src[RIDX(i, j, dim)];
}
\end{lstlisting}
    \begin{tikzpicture}
        \draw[thick] (0, 0) rectangle (4, -4);
        \node at (2, -2) {\includegraphics[width=4cm]{../optimization/example-pic}};
        \node at (7, -2) {\includegraphics[width=4cm,angle=90]{../optimization/example-pic}};
        \draw[thin,-Latex] (0.5, .5) --  (3.5, .5);
        \draw[thin,-Latex] (-0.5, -.5) --  (-0.5, -3.5);
        \draw[thick] (5, 0) rectangle (9, -4);
        \draw[thin,-Latex] (5.5, .5) --  (8.5, .5);
    \end{tikzpicture}
\end{frame}

\begin{frame}[fragile,label=MacrosExplain1]{preprocessor macros}
\lstset{
    language=C,style=small,
}
\begin{lstlisting}
#define DOUBLE(x) x*2

int y = DOUBLE(100);
// expands to:
int y = 100*2;
\end{lstlisting}
\end{frame}

\begin{frame}[fragile,label=MacrosExplain2]{macros are text substitution (1)}
\lstset{
    language=C,style=small,
}
\begin{lstlisting}
#define BAD_DOUBLE(x) x*2

int y = BAD_DOUBLE(3 + 3);
// expands to:
int y = 3+3*2;
// y == 9, not 12
\end{lstlisting}
\end{frame}


\begin{frame}[fragile,label=MacrosExplain3]{macros are text substitution (2)}
\lstset{
    language=C,style=small,
}
\begin{lstlisting}
#define FIXED_DOUBLE(x) (x)*2

int y = DOUBLE(3 + 3);
// expands to:
int y = (3+3)*2;
// y == 9, not 12
\end{lstlisting}
\end{frame}

\begin{frame}[fragile,label=RIDXExplain]{RIDX?}
\lstset{
    language=C,style=small,
}
\begin{lstlisting}
#define RIDX(x, y, n) ((x) * (n) + (y))

dst[RIDX(dim - 1 - j, 1, dim)]
// becomes *at compile-time*:
dst[((dim - 1 - j) * (dim) + (1))]
\end{lstlisting}
\end{frame}

\begin{frame}{performance grading}
    \begin{itemize}
    \item you can submit multiple variants in one file
        \begin{itemize}
            \item grade: best performance
            \item don't delete stuff that works!
        \end{itemize}
    \item we will measure speedup on \myemph{my machine}
        \begin{itemize}
        \item web viewer for results (with some delay --- has to run)
        \end{itemize}
    \item grade: achieving certain speedup on my machine
        \begin{itemize}
        \item thresholds based on results with certain optimizations
        \end{itemize}
    \end{itemize}
\end{frame}

\begin{frame}{general advice}
    \begin{itemize}
    \item (for when we don't give specific advice)
    \item try techniques from book/lecture that seem applicable
    \item vary numbers (e.g. cache block size)
        \begin{itemize}
        \item often --- too big/small is worse
        \end{itemize}
    \item some techniques combine well
        \begin{itemize}
        \item loop unrolling and cache blocking
        \item loop unrolling and reassociation/multiple accumulators
        \end{itemize}
    \end{itemize}
\end{frame}
