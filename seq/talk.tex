\graphicspath{{./figures/}}
\usepackage{xspace}
\usepackage{cancel}
\title{SEQ + HCL}
\date{}
\begin{document}

\begin{frame}
    \titlepage
\end{frame}

% FIXME: -q/--quiet
\usetikzlibrary{arrows.meta,chains,positioning,matrix,fit}


\ifdefined\NOBEAMER\else
\newcommand<>{\mathHL}[1]{%
\alt#2{
\text{\colorbox{green!20}{\ensuremath{#1}}}%
}{
\text{\colorbox{white!20}{\ensuremath{#1}}}%
}%
}
\fi

\makeatletter
\pgfdeclareshape{myregister}%
{
    \inheritsavedanchors[from=rectangle]
    \inheritanchorborder[from=rectangle]
    \inheritanchor[from=rectangle]{center}
    \inheritanchor[from=rectangle]{north}
    \inheritanchor[from=rectangle]{south}
    \inheritanchor[from=rectangle]{west}
    \inheritanchor[from=rectangle]{east}
    \inheritanchor[from=rectangle]{north east}
    \inheritanchor[from=rectangle]{north west}
    \inheritanchor[from=rectangle]{south east}
    \inheritanchor[from=rectangle]{south west}
    \inheritbackgroundpath[from=rectangle]
    \saveddimen{\halfbaselength}{%
         \pgf@x=0.5\wd\pgfnodeparttextbox
         % get xsep
         \pgfmathsetlength\pgf@xc{\pgfkeysvalueof{/pgf/inner xsep}}%
         \advance\pgf@x by \pgf@xc%
         % get \ht of textbox, add to baselength 
         \advance\pgf@x by \wd\pgfnodeparttextbox
         % get minimum width
         \pgfmathsetlength\pgf@xb{\pgfkeysvalueof{/pgf/minimum width}}%
         \divide\pgf@xb by 2
         \ifdim\pgf@x<\pgf@xb%
             % yes, too small. Enlarge...
             \pgf@x=\pgf@xb%
         \fi%
     }
    \backgroundpath{
        \pgfpathrectanglecorners{\southwest}{\northeast}
        \southwest \pgf@xa=\pgf@x \pgf@ya=\pgf@y 
        \pgf@yb=\pgf@ya
        \northeast \pgf@xb=\pgf@x %\pgf@yb=\pgf@y
        \pgf@xc = \pgf@xa
        \advance\pgf@xc by \halfbaselength
        \pgf@yc=\pgf@ya
        \advance\pgf@yc by \halfbaselength
        \pgfpathmoveto{\pgfpoint{\pgf@xa}{\pgf@ya}}
        \pgfpathlineto{\pgfpoint{\pgf@xc}{\pgf@yc}}
        \pgfpathlineto{\pgfpoint{\pgf@xb}{\pgf@yb}}
        \pgfpathclose
    }
}
\makeatother

\newcommand{\rA}{\it rA}
\newcommand{\rB}{\it rB}
\newcommand{\V}{\it V}
\newcommand{\D}{\it D}
\newcommand{\fn}{\it fn}
\newcommand{\Dest}{\it Dest}
\newcommand{\cc}{\it cc}
\tikzset{extra box/.style={},
         extra box opcode/.style={},
         extra box fn/.style={},
         extra box cc/.style={},
         extra box register/.style={},
         extra box immediate/.style={},
         extra box shorter width/.style={},
         extra box fake/.style={},
         }
\newcommand{\instrEncodingStyles}{
\tikzset{
    empty box/.style={text height=1ex,text depth=.4ex,font=\tt\fontsize{9}{10}\selectfont},
    box/.style={draw,rectangle,thick,empty box,extra box,align=center},
    opcode/.style={box,fill=blue!40!white,extra box opcode},
    secondOpcode/.style={box,fill=violet!40!white},
    secondOpcodeFN/.style={secondOpcode,extra box fn},
    secondOpcodeCC/.style={secondOpcode,extra box cc},
    literal/.style={box,fill=white!90!black},
    register/.style={box,fill=red!40!white,extra box register},
    fake/.style={empty box,pattern color=red!40!white,pattern=north west lines,inner sep=-1pt,extra box fake},
    immediate/.style={box,fill=green!40!white,extra box immediate},
    immediateLabel/.style={box,fill=green!40!white,extra box immediate,label={center:\fontsize{9}{10}\selectfont##1}},
}
}
\newcommand{\ccify}[2]{\begin{tikzpicture}[baseline]\node[anchor=base,secondOpcodeCC,text width=.35cm,inner xsep=0pt,inner sep=2pt,outer sep=0pt]{#1};\end{tikzpicture}}
\newcommand{\fnify}[1]{\begin{tikzpicture}[baseline]\node[anchor=base,secondOpcodeFN,text width=.35cm,inner xsep=0pt,inner sep=2pt,outer sep=0pt]{#1};\end{tikzpicture}}
\newcommand{\fnifyWide}[1]{\begin{tikzpicture}[baseline]\node[anchor=base,secondOpcodeFN,inner xsep=0pt,inner sep=2pt,outer sep=0pt,dashed]{#1};\end{tikzpicture}}
\newcommand{\opify}[1]{\begin{tikzpicture}[baseline]\node[anchor=base,opcode,text width=.35cm,inner xsep=0pt,inner sep=2pt,outer sep=0pt]{#1};\end{tikzpicture}}
\newcommand{\opifyWide}[1]{\begin{tikzpicture}[baseline]\node[anchor=base,opcode,inner xsep=0pt,inner sep=2pt,outer sep=0pt]{#1};\end{tikzpicture}}
\newcommand{\literalify}[1]{\begin{tikzpicture}[baseline]\node[anchor=base,literal,text width=.35cm,inner xsep=0pt,inner sep=2pt,outer sep=0pt]{#1};\end{tikzpicture}}
\newcommand{\immedify}[1]{\begin{tikzpicture}[baseline]\node[anchor=base,immediate,inner xsep=0pt,inner sep=2pt,outer sep=0pt]{#1};\end{tikzpicture}}
\newcommand{\rnify}[1]{\begin{tikzpicture}[baseline]\node[anchor=base,register,text width=.35cm,inner xsep=0pt,inner sep=2pt,outer sep=0pt]{#1};\end{tikzpicture}}
\newcommand{\rnifyWide}[1]{\begin{tikzpicture}[baseline]\node[anchor=base,register,inner xsep=0pt,inner sep=2pt,outer sep=0pt,dashed]{#1};\end{tikzpicture}}

\newcommand{\movq}{{\keywordstyle movq}\xspace}
\newcommand{\rmmovq}{{\keywordstyle rmmovq}\xspace}
\newcommand{\mrmovq}{{\keywordstyle mrmovq}\xspace}
\newcommand{\addq}{{\keywordstyle addq}\xspace}
\newcommand{\subq}{{\keywordstyle subq}\xspace}
\newcommand{\xorq}{{\keywordstyle xorq}\xspace}
\newcommand{\andq}{{\keywordstyle andq}\xspace}
\newcommand{\asmj}{{\keywordstyle jmp}\xspace}
\newcommand{\call}{{\keywordstyle call}\xspace}
\newcommand{\halt}{{\keywordstyle halt}\xspace}
\newcommand{\ret}{{\keywordstyle ret}\xspace}
\newcommand{\nop}{{\keywordstyle nop}\xspace}
\newcommand{\irmovq}{{\keywordstyle irmovq}\xspace}
\newcommand{\rrmovq}{{\keywordstyle rrmovq}\xspace}
\newcommand{\jmp}{{\keywordstyle jmp}\xspace}

\tikzset{
    hReg/.style={draw,myregister,minimum width=.4cm,minimum height=2cm,label={[font=\small,align=center]-90:#1}},
    hhReg/.style={draw,myregister,minimum width=.3cm,minimum height=5.5cm,label={[font=\small,align=center]-90:#1}},
    horizReg/.style={draw,myregister,rotate=-90,minimum width=.1cm,minimum height=1cm,label={[font=\small,align=center,fill=white]90:#1}},
    wReg/.style={draw,myregister,minimum width=2cm,minimum height=.4cm,label={[font=\small]-90:#1}},
    hRegSmall/.style={draw,myregister,minimum height=.6cm,minimum width=.2cm,label={[font=\small,inner sep=.5mm,align=center]-90:#1}},
    hRegT/.style={hRegSmall,minimum height=.4cm},
    mem/.style={draw,rectangle,minimum height=1.5cm,minimum width=1cm,inner sep=4pt,align=center,font=\small},
    memBig/.style={draw,rectangle,minimum height=3cm,minimum width=3cm,align=center,font=\small},
    regFile/.style={draw,rectangle,minimum height=4cm,minimum width=2cm,align=center,font=\small},
    ll/.style={font=\scriptsize},
    a/.style={-{Latex[length=5pt,width=3pt]},thick},
    aR/.style={{Latex[length=5pt,width=3pt]}-,thick},
    aN/.style={thick},
    aa/.style={-{Latex[length=5pt,width=3pt]},line width=1.2pt},
    aaR/.style={{Latex[length=5pt,width=3pt]}-,line width=1.2pt},
    aaN/.style={line width=1.2pt},
    b/.style={-{Latex[length=2pt,width=2pt]}},
    bN/.style={thin},
    bb/.style={line width=.5pt,-{Latex[length=2pt,width=2pt]}},
    bbR/.style={line width=.5pt,{Latex[length=2pt,width=2pt]}-},
    bR/.style={{Latex[length=2pt,width=2pt]}-},
    global scale/.style={scale=#1,every node/.style={scale=#1}},
    %logicBlock/.style={draw,cloud,cloud puffs=13.7,inner sep=0pt,cloud ignores aspect,align=center,draw},
    logicBlock/.style={draw,rectangle,inner sep=1pt,align=center,draw,fill=blue!20},
    logicBlockS/.style={draw,rectangle,inner sep=1pt,align=center,draw,fill=blue!20,font=\small},
    control/.style={dashed,color=blue!60},
    logicFill/.style={fill=blue!20},
    offsetBox/.style={align=left,font=\small,draw=blue!60!black,line width=2pt,rectangle}
}

\tikzset{
    bookLabel/.style={color=red!60!black,font=\small\bfseries,outer sep=0pt,inner sep=1pt,fill=white},
    imemPcPre/.style={invisible},
    imemPc/.style={},
    instrRegs/.style={},
    instrRegsPre/.style={invisible},
    instrRegsSplitOut/.style={invisible},
    instrRegsSplitImmed/.style={instrRegsSplitOut},
    instrRegsRS1/.style={instrRegs},
    instrRegsMux/.style={instrRegs},
    instrRegsMuxRS2/.style={instrRegsMux},
    instrRegsMuxRS3/.style={instrRegsMux},
    instrRegsMuxRS3F/.style={instrRegsMuxRS3},
    instrRegsMuxRS4/.style={instrRegsMux},
    instrRegsRS34Loop/.style={invisible},
    instrRegsNoMux/.style={invisible},
    instrRegsNoMuxRS2/.style={instrRegsNoMux},
    instrRegsNoMuxRS3/.style={instrRegsNoMux},
    instrRegsNoMuxRS4/.style={instrRegsNoMux},
    instrRegsPreSingle/.style={invisible},
    regsLogic/.style={},
    regsLogicMux/.style={regsLogic},
    regsLogicMuxA/.style={regsLogicMux},
    regsLogicMuxB/.style={regsLogicMux},
    regsLogicNoMux/.style={invisible},
    regsLogicNoMuxA/.style={regsLogicNoMux},
    regsLogicNoMuxB/.style={regsLogicNoMux},
    logicDmem/.style={},
    logicDmemMux/.style={logicDmem},
    logicDmemNoMux/.style={invisible},
    dmemWB/.style={},
    dmemWBFromMem/.style={dmemWB},
    dmemWBvalENoMux/.style={dmemWB},
    dmemWBvalELoop/.style={invisible},
    dmemWBvalEMux/.style={invisible},
    dmemOutToPC/.style={dmemWB},
    dmemPC/.style={},
    dmemPCMux/.style={dmemPC},
    dmemPCNoMux/.style={invisible},
    pcDecode/.style={},
    isStatReg/.style={hRegSmall=#1},
    isStat/.style={invisible},
    dmemNorm/.style={},
    dmemInputLabel/.style={dmemNorm},
    dmemLabel/.style={invisible},
    dmemPre/.style={invisible},
    dmemPreSingle/.style={invisible},
    regNorm/.style={},
    regNormLabel/.style={},
    regNormLabelE/.style={regNormLabel},
    regNormLabelM/.style={regNormLabel},
    regPre/.style={},
    regPreSingle/.style={},
    ccsNorm/.style={invisible},
    smallLabel/.style={font=\scriptsize,inner sep=1pt,outer sep=0pt},
    smallerLabel/.style={font=\tiny,inner sep=1pt,outer sep=0pt},
    pcStyle/.style={},
    wbPCLine/.style={},
    aluOpExplain/.style={regsLogic},
    funcOpExplain/.style={logicDmem},
    muxDst/.style={},
}
\newcommand{\instrEncodingSubTable}[3]{
\matrix[matrix of nodes,
    column sep=-2\pgflinewidth,
    row sep=2.5pt,
    nodes={empty box,text width=.35cm,inner xsep=0pt, inner sep=2pt,outer sep=0pt},
    column 1/.style={nodes={font=\tt\fontsize{9}{10}\selectfont,text width=3.5cm}},
    column 6/.style={nodes={extra box shorter width}},
    column 7/.style={nodes={extra box shorter width}},
    column 8/.style={nodes={extra box shorter width}},
    column 9/.style={nodes={extra box shorter width}},
    column 10/.style={nodes={extra box shorter width}},
    column 11/.style={nodes={extra box shorter width}},
    column 12/.style={nodes={extra box shorter width}},
    column 13/.style={nodes={extra box shorter width}},
    column 14/.style={nodes={extra box shorter width}},
    column 15/.style={nodes={text width=.2cm,extra box shorter width}},
    column 16/.style={nodes={text width=.2cm,extra box shorter width}},
    column 17/.style={nodes={text width=.2cm,extra box shorter width}},
    column 18/.style={nodes={text width=.2cm,extra box shorter width}},
    column 19/.style={nodes={text width=.2cm,extra box shorter width}},
    column 20/.style={nodes={text width=.2cm,extra box shorter width}},
    column 21/.style={nodes={text width=.2cm,extra box shorter width}},
#1,
] (#2) {#3}}

\newcommand{\var}[1]{\ensuremath{\text{#1}}}
\newcommand{\icode}{\var{icode}}
\newcommand{\ifun}{\var{ifun}}
\newcommand{\vrA}{\var{rA}}
\newcommand{\vrB}{\var{rB}}
\newcommand{\valP}{\var{valP}}
\newcommand{\valA}{\var{valA}}
\newcommand{\valB}{\var{valB}}
\newcommand{\valC}{\var{valC}}
\newcommand{\valE}{\var{valE}}
\newcommand{\valM}{\var{valM}}
\newcommand{\PC}{\var{PC}}
\newcommand{\srcA}{\var{srcA}}
\newcommand{\srcB}{\var{srcB}}
\newcommand{\dstE}{\var{dstE}}
\newcommand{\dstM}{\var{dstM}}
\newcommand{\aluA}{\var{aluA}}
\newcommand{\aluB}{\var{aluB}}
\newcommand{\pcMemDist}{2.cm}
\newcommand{\imemRegsDist}{3.5cm}
\newcommand{\regAluDist}{1.25cm}
\newcommand{\regMemDist}{3cm}
\newcommand{\regMuxDmemDist}{1.5cm}
\newcommand{\regReadOffset}{0cm}
\newcommand{\dstMuxDelta}{2.5mm}
\newcommand{\ilenOffset}{0cm}
\newcommand{\pcLabel}{PC}
\newcommand{\prePcDist}{2.5mm}
\newcommand{\regRegLabelDist}{1.5cm}
\newcommand{\regBDist}{.4cm}

\newcommand{\circuitStateToALU}{
        \node[hReg=\pcLabel,pcStyle] (pc) {};
        \node[mem,right=\pcMemDist of pc,font=\scriptsize] (imem) {Instr. \\ Mem.};
        \coordinate (imemData) at (imem.east);
        \coordinate (imemAddr) at (imem.west);
        \begin{scope}[regNorm]
            \node[regFile,right=\imemRegsDist of imem,label={[label distance=1pt,inner sep=0pt]\small register file}] (regs) {};
            \coordinate (regSelect1) at ($(regs.north west) - (0cm, .5cm)$);
            \coordinate (regSelect2) at ($(regs.north west) - (0cm, 1cm)$);
            \coordinate (regSelect3) at ($(regs.north west) - (0cm, 1.5cm)$);
            \coordinate (regSelect4) at ($(regs.north west) - (0cm, 2cm)$);
            \coordinate (regWriteIn1) at ($(regs.north west) - (0cm, 3.0cm)$);
            \coordinate (regWriteIn2) at ($(regs.north west) - (0cm, 3.5cm)$);
            \coordinate (regRead1) at ($(regs.north east) - (0cm, .4cm)$);
            \coordinate (regRead2) at ($(regs.north east) - (0cm, .4cm) - (0cm, \regBDist)$);
            %\node[ll,below left=0pt of regWriteIn,outer sep=1pt,inner sep=0pt] {data};
        \end{scope}
        \begin{scope}[regNormLabel]
            \node[smallLabel,right=0mm of regSelect1] (srcALabel) {\srcA};
            \node[smallLabel,right=0mm of regSelect2] (srcBLabel) {\srcB};
            \node[smallLabel,left=0mm of regRead1] {R[\srcA]};
            \node[smallLabel,left=0mm of regRead2] {R[\srcB]};
        \end{scope}[regNormLabel]

        \begin{scope}[regNormLabelE]
            \node[smallLabel,right=0mm of regSelect4] (dstELabel) {\dstE};
            \node[smallLabel,right=0mm of regWriteIn2] {next R[\dstE]};
        \end{scope}
        \begin{scope}[regNormLabelM]
            \node[smallLabel,right=0mm of regSelect3] (dstMLabel) {\dstM};
            \node[smallLabel,right=0mm of regWriteIn1] {next R[\dstM]};
        \end{scope}
}

\newcommand{\circuitState}{
        \circuitStateToALU
        \begin{scope}[dmemNorm]
            \node[mem,right=\regMemDist of regs,minimum width=1.3cm] (dmem) {};
            \node[dmemLabel,align=center] at (dmem) {Data\\Mem.};
            \coordinate (dmemIn) at (dmem.west);
            \coordinate (dmemInHigh) at ([yshift=.3cm]dmem.west);
            \coordinate (dmemInLow) at ([yshift=-.3cm]dmem.west);
            \coordinate (dmemDataOut) at (dmem.east);
        \end{scope}
        \begin{scope}[ccsNorm]
            \node[below=1cm of dmem,hRegSmall=ZF/SF] (ccs) {};
        \end{scope}
        \begin{scope}[isStat]
            \node[isStatReg=Stat,below=.25cm of dmem] (Stat) {};
        \end{scope}
}
\newcommand{\circuitStatePre}{
        \begin{scope}[imemPcPre]
            \draw[thick,latex-] (pc.west) -- +(-.5cm,0cm);
            \draw[thick,latex-] (imemAddr) -- +(-.3cm,0cm);
            \draw[thick,-latex] (pc.east) -- +(.3cm,0cm);
            \draw[thick,-latex] (imemData) -- +(.5cm,0cm);
        \end{scope}
        \begin{scope}[regPre]
            \draw[a,-latex,double] (regRead1) -- +(.5cm,0cm);
            \draw[thick,double,latex-] (regWriteIn1) -- +(-.5cm,0cm);
        \end{scope}
        \begin{scope}[regPreSingle]
            \draw[a,-latex] (regRead1) -- +(.5cm,0cm);
            \draw[thick,latex-] (regWriteIn1) -- +(-.5cm,0cm);
            \draw[a,-latex] (regRead2) -- +(.5cm,0cm);
            \draw[thick,latex-] (regWriteIn2) -- +(-.5cm,0cm);
        \end{scope}
        \begin{scope}[instrRegsPre]
            %\foreach \x in {regSelect1,regSelect2} {
            %    \draw[latex-] (\x) -- +(-.5cm,0cm);
            %}
            \draw[b,latex-,double] (regSelect1) -- +(-.5cm,0cm);
        \end{scope}
        \begin{scope}[instrRegsPreSingle]
            \foreach \x in {regSelect1,regSelect2,regSelect3,regSelect4} {
                \draw[latex-] (\x) -- +(-.5cm,0cm);
            }
        \end{scope}
        \begin{scope}[dmemPre]
            \draw[thick,-latex] (dmemDataOut) -- +(.5cm,0cm);
            \draw[thick,double,latex-] (dmemIn) -- +(-.5cm,0cm);
        \end{scope}
        \begin{scope}[dmemPreSingle]
            \draw[thick,-latex] (dmemDataOut) -- +(.5cm,0cm);
            \draw[thick,latex-] (dmemInHigh) -- +(-.5cm,0cm);
            \draw[thick,latex-] (dmemInLow) -- +(-.5cm,0cm);
        \end{scope}
}
\newcommand{\dmemInput}{
    \begin{scope}[dmemInputLabel]
        \node[smallLabel,right=0mm of dmemInHigh] {Data in};
        \node[smallLabel,right=0mm of dmemInLow] {Addr in};
        \node[smallLabel,left=0mm of dmemDataOut] {Data out};
    \end{scope}
}

\newcommand{\circuitConnectDetail}{
    \dmemInput

    % PC/IMEM
    \begin{scope}[imemPc]
        \draw[a] (pc.east) -- (imemAddr);
    \end{scope}

    % IMEM/REGISTERS
    \begin{scope}[instrRegs]
        \coordinate (split) at ([xshift=1cm]imemData);
        \coordinate (splitIcode) at ([xshift=.15cm]imemData);
        \coordinate (splitImmed) at ([xshift=.25cm]imemData);
        \coordinate (splitrA) at ([xshift=.5cm]imemData);
        \coordinate (splitrB) at ([xshift=.75cm]imemData);
        \draw[line width=1.5pt] (imemData) -- (splitIcode);
        \draw[line width=1.25pt] (imemData) -- (splitImmed);
        \coordinate (aboveRegFile) at ([yshift=.5cm]regs.north);
        \coordinate (furtherAboveRegFile) at ([yshift=.75cm]regs.north);

        \begin{scope}[instrRegsSplitOut]
            \draw[b] (splitrA) |- ([xshift=-1cm]regSelect1);
            \draw[b] (splitrB) |- ([xshift=-1cm]regSelect2);
        \end{scope}
        \begin{scope}[instrRegsSplitImmed]
            \draw[a] (splitImmed) |- (aboveRegFile) node[right,bookLabel] {valC};
        \end{scope}

        \begin{scope}[instrRegsRS1]
            \draw[b] (splitrA) |- (regSelect1);
        \end{scope}

        \begin{scope}[instrRegsNoMuxRS2]
           \draw[b] (splitrB) |- (regSelect2);
        \end{scope}
        \begin{scope}[instrRegsNoMuxRS3]
            \draw[b] (splitrA) |- (regSelect3);
        \end{scope}
        \begin{scope}[instrRegsNoMuxRS4]
            \draw[b] (splitrB) |- (regSelect4);
        \end{scope}

        \draw[line width=.75pt] (imemData) -- (splitrA);
        \draw[line width=0.5pt] (splitrA) -- (splitrB);
        
            \begin{scope}[instrRegsMuxRS3]
                \node[draw,minimum height=2cm,minimum width=1.25cm,left=\dstMuxDelta of regSelect3,mux,inputs={nn},global scale=0.25] (muxDstM) {};
                \draw[thin] (splitrA |- muxDstM.input 1) -- ([xshift=-2pt]splitrB |- muxDstM.input 1);
                \draw[b] ([xshift=2pt]splitrB |- muxDstM.input 1) -- (muxDstM.input 1);
                \draw[b] (muxDstM.output) -- (regSelect3);
                %\draw[bR] (muxDstM.input 3) -| (splitrB);
            \end{scope}
            \begin{scope}[instrRegsMuxRS3F]
                \draw[bR] (muxDstM.input 2) -- ++(180:2.5mm) node[left,inner sep=0pt,font=\tiny\tt] {0xF};
            \end{scope}
            \begin{scope}[instrRegsMuxRS4]
                \node[draw,minimum height=2cm,minimum width=1.25cm,left=\dstMuxDelta of regSelect4,mux,inputs={nnn},global scale=0.25] (muxDstE) {};
                \draw[b] (muxDstE.output) -- (regSelect4);
                \draw[bR] (muxDstE.input 2) -- ++(180:2.5mm) node[left,inner sep=0pt,font=\tiny\tt] {0xF};

                \draw[b] (splitrB) |- (muxDstE.input 1);
                \draw[bR] (muxDstE.input 3) -- ++(-.25cm,-.4mm) node[left,font=\tiny\tt,inner sep=0pt] {\%rsp};
            \end{scope}

            \begin{scope}[instrRegsMuxRS2]
                \node[draw,mux,minimum height=1.5cm,minimum width=0.5cm,global scale=0.25,inputs={nn},anchor=output,minimum height=1cm] (muxSrcB) at ([xshift=-.25cm]regSelect2) {};

                \draw[b] (muxSrcB.output) -- (regSelect2);
                \draw[b] (splitrB) |- (muxSrcB.input 1);
                \draw[bR] (muxSrcB.input 2) -- ++(-.25cm,-.25mm) node[left,font=\tiny\tt,inner sep=0pt] {\%rsp};
            \end{scope}

            \begin{scope}[instrRegsRS34Loop]
                \node[draw,minimum height=2cm,minimum width=1cm,mux,inputs={nnn},global scale=0.25,muxDst] (muxDstEAbove)
                    at ([yshift=1cm]regs.north) {};
                \node[draw,minimum height=2cm,minimum width=1cm,mux,inputs={nn},global scale=0.25,muxDst] (muxDstMAbove)
                    at ([yshift=1.5cm]regs.north) {};

                \coordinate (splitOffRBLoop) at ([xshift=-1cm]regSelect2 |- muxSrcB.input 1);
                \draw[bN] ([xshift=-1.25cm]regSelect1) |- ([xshift=-1.25cm,yshift=-2pt]regSelect1 |- aboveRegFile);
                \draw[b] ([xshift=-1.25cm,yshift=2pt]regSelect1 |- aboveRegFile) |- (muxDstMAbove.input 2);
                \draw[bN] (splitOffRBLoop) -- ([yshift=-2pt]splitOffRBLoop |- regSelect1);
                \draw[bN] ([yshift=2pt]splitOffRBLoop |- regSelect1) -- ([yshift=-2pt]splitOffRBLoop |- aboveRegFile);
                %\draw[b] ([yshift=2pt]splitOffRBLoop |- aboveRegFile) |- (muxDstMAbove.input 3);
                \draw[bR] (muxDstMAbove.input 1) -- ++(180:2.5mm) node[left,inner sep=0pt,font=\tiny\tt] {0xF};

                \draw[bR] (muxDstEAbove.input 1) -- ++(180:2.5mm) node[left,inner sep=0pt,font=\tiny\tt] {0xF};
                \draw[bR] (muxDstEAbove.input 2) -- ++(180:2.5mm) node[left,inner sep=0pt,font=\tiny\tt] {\%rsp};
                \draw[b] ([yshift=2pt]splitOffRBLoop |- aboveRegFile) |- (muxDstEAbove.input 3);

                \coordinate (dstMUpperRightCorner) at ([xshift=.5cm,yshift=2cm]dmem.north east);
                \coordinate (dstEUpperRightCorner) at ([xshift=.4cm,yshift=1.9cm]dmem.north east);
                \coordinate (dstMLowerRightCorner) at ([xshift=.5cm,yshift=-2.3cm]dmem.south east);
                \coordinate (dstELowerRightCorner) at ([xshift=.4cm,yshift=-2.2cm]dmem.south east);
                \coordinate (dstELeft) at ([xshift=-.75cm]regs.west);
                \coordinate (dstMLeft) at ([xshift=-1cm]regs.west);
                \coordinate (badLineHeight) at ([yshift=-.75cm]regs.south west);
                \draw[bN] (muxDstMAbove.output) -- ++(0.35cm, 0cm) |- (dstMUpperRightCorner) -- (dstMLowerRightCorner)
                    -| ([yshift=-2pt]dstMLeft |- badLineHeight);
                \draw[b] ([yshift=2pt]dstMLeft |- badLineHeight) |- (regSelect3);
                \draw[bN] (muxDstEAbove.output) -- ++(0.25cm, 0cm) |- (dstEUpperRightCorner) -- (dstELowerRightCorner)
                    -| ([yshift=-2pt]dstELeft |- badLineHeight);
                \draw[b] ([yshift=2pt]dstELeft |- badLineHeight) |- (regSelect4);
            \end{scope}
        
        \node[left=\regRegLabelDist of regSelect1,fill=white,font=\tiny,inner sep=1pt,outer sep=0pt] (vrALabel) {\vrA};
        \node[left=\regRegLabelDist of regSelect2,fill=white,font=\tiny,outer sep=0pt,inner sep=1pt] (vrBLabel) {\vrB};
    \end{scope}


    % REGISTERS/ALU
    % + ALU component
    \begin{scope}[regsLogic]
        \node[right=\regAluDist of regs,minimum height=1cm,minimum width=.75cm,logicBlock,label={[label distance=1pt,inner sep=0pt]\small ALU}] (alu) {};
        \coordinate (aluTop) at ([yshift=-2mm]alu.north west);
        \coordinate (aluBottom) at ([yshift=2mm]alu.south west);
        \node[smallerLabel,right=0mm of aluTop] {\aluA};
        \node[smallerLabel,right=0mm of aluBottom] {\aluB};
        \node[smallerLabel,left=0mm of alu.east] {\valE};
        \coordinate (afterAlu) at ([xshift=1mm]alu.east);

        \coordinate (regReadAfter1Pre) at ([xshift=4mm]regRead1);
        \coordinate (regReadAfter1) at ([xshift=\regReadOffset]regReadAfter1Pre);
        \coordinate (regReadAfter2Pre) at ([xshift=1mm]regRead2);
        \coordinate (regReadAfter2) at ([xshift=\regReadOffset]regReadAfter2Pre);
        
        \begin{scope}[regsLogicMuxA]
            \node[draw,mux,global scale=0.45,minimum height=1cm,left=2.5mm of aluTop,inputs={nnn}] (muxAluA) {};
            \draw[a] (muxAluA.output) -- (aluTop);
            \draw[a] (regRead1) -- (regReadAfter1) |- (muxAluA.input 2);
            \coordinate (dmemInBefore) at ([xshift=-2.5mm]dmemInHigh);
            \coordinate (beforeMux1) at ([xshift=-2.5mm]muxAluA.input 1);
            \coordinate (immedPreAlu) at ([xshift=-2.5mm,yshift=4pt] regRead1 -| muxAluA.input 1);
            \draw[aN] (splitImmed) |- (aboveRegFile) -| (immedPreAlu);
            \draw[a] ([xshift=-2.5mm,yshift=-4pt] regRead1 -| muxAluA.input 1) |- (muxAluA.input 1);
            \draw[bR] (muxAluA.input 3) -- ++(-.1cm,-.2cm) node[left,font=\tiny\tt,inner sep=0pt] {8};
        \end{scope}

        \begin{scope}[regsLogicMuxB]
            \node[draw,mux,minimum height=1.5cm,minimum width=.5cm,global scale=0.35,inputs={nn},anchor=output] (muxAluB) at ([xshift=-.5cm]aluBottom) {};
            \draw[a] (regRead2) -- (regReadAfter2) |- (muxAluB.input 1);
            \draw[aR] (muxAluB.input 2) -- ++(-.25cm,0cm) node[left,font=\tiny\tt,inner sep=0pt]{0};
            \draw[a] (muxAluB.output) -- (aluBottom);
        \end{scope}

        \begin{scope}[regsLogicNoMuxB]
            \draw[a] (regRead2) -- (regReadAfter2) |- (aluBottom);
        \end{scope}
        \begin{scope}[regsLogicNoMuxA]
            \draw[a] (regRead1) -- (regReadAfter1) |- (aluTop);
        \end{scope}

        % ALU Registers
        \draw[bR,aluOpExplain] (alu.south) -- ++(0,-2.5mm) node[below,inner sep=0pt,align=center,font=\tiny,aluOpExplain] (aluOpExplain) {add/sub\\ xor/and \\ (function \\ of instr.)};
    \end{scope}

    \begin{scope}[dmemPC]
        \node[dmemPCMux,draw,minimum height=2cm,left=.25cm of pc,mux,inputs={nnn},global scale=0.5] (muxPc) {};
    \end{scope}


    % ALU/MEMORY
    \begin{scope}[logicDmem]
        \node[smallerLabel,above=0mm of dmem.south] {write?};
        
        \draw[bR,funcOpExplain] (dmem.south) -- ++(0,-2.5mm) node[funcOpExplain,below,align=center,inner sep=0pt,font=\tiny] (funcOpExplain) {function\\of opcode};

        % MEMORY Registers
            % FIXME: correct input?
        \begin{scope}[logicDmemMux]
            \node[draw,mux,minimum height=1cm,global scale=0.45,left=5mm of dmemInLow,inputs={nn}] (muxDAddr) {};
            \draw[a] (muxDAddr.output) -- (dmemInLow);
            \draw[a] (regRead2) -- (regReadAfter2) |- ([yshift=-1mm,xshift=.5mm]alu.south east) |- (muxDAddr.input 2);
            \draw[a] (alu.east) -- (afterAlu) |- (muxDAddr.input 1);

            \node[draw,mux,minimum height=1cm,global scale=0.5,inputs={nn},anchor=input 2,minimum height=1cm] (muxDMem) at ([xshift=\regMuxDmemDist]regRead1) {};
            \draw[a] (muxDMem.output) -| (dmemInBefore) -- (dmemInHigh);
            \draw[a] (regRead1) -- (muxDMem.input 2);
            \coordinate (immedIntersect) at ([xshift=.5cm,yshift=2.5mm]pc.north east);
            %\draw[aN] (pc.east) -| ([yshift=-4pt]immedIntersect); 
            %\draw[a] ([yshift=4pt]immedIntersect) |- (furtherAboveRegFile) -| ([xshift=-2.5mm]muxDMem.input 1)
            %            -- (muxDMem.input 1);
            \draw[aR] (muxDMem.input 1) -- ++ (-.25cm, 0cm) -- ++ (0cm, .25cm) node[above,font=\scriptsize,inner sep=0.5mm] {PC+9};
        \end{scope}

        \begin{scope}[logicDmemNoMux]
            \draw[a] (alu.east) -- (afterAlu) |- (dmemInLow);
            \draw[a] (regRead1) -| ([xshift=-.5cm]dmemInHigh) -- (dmemInHigh);
        \end{scope}
    \end{scope}
    
    \begin{scope}[dmemWBvalENoMux]
        \draw[aN] (alu.east) -- (afterAlu) -- ([yshift=.025cm]afterAlu |- muxDAddr.input 2);
        \draw[a] ([yshift=-.025cm]afterAlu |- muxDAddr.input 2) |- ([yshift=-.25cm,xshift=-.25cm]regs.south west) |- (regWriteIn2);
    \end{scope}
    \begin{scope}[dmemWBvalELoop]
        \draw[aN] (alu.east) -- (afterAlu) -- ([yshift=.025cm]afterAlu |- muxDAddr.input 2);
        \draw[a] ([yshift=-.025cm]afterAlu |- muxDAddr.input 2) |- ([yshift=-.15cm]dmem.south west) |- ([yshift=-.25cm,xshift=-.25cm]regs.south west) |- (regWriteIn2);
    \end{scope}
    \begin{scope}[dmemWBvalEMux]
        \node[draw,mux,minimum height=1cm,global scale=0.5,inputs={nnn},rotate=180] (muxValE) at ([yshift=-.25cm, xshift=.1cm]regs.south east) {};
        \draw[a] (alu.east) -- (afterAlu) |- (muxValE.input 3);
        \draw[aN] (regRead1) -- (regReadAfter1) -- ([yshift=.05cm]regReadAfter1 |- aluBottom);
        \draw[aN] ([yshift=-.05cm]regReadAfter1 |- aluBottom) -- ([yshift=-.05cm]regReadAfter1 |- alu.south);
        \draw[a] ([yshift=-.15cm]regReadAfter1 |- alu.south) |- (muxValE.input 1);
        \draw[a] (muxValE.output) -- ([yshift=-.25cm,xshift=-.25cm]regs.south west) |- (regWriteIn2);
        \draw[aN] ([yshift=1cm]immedPreAlu) -- ([yshift=.05cm]immedPreAlu |- muxAluA.input 2);
        \draw[aN] ([yshift=-.05cm] immedPreAlu |- muxAluA.input 2) -- ([yshift=.05cm]immedPreAlu |- aluBottom);
        \draw[a] ([yshift=-.05cm] immedPreAlu |- aluBottom) |- (muxValE.input 2);
    \end{scope}
    \begin{scope}[dmemWB]
        \draw[a] (dmemDataOut) -- ++(2.5mm,0) |- ([yshift=-.5cm,xshift=-.5cm]regs.south west) |- (regWriteIn1);
    \end{scope}
    \begin{scope}[dmemOutToPC]
        \draw[a] (dmemDataOut) -- ++(2.5mm,0) |- ([yshift=-.5cm,xshift=-.5cm]regs.south west) --
            ++ (0cm,-.25cm) -| ([xshift=-3.5mm]muxPc.input 2) -- (muxPc.input 2);
    \end{scope}

    % PC update mux
    \begin{scope}[dmemPC]
        % above: \node[left=.25cm of pc,mux,inputs={nnn},global scale=0.5] (muxPc) {};
        \node[xshift=\ilenOffset,below=1cm of imem,logicBlock,font=\small] (iLen) {instr.\\length};
        \node[left=.25cm of iLen,logicBlock,font=\small] (iLenPlus) {+};
        
        \draw[b] ([xshift=\ilenOffset]splitIcode) |- (iLen.east);
        \draw[b] (iLen.west) -- (iLenPlus.east);
        \draw[a] (pc.east) -| (iLenPlus.north);

        \begin{scope}[dmemPCMux]
            \draw[a] (iLenPlus.west) -| ([xshift=-2.5mm]muxPc.input 3) -- (muxPc.input 3);

            \draw[a] (splitImmed) |- ([yshift=2.5mm]pc.north) -| ([xshift=-2.5mm]muxPc.input 1) -- (muxPc.input 1);

            \draw[a] (muxPc.output) -- (pc.west);
        \end{scope}
        
        \begin{scope}[dmemPCNoMux]
            \draw[a] (iLenPlus.west) -| ([xshift=-\prePcDist]pc.west) -- (pc.west);
        \end{scope}
    \end{scope}
    
}

\newcommand{\circuitConnect}{
        \begin{scope}[imemPc]
            \draw[a] (pc.east) -- (imemAddr);
        \end{scope}
        \begin{scope}[instrRegs]
            \node[logicBlock,right=1cm of imem,minimum height=6cm,minimum width=1cm] (decode) {logic};
            \draw[a] (imemData) -- (decode);
            \draw[b] (decode.east |- regSelect1) -- (regSelect1);
            \draw[b] (decode.east |- regSelect2) -- (regSelect2);
            \draw[b] (decode.east |- regSelect3) -- (regSelect3);
            \draw[b] (decode.east |- regSelect4) -- (regSelect4);
        \end{scope}
        \begin{scope}[regsLogic]
            \node[logicBlock,right=1cm of regs,minimum height=5.5cm,yshift=.25cm,minimum width=.1cm] (execute) {logic \\ (with\\ ALU)};
            \draw[a] (regRead1) -- (execute.west |- regRead1);
            \draw[a] (regRead2) -- (execute.west |- regRead2);
            \begin{scope}[ccsNorm]
                \draw[b] (ccs.east) -- (execute.west |- ccs.east);
            \end{scope}
        \end{scope}
        \begin{scope}[logicDmem]
            \draw[a, double] (execute.east |- dmemIn) -- (dmemIn);
            \draw[b,isStat] (execute.east |- Stat.west) -- (Stat.west);
            \coordinate (leftBelowExecute) at ($(execute.south east) + (.5cm,-.4cm)$);
            \coordinate (leftBelowExecute2) at ($(execute.south east) + (.75cm,-.5cm)$);
            \coordinate (leftBelowExecute3) at ($(execute.south east) + (.1cm,-.6cm)$);
            \coordinate (rightBelowExecute) at ($(execute.south east) + (-3cm,-.6cm)$);
            \coordinate (beforeReg1) at ($(regWriteIn1) + (-.125cm,0cm)$);
            \coordinate (beforeReg2) at ($(regWriteIn2) + (-.25cm,0cm)$);
            \coordinate (regWriteInMid) at ($(regWriteIn1)!0.5!(regWriteIn2)$);
            \coordinate (beforeRegMid) at ($(regWriteInMid) + (-.25cm,0cm)$);
            \coordinate (rightBelowRegs1) at (beforeReg1 |- leftBelowExecute2);
            \coordinate (rightBelowRegs2) at (beforeReg2 |- leftBelowExecute3);
            \coordinate (rightBelowRegsMid) at (beforeRegMid |- leftBelowExecute3);
            \begin{scope}[ccsNorm]
                \draw[b] ($(execute.south east) + (0cm, .25cm)$) -| (leftBelowExecute) -- (rightBelowExecute) |- (ccs.west);
            \end{scope}
            \coordinate (logicInstr) at ($(execute.north west) + (0cm, -.5cm)$);
            \draw[a,double] (decode.east |- logicInstr) -- (logicInstr);
        \end{scope}
        \begin{scope}[dmemWB]
            \node[logicBlock,right=.75cm of dmem,minimum height=4cm] (writeMux) {l\\o\\g\\i\\c};
            \coordinate (writeMuxTopIn) at ($(writeMux.north west) + (0cm,-0.5cm)$);
            \draw[a,double] (execute.east |- writeMuxTopIn) -- (writeMuxTopIn);
            \coordinate (writeMuxAfter) at ($(writeMux.east) + (.5cm, 0cm)$);
            \node[above=1pt of writeMuxAfter,font=\scriptsize,inner sep=1pt,outer sep=1pt,xshift=-.5ex] (toRegLabel) {to reg};
            \coordinate (rightBelowWriteMux) at (leftBelowExecute3 -| writeMuxAfter);
            \draw[a,double] (writeMux) -| (rightBelowWriteMux) --(leftBelowExecute3) |- (rightBelowRegsMid) |- (regWriteInMid);
        \end{scope}
        \begin{scope}[dmemWBFromMem]
            \draw[a] (dmem.east) -- (writeMux.west |- dmem.west);
        \end{scope}
        \begin{scope}[pcDecode]
            \coordinate (afterPC) at ($(pc.east) + (.25cm, 0cm)$);
            \draw[a] (afterPC) |- (decode.west |- logicInstr);
        \end{scope}
        \begin{scope}[dmemPC]
            %\draw[a] (dmem.east) -| (leftBelowExecute3) -- (rightBelowRegs2) |- ($(writeMux.east) + (0cm,.25cm)$);
            \coordinate (leftAboveWM) at ($(writeMux.north east) + (.5cm,1.5cm)$);
            \coordinate (aboveWM) at ($(writeMux.north) + (0cm,1.5cm)$);
            \coordinate (beforePC) at ($(pc.west) + (-.3cm, 0cm)$);
            \coordinate (rightAbovePC) at (beforePC |- leftAboveWM);
            \coordinate (leftAbovePC) at (beforePC |- leftAboveWM);
            \node[logicBlock,font=\tiny] (nextPC) at (afterPC |- aboveWM) {l\\o\\g\\i\\c};
            \coordinate (writeMuxTopOut) at ($(writeMux.north east) + (0cm,-.5cm)$);
            \draw[a,wbPCLine] (writeMuxTopOut) -| (leftAboveWM) -- (nextPC.east |- leftAboveWM);
            \node[below right=0pt of writeMuxTopOut,font=\scriptsize,inner sep=1pt,outer sep=1pt] {to PC};
            \draw[a] (afterPC) -- (nextPC);
            \draw[overlay,a] (nextPC.west |- rightAbovePC) -- (rightAbovePC) -- (beforePC) -- (pc);
        \end{scope}
}

\newcommand{\circuitLayout}{
        \circuitState
        \circuitStatePre
        \circuitConnect
}



\newcommand{\assign}{\ensuremath{\leftarrow}}

\newcommand{\instrEncodingTable}{
\instrEncodingStyles
\matrix[matrix of nodes,
    column sep=-2\pgflinewidth,
    row sep=2.5pt,
    nodes={empty box,text width=.35cm,inner xsep=0pt, inner sep=2pt,outer sep=0pt},
    column 1/.style={nodes={font=\tt\fontsize{9}{10}\selectfont,text width=3.5cm}},
    column 6/.style={nodes={extra box shorter width}},
    column 7/.style={nodes={extra box shorter width}},
    column 8/.style={nodes={extra box shorter width}},
    column 9/.style={nodes={extra box shorter width}},
    column 10/.style={nodes={extra box shorter width}},
    column 11/.style={nodes={extra box shorter width}},
    column 12/.style={nodes={extra box shorter width}},
    column 13/.style={nodes={extra box shorter width}},
    column 14/.style={nodes={extra box shorter width}},
    column 15/.style={nodes={text width=.2cm,extra box shorter width}},
    column 16/.style={nodes={text width=.2cm,extra box shorter width}},
    column 17/.style={nodes={text width=.2cm,extra box shorter width}},
    column 18/.style={nodes={text width=.2cm,extra box shorter width}},
    column 19/.style={nodes={text width=.2cm,extra box shorter width}},
    column 20/.style={nodes={text width=.2cm,extra box shorter width}},
    column 21/.style={nodes={text width=.2cm,extra box shorter width}},
] (table) {
    % row 1
    \bf byte: \& 0 \& ~ \& 1 \& ~ \& 2 \& ~ \& 3 \& ~ \& 4 \& ~ \& 5 \& ~ \& 6 \& ~ \& 7 \& ~ \& 8 \& ~ \& 9 \& ~ \\
    % row 2
    \halt     \& |[opcode]| 0 \& |[literal]| 0 \& ~ \& ~ \& ~ \& ~ \\
    % row 3
    \nop      \& |[opcode]| 1 \& |[literal]| 0 \\
    % row 4
    \rrmovq/{\keywordstyle cmovCC} \rA, \rB \& |[opcode]| 2 \& |[secondOpcodeCC]| \cc \& |[register]| \rA \& |[register]| \rB \\
    % row 5
    \irmovq \V, \rB \& |[opcode]| 3 \& |[literal]| 0 \& |[literal]| F \& |[register]| \rB 
        \& ~ \& ~ \& ~ \& ~ 
        \& ~ \& ~ \& ~ \& ~ 
        \& ~ \& ~ \& ~ \& ~
        \& ~ \& ~ \& ~ \& ~
    \\
    % row 6
    \rmmovq \rA, \D(\rB) \& |[opcode]| 4 \& |[literal]| 0 \& |[register]| \rA \& |[register]| \rB
        \& ~ \& ~ \& ~ \& ~ 
        \& ~ \& ~ \& ~ \& ~ 
        \& ~ \& ~ \& ~ \& ~
        \& ~ \& ~ \& ~ \& ~
    \\
    % row 7
    \mrmovq \D(\rB), \rA \& |[opcode]| 5 \& |[literal]| 0 \& |[register]| \rA \& |[register]| \rB
        \& ~ \& ~ \& ~ \& ~ 
        \& ~ \& ~ \& ~ \& ~ 
        \& ~ \& ~ \& ~ \& ~
        \& ~ \& ~ \& ~ \& ~
    \\
    % row 8
    {\keywordstyle {\it OP}q} \rA, \rB \& |[opcode]| 6 \& |[secondOpcodeFN]| \fn \& |[register]| \rA \& |[register]| \rB \\
    % row 9
    {\keywordstyle j{\it CC}} \Dest \& |[opcode]| 7 \& |[secondOpcodeCC]| \cc 
        \& ~ \& ~ \& ~ \& ~ 
        \& ~ \& ~ \& ~ \& ~ 
        \& ~ \& ~ \& ~ \& ~
        \& ~ \& ~ \& ~ \& ~
    \\
    % row 10
    {\keywordstyle call} \Dest \& |[opcode]| 8 \& |[literal]| 0 
        \& ~ \& ~ \& ~ \& ~ 
        \& ~ \& ~ \& ~ \& ~ 
        \& ~ \& ~ \& ~ \& ~
        \& ~ \& ~ \& ~ \& ~
    \\
    {\keywordstyle ret} \& |[opcode]| 9 \& |[literal]| 0 \\
    {\keywordstyle pushq} \rA \& |[opcode]| A \& |[literal]| 0 \& |[register]| \rA \& |[literal]| F \\
    {\keywordstyle popq} \rA \& |[opcode]| B \& |[literal]| 0 \& |[register]| \rA \& |[literal]| F \\
};
\foreach \x in {5} {
    \node[immediate,inner sep=0pt,outer sep=0pt,fit=(table-\x-6) (table-\x-21)] (V-\x) {\V};
}
\foreach \x in {6,7} {
    \node[immediate,inner sep=0pt,outer sep=0pt,fit=(table-\x-6) (table-\x-21)] (D-\x) {\D};
}
\foreach \x in {9,10} {
    \node[immediate,inner sep=0pt,outer sep=0pt,fit=(table-\x-4) (table-\x-19)] (Dest-\x) {\Dest};
}
}

\newcommand{\instrEncodingTableReversed}{
\instrEncodingStyles
\matrix[matrix of nodes,
    column sep=-2\pgflinewidth,
    row sep=2.5pt,
    nodes={empty box,text width=.35cm,inner xsep=0pt, inner sep=2pt,outer sep=0pt},
    column 1/.style={nodes={font=\tt\fontsize{9}{10}\selectfont,text width=3.5cm}},
    column 17/.style={nodes={extra box shorter width}},
    column 16/.style={nodes={extra box shorter width}},
    column 15/.style={nodes={extra box shorter width}},
    column 14/.style={nodes={extra box shorter width}},
    column 13/.style={nodes={extra box shorter width}},
    column 12/.style={nodes={extra box shorter width}},
    column 11/.style={nodes={extra box shorter width}},
    column 10/.style={nodes={extra box shorter width}},
    column 9/.style={nodes={extra box shorter width}},
    column 8/.style={nodes={text width=.2cm,extra box shorter width}},
    column 7/.style={nodes={text width=.2cm,extra box shorter width}},
    column 6/.style={nodes={text width=.2cm,extra box shorter width}},
    column 5/.style={nodes={text width=.2cm,extra box shorter width}},
    column 4/.style={nodes={text width=.2cm,extra box shorter width}},
    column 3/.style={nodes={text width=.2cm,extra box shorter width}},
    column 2/.style={nodes={text width=.2cm,extra box shorter width}},
] (table) {
    % row 1
    \bf byte: \& 9 \& ~ \& 8 \& ~ \& 7 \& ~ \& 6 \& ~ \& 5 \& ~ \& 4 \& ~ \& 3 \& ~ \& 2 \& ~ \& 1 \& ~ \& 0 \& ~ \\
    % row 2
    \halt     \& 
           ~ \& ~ \& ~ \& ~ 
        \& ~ \& ~ \& ~ \& ~ 
        \& ~ \& ~ \& ~ \& ~
        \& ~ \& ~ \& ~ \& ~ 
        \& ~ \& ~
        \&  |[opcode]| 0 \& |[literal]| 0 \\
    % row 3
    \nop      \& 
           ~ \& ~ \& ~ \& ~ 
        \& ~ \& ~ \& ~ \& ~ 
        \& ~ \& ~ \& ~ \& ~
        \& ~ \& ~ \& ~ \& ~ 
        \& ~ \& ~ \&
        |[opcode]| 1 \& |[literal]| 0 \\
    % row 4
    \rrmovq/{\keywordstyle cmovCC} \rA, \rB \& 
           ~ \& ~ \& ~ \& ~ 
        \& ~ \& ~ \& ~ \& ~ 
        \& ~ \& ~ \& ~ \& ~
        \& ~ \& ~ \& ~ \& ~ 
        \&
    |[register]| \rA \& |[register]| \rB \&
    |[opcode]| 2 \& |[secondOpcodeCC]| \cc \\
    % row 5
    \irmovq \V, \rB \& 
           ~ \& ~ \& ~ \& ~ 
        \& ~ \& ~ \& ~ \& ~ 
        \& ~ \& ~ \& ~ \& ~
        \& ~ \& ~ \& ~ \& ~ \&
        |[literal]| F \& |[register]| \rB  \&
        |[opcode]| 3 \& |[literal]| 0
    \\
    % row 6
    \rmmovq \rA, \D(\rB) \& 
           ~ \& ~ \& ~ \& ~ 
        \& ~ \& ~ \& ~ \& ~ 
        \& ~ \& ~ \& ~ \& ~
        \& ~ \& ~ \& ~ \& ~ \&
    |[register]| \rA \& |[register]| \rB \&
        |[opcode]| 4 \& |[literal]| 0
    \\
    % row 7
    \mrmovq \D(\rB), \rA \& 
           ~ \& ~ \& ~ \& ~ 
        \& ~ \& ~ \& ~ \& ~ 
        \& ~ \& ~ \& ~ \& ~
        \& ~ \& ~ \& ~ \& ~ \&
    |[register]| \rA \& |[register]| \rB \&
    |[opcode]| 5 \& |[literal]| 0
    \\
    % row 8
    {\keywordstyle {\it OP}q} \rA, \rB \& 
           ~ \& ~ \& ~ \& ~ 
        \& ~ \& ~ \& ~ \& ~ 
        \& ~ \& ~ \& ~ \& ~
        \& ~ \& ~ \& ~ \& ~ \&
    |[register]| \rA \& |[register]| \rB \&
    |[opcode]| 6 \& |[secondOpcodeFN]| \fn \\
    % row 9
    {\keywordstyle j{\it CC}} \Dest \& 
           ~ \& ~ \& ~ \& ~ 
        \& ~ \& ~ \& ~ \& ~ 
        \& ~ \& ~ \& ~ \& ~
        \& ~ \& ~ \& ~ \& ~ \& ~ \& ~ \&
    |[opcode]| 7 \& |[secondOpcodeCC]| \cc 
    \\
    % row 10
    {\keywordstyle call} \Dest \& 
           ~ \& ~ \& ~ \& ~ 
        \& ~ \& ~ \& ~ \& ~ 
        \& ~ \& ~ \& ~ \& ~
        \& ~ \& ~ \& ~ \& ~ \& ~ \& ~ \&
    |[opcode]| 8 \& |[literal]| 0 
    \\
    {\keywordstyle ret} \& 
           ~ \& ~ \& ~ \& ~ 
        \& ~ \& ~ \& ~ \& ~ 
        \& ~ \& ~ \& ~ \& ~
        \& ~ \& ~ \& ~ \& ~ \& ~ \& ~ \&
    |[opcode]| 9 \& |[literal]| 0 \\
    {\keywordstyle pushq} \rA \& 
           ~ \& ~ \& ~ \& ~ 
        \& ~ \& ~ \& ~ \& ~ 
        \& ~ \& ~ \& ~ \& ~
        \& ~ \& ~ \& ~ \& ~ \&
    |[register]| \rA \& |[literal]| F \&
    |[opcode]| A \& |[literal]| 0 \\
    {\keywordstyle popq} \rA \& 
           ~ \& ~ \& ~ \& ~ 
        \& ~ \& ~ \& ~ \& ~ 
        \& ~ \& ~ \& ~ \& ~
        \& ~ \& ~ \& ~ \& ~ 
    \& |[register]| \rA \& |[literal]| F \&
    |[opcode]| B \& |[literal]| 0  \\
};
\foreach \x in {5} {
    \node[immediate,inner sep=0pt,outer sep=0pt,fit=(table-\x-2) (table-\x-17)] (V-\x) {\V};
}
\foreach \x in {6,7} {
    \node[immediate,inner sep=0pt,outer sep=0pt,fit=(table-\x-2) (table-\x-17)] (D-\x) {\D};
}
\foreach \x in {9,10} {
    \node[immediate,inner sep=0pt,outer sep=0pt,fit=(table-\x-4) (table-\x-19)] (Dest-\x) {\Dest};
}
}


\usepgflibrary{shapes.gates.logic.mux}
\usetikzlibrary{calc,chains,shapes.misc,shapes.callouts,shapes.geometric,shapes.gates.logic.US,circuits.logic.US}
\providecommand{\vemphA}[1]{\myemph<2>{#1}}
\providecommand{\vemphB}[1]{\myemph<3>{#1}}
\providecommand{\vemphC}[1]{\myemph<4>{#1}}

\section{HCLs, generally}
\tikzset{wire/.style={draw,thick,>=Latex}}

\begin{frame}{describing hardware}
    \begin{itemize}
    \item how do we describe hardware?
    \item pictures?
    \end{itemize}
    \begin{tikzpicture}
        \node[draw,minimum width=1cm,minimum height=1cm,fill=blue!20] (add1) at (3, 2) { add 1 };
        \node[hRegSmall={count}] (reg) at (4.5, 2) {};
        \draw[wire,->] (add1) -- (reg);
        \draw[wire,->] (reg) -- (6, 2) -- (6, 0) -- (0, 0) -- (0, 2) -- (add1);
        \coordinate (pinPoint) at (6, 1);
    \end{tikzpicture}
\end{frame}

\begin{frame}{circuits with pictures?}
    \begin{itemize}
    \item yes, something you can do
    \item such commercial tools exist, but\ldots
    \vspace{.5cm}
    \item not commonly used for processors
    \end{itemize}
\end{frame}

\begin{frame}{hardware description language}
    \begin{itemize}
    \item \myemph{programming language for hardware}
    \item (typically) text-based representation of circuit
    \vspace{.5cm}
    \item often abstracts away details like:
        \begin{itemize}
        \item how to build arithmetic operations from gates
        \item how to build registers from transistors
        \item how to build memories from transistors
        \item how to build MUXes from gates
        \item \ldots
        \end{itemize}
    \item those details also not a topic in this course
    \end{itemize}
\end{frame}

\begin{frame}{our tool: HCLRS}
    \begin{itemize}
    \item built for this course
    \item assumes you're making a processor
    \vspace{.5cm}
    \item somewhat different from textbook's HCL
    \end{itemize}
\end{frame}



\section{wires}

\subsection{abstractly}
\usetikzlibrary{chains}

\tikzset{
    simple wire/.style={very thick,>=Latex},
    wire/.style={line width=1.5pt,>=Latex},
    wire small/.style={line width=1pt,>=Latex},
    binLabel/.style={font=\tt},
    hiBox/.style={red, very thick, draw}
}

\begin{frame}{circuits: wires}
\begin{tikzpicture}
\foreach \x/\v in {0/0,0.4/1,0.8/0,1.2/1,1.6/1} {
    \draw[simple wire] (0, \x) node[left,binLabel] (\x-\v-num) {\v} -- (10, \x) node[right,binLabel]{\v};
};
\onslide<2->{
    \node[hiBox,inner sep=1mm,fit=(0-0-num),label={south east:binary value --- actually voltage}] {};
};
\onslide<3->{
    \draw[ultra thick,blue,dashed] (1, 0) -- (9, 0) 
        node[midway,below] {value propagates to rest of wire (small delay)};
}
\end{tikzpicture}
\end{frame}

\begin{frame}{circuits: wire bundles}
\begin{tikzpicture}
\foreach \x/\v in {0/0,0.4/1,0.8/0,1.2/1,1.6/1} {
    \draw[simple wire] (0, \x) node[left] (\x-\v-num) {\tt\v} -- (10, \x) node[right]{\tt\v};
};
\node[anchor=center] at (5,-.5) {{\tt 11010} = 26};
\begin{visibleenv}<2->
    \node[anchor=center] at (3, 2) {same as};
    \foreach \x in {0,0.1,0.2,0.3,0.4} {
        \draw[simple wire] ($(0, \x) + (0, 3)$) -- ($(10, \x) + (0, 3)$);
    };
    \node[anchor=east] at (0, 3.2) {26};
    \node[anchor=west] at (10, 3.2) {26};
\end{visibleenv}
\begin{visibleenv}<3->
    \node[anchor=center] at (3, 4) {same as};
    \draw[wire] (0, 6) node[left] {26} -- (10, 6) node[right] {26};
\end{visibleenv}
\begin{visibleenv}<4-5>
    \draw[alt=<4>{red},ultra thick] (5, 6) ++ (+0.05, .1) -- ++(-0.1, -.2) node[below] (bit count) {5};
\end{visibleenv}
\begin{visibleenv}<4>
    \draw[very thick,<-] (bit count.east) -- ++ (1, -.5cm) node[right,align=left] {
        explicit marker for 5-bit wire bundle \\
        often omitted to avoid clutter
    };
\end{visibleenv}
\end{tikzpicture}
\end{frame}

\begin{frame}{circuits: gates}
    \begin{tikzpicture}[circuit logic US]
        \begin{scope}[start chain=inputs going below, every node/.style={on chain}, node distance=.3cm];
            \node {0};
            \node {0};

            \node {0};
            \node {1};

            \node {1};
            \node {0};

            \node {1};
            \node {1};
        \end{scope}
        \foreach \x/\y/\v in {1/2/0,3/4/1,5/6/1,7/8/0} {
            \node[xor gate] (xor-\x) at ([xshift=3cm,yshift=-.5cm]inputs-\x) {};
            \draw[wire] (inputs-\x) -- ++(.5cm,0cm) |- (xor-\x.input 1);
            \draw[wire] (inputs-\y) -- ++(.5cm,0cm) |- (xor-\x.input 2);
            \draw[wire] (xor-\x.output) -- ++(1cm,0cm) node[right] {\v};
        }
    \end{tikzpicture}
\end{frame}

\begin{frame}{circuits: logic}
    \begin{itemize}
    \item want to do calculations?
    \item generalize gates:
    \item<2-> output wires contain result of function on input
        \begin{itemize}
        \item changes as input changes (with delay)
        \end{itemize}
    \item<3-> need not be same width as output
    \end{itemize}
\begin{tikzpicture}
    \node[fill=blue!20] (theLogic) at (0, -1) {``logic''};
    \draw[wire] (0, 0) node[above] {12} -- (theLogic);
    \draw[wire small,alt=<3>{red}{}] (theLogic) -- (0, -2) node[left,xshift=.4cm, yshift=-.2cm] (funcValue) {function(12) = ??};
\end{tikzpicture}
\end{frame}

 % FIXME: split out logic part to interleave better with HCLRS stuff

\subsection{in HCLRS}
\usetikzlibrary{arrows.meta,calc,decorations.pathreplacing}
\usetikzlibrary{circuits.logic.US}

\tikzset{
    simple wire/.style={very thick,>=Latex},
    wire/.style={line width=1.5pt,>=Latex},
    wire thick/.style={line width=2.5pt,>=Latex},
    wire small/.style={line width=1pt,>=Latex},
    binLabel/.style={font=\tt},
    hiBox/.style={red, very thick, draw}
}

\begin{frame}{HCLRS: wire (bundle)s}
\begin{tikzpicture}
\foreach \x/\v in {0/0,0.4/1,0.8/0,1.2/1,1.6/1} {
    \draw[simple wire] (0, \x) node[left,binLabel] (\x-\v-num) {\v} -- (10, \x) node[right,binLabel]{\v};
};
\begin{scope}[yshift=-1.4cm]
\foreach \x in {0,0.1,0.2,0.3,0.4} {
    \draw[simple wire] ($(0, \x) + (0, 0)$) -- ($(10, \x) + (0, 0)$);
};
\node[anchor=east] at (0, 0.2) {26};
\node[anchor=west] at (10, 0.2) {26};
\end{scope}
\begin{scope}[yshift=-2.2cm]
\draw[wire thick] (0, 0) node[left] {26} -- ++(10, 0) node[right] {26};
\end{scope}
\end{tikzpicture}
\begin{itemize}
\item \texttt{wire \tikzmark{startName1a}foo\tikzmark{endName1a} : \tikzmark{startWidth1a}5\tikzmark{endWidth2a}; \tikzmark{startAssign1a}foo = 0b11010\tikzmark{endAssign1a};} \hspace{0.7cm} \textit{OR}
\item \texttt{wire \tikzmark{startName2a}foo\tikzmark{endName2a} : \tikzmark{startWidth2a}5\tikzmark{endWidth2a}; foo = 26;} \hspace{2.25cm} \textit{OR}
\item \texttt{wire \tikzmark{startName3a}foo\tikzmark{endName3a} : \tikzmark{startWidth3a}5\tikzmark{endWidth3a}; \tikzmark{startAssign3a}foo = 0x1a\tikzmark{endAssign3a};}
\end{itemize}
\begin{tikzpicture}[overlay,remember picture]
\begin{visibleenv}<2>
\coordinate (nameTopLeftA) at ([yshift=1em]pic cs:startName1a);
\coordinate (nameBottomRightA) at ([yshift=-.25em]pic cs:endName3a);
\coordinate (nameBottomLeftA) at (nameBottomRightA -| nameTopLeftA);
\coordinate (nameBottomCenterA) at ($(nameBottomRightA)!0.5!(nameBottomLeftA)$);
\draw[red,very thick] (nameTopLeftA) rectangle (nameBottomRightA);
\node[anchor=north] at (nameBottomCenterA) {
    \textit{name}
};
\end{visibleenv}
\begin{visibleenv}<3>
\coordinate (widthTopLeftA) at ([yshift=1em]pic cs:startWidth1a);
\coordinate (widthBottomRightA) at ([yshift=-.25em]pic cs:endWidth3a);
\coordinate (widthBottomLeftA) at (widthBottomRightA -| widthTopLeftA);
\coordinate (widthBottomCenterA) at ($(widthBottomRightA)!0.5!(widthBottomLeftA)$);
\draw[red,very thick] (widthTopLeftA) rectangle (widthBottomRightA);
\node[anchor=north] at (widthBottomCenterA) {
    \textit{width} (in bits)
};
\end{visibleenv}
\begin{visibleenv}<4>
\coordinate (assignTopLeftA) at ([yshift=1em]pic cs:startAssign1a);
\coordinate (assignBottomRightA) at ([yshift=-.25em,xshift=1.75cm]pic cs:endAssign3a);
\coordinate (assignBottomLeftA) at (assignBottomRightA -| assignTopLeftA);
\coordinate (assignBottomCenterA) at ($(assignBottomRightA)!0.5!(assignBottomLeftA)$);
\draw[red,very thick] (assignTopLeftA) rectangle (assignBottomRightA);
\node[anchor=north,align=center] at (assignBottomCenterA) {
    \textit{assignment} \\
    indicates wire is \textit{connected} to value
};
\end{visibleenv}
\end{tikzpicture}
\end{frame}

\begin{frame}{HCLRS: gates + calcuations (1)}
\texttt{wire a : 2; wire b : 2; wire c : 2;} \\
\texttt{c = \myemph<3>{b \& a};}\tikzmark{afterCAssign} \\
\texttt{a = 0b10;} \\
\texttt{b = 0b11;}
\vspace{2cm}
\begin{tikzpicture}[overlay,remember picture]
\begin{visibleenv}<2>
\node[align=left,font=\tt,anchor=north west] (alt code) at ([yshift=1em,xshift=3cm]pic cs:afterCAssign) {
a = 0b10; \\
b = 0b11; \\
c = \myemph<3>{b \& a};
};
\node[red] at ([xshift=-1.5cm]alt code.west) {same as};
\draw[red,decorate,decoration={brace},ultra thick] ([xshift=-2.9cm]alt code.north west) -- ([xshift=-2.9cm]alt code.south west);
\draw[red,decorate,decoration={brace,mirror},ultra thick] ([xshift=-0.1cm]alt code.north west) -- ([xshift=-0.1cm]alt code.south west);
\node[anchor=north,red,align=center] at ([xshift=-1.5cm,yshift=-.25cm]alt code.south west) {
    \textbf{order doesn't matter} \\
    connected or not
};
\end{visibleenv}
\begin{visibleenv}<3>
\node[anchor=north,red,align=center] at ([xshift=2cm,yshift=.9cm]alt code.west) {
    C-like expressions supported \\
    \texttt{0b10 \& 0b11 = 0b10}
};
\end{visibleenv}
\end{tikzpicture}

\begin{tikzpicture}[circuit logic US]
\draw[wire] (0, -.05) -- ++(3, 0) coordinate (end A bot);
\draw[wire] (0, .05) -- ++(3, 0) node[midway,yshift=-.05cm,fill=white,font=\tt] {a} coordinate (end A top);
\draw[wire] (0, -1.05) -- ++(3, 0) coordinate (end B bot);
\draw[wire] (0, -0.95) -- ++(3, 0) node[midway,yshift=-.05cm,fill=white,font=\tt] {b} coordinate (end B top);
\node[alt=<3>{red},thick,and gate,label={[alt=<3>{red},font=\scriptsize]center:AND},anchor=input 1] (and-1) at (5, .05) {};
\draw[wire] (end A top) -- (and-1.input 1);
\draw[wire] (end B top) -- ++ (0.75, 0) |- (and-1.input 2);
\node[alt=<3>{red},thick,and gate,label={[alt=<3>{red},font=\scriptsize]center:AND},anchor=input 2] (and-2) at (5, -1.05) {};
\draw[white,line width=3pt] (end A bot) -- ++(1.5, 0) |- (and-2.input 1);
\draw[wire] (end A bot) -- ++ (1.5, 0) |- (and-2.input 1);
\draw[wire] (end B bot) -- (and-2.input 2);

\coordinate (c height) at ($(and-1.output)!0.5!(and-2.output)$);
\coordinate (c join) at ([xshift=2cm]c height);
\draw[wire] (and-1.output) -- ++(1, 0) |- ([yshift=.05cm]c join) -- ++(2, 0);
\draw[wire] (and-2.output) -- ++(1, 0) |- ([yshift=-.05cm]c join) -- ++(2, 0);
\node[fill=white] at ([xshift=1cm]c join) {c};
\end{tikzpicture}
\end{frame}

\begin{frame}{HCLRS: gates + calcuations (2)}
\texttt{wire a : 2; wire b : 2; wire c : 2;} \\
\texttt{\myemph<1>{c = \large b \textbf{+} a}; /* was \sout{b \& a} */}\tikzmark{afterCAssign} \\
\texttt{a = 0b10;} \\
\texttt{b = 0b11;}
\vspace{2cm}
\begin{tikzpicture}[overlay,remember picture]
\begin{visibleenv}<1>
\node[anchor=north,red,align=center] at ([xshift=2cm,yshift=-0.5cm]alt code.west) {
    more than bitwise operators supported \\
    0b10 + 0b11 = 0b101 $\rightarrow$ 0b01 (extra bits lost)
};
\end{visibleenv}
\end{tikzpicture}

\begin{tikzpicture}[circuit logic US]
\draw[wire] (0, -.05) -- ++(3, 0) coordinate (end A bot);
\draw[wire] (0, .05) -- ++(3, 0) node[midway,yshift=-.05cm,fill=white,font=\tt] {a} coordinate (end A top);
\coordinate (end A mid inter) at ($(end A bot)!0.5!(end A top)$);
\coordinate (end A mid) at ([xshift=.25cm]end A mid inter);
\draw[wire] (end A bot) -- (end A mid);
\draw[wire] (end A top) -- (end A mid);
\draw[wire] (0, -1.05) -- ++(3, 0) coordinate (end B bot);
\draw[wire] (0, -0.95) -- ++(3, 0) node[midway,yshift=-.05cm,fill=white,font=\tt] {b} coordinate (end B top);
\coordinate (end B mid inter) at ($(end B bot)!0.5!(end B top)$);
\coordinate (end B mid) at ([xshift=.25cm]end B mid inter);
\draw[wire] (end B bot) -- (end B mid);
\draw[wire] (end B top) -- (end B mid);

\coordinate (end mid) at ($(end A mid)!0.5!(end B mid)$);

\node[red,draw,very thick,font=\Large] (add box) at ([xshift=2cm]end mid) {+};
\draw[red,wire thick,-Latex] ([xshift=-1mm]end A mid) -- ++ (1, 0) |- ([yshift=-.25cm]add box.north west);
\draw[red,wire thick,-Latex] ([xshift=-1mm]end B mid) -- ++ (1, 0) |- ([yshift=+.25cm]add box.south west);
\draw[red,wire thick,-Latex] (add box.east) -- ++ (4, 0) node[midway,fill=white] {c};
\end{tikzpicture}
\end{frame}


\section{circuits and state}

\subsection{counter example}

% FIXME: show counter circuit, then interlude:
    % this output should be + 1?
    % when will it change?

\usetikzlibrary{arrows.meta,calc,chains,fit,matrix,patterns,positioning,shapes.callouts,shapes.geometric,shapes.misc}

\usetikzlibrary{circuits.logic.US}
\usepgflibrary{shapes.geometric,shapes.gates.logic.mux}
\tikzset{
    simple wire/.style={very thick,>=Latex},
    wire/.style={line width=1.5pt,>=Latex},
    wire small/.style={line width=1pt,>=Latex},
    binLabel/.style={font=\tt},
    hiBox/.style={red, very thick, draw}
}

\providecommand{\counterCircuitBase}{
    \node[draw,minimum width=1cm,minimum height=1cm,fill=blue!20] (add1) at (3, 2) { add 1 };
    \draw[wire,->] (add1) -- (6, 2) -- (6, 0) -- (0, 0) -- (0, 2) -- (add1);
    \draw[very thick] (1, 2) ++ (.05,.1) -- ++(-.1,-.2)
        node[below] {3};
    \coordinate (pin point) at (6, 0.5);
    \coordinate (pin point 2) at (5, 0);
}

\providecommand{\counterCircuitHCL}{
    \node[font=\tt,align=left] (hcl code) at (1, -1) {
        wire x : 3; \\
        x = x + 1;
    };
}

\begin{frame}{example: (broken) counter circuit (1)}
\begin{tikzpicture}
    \counterCircuitBase
    \begin{visibleenv}<2->
    \counterCircuitHCL
    \end{visibleenv}
    \begin{visibleenv}<3->
    \draw[thick] (pin point) -- ++(0.5, 0) node[right,align=left] (which) {
        {time 0: \tt 000} \\
        \only<5->{time 1: \tt \myemph<5>{001}?} \\
        \only<5->{time 2: \tt \myemph<5>{010}?} \\
        \only<5->{time 3: \tt 011?} \\
    };
    \end{visibleenv}
    \begin{visibleenv}<4>
    \node[anchor=north west,xshift=0cm] at (which.north east) {
        $\leftarrow$ set how???
    };
    \end{visibleenv}
\end{tikzpicture}
\end{frame}

\begin{frame}{example: (broken) counter circuit (2)}
    \begin{tikzpicture}
    \counterCircuitBase
    \counterCircuitHCL
    \node[align=left,anchor=north west,fill=white,draw=red,ultra thick] (explain box 2) at ([yshift=-1cm]hcl code.south west) { 
        HCLRS: compile error \\ ``\texttt{Circular dependency detected:} \\ \texttt{x depends on x}''
    };
    \node[draw,red,very thick, cross out,fit=(hcl code)] {};
    \end{tikzpicture}
\end{frame}

\begin{frame}{example: (broken) counter circuit (3)}
    \begin{tikzpicture}
    \counterCircuitBase
    \counterCircuitHCL
    \draw[thick] (pin point) -- ++(0.5, 0) node[right,align=left] (which) {
        {time 0: \tt 0\myemph<3>{0}\myemph<1>{0}} \\
        {time 1: \tt 0\myemph<3>{0}\myemph<1>{1}?} \\
        {time 2: \tt 0\myemph<3>{1}\myemph<1>{0}?} \\
        {time 3: \tt 0\myemph<3>{1}\myemph<1>{1}?} \\
    };
        \coordinate (wire diag bottom left) at ([yshift=-2.5cm,xshift=-2cm]which.south west);
        \coordinate (wire diag top right) at ([yshift=2.5cm,xshift=4.5cm]wire diag bottom left);
        %\draw[thick,dotted] (pin point 2) -- (wire diag top right);
        %\draw[thick,dotted] (pin point 2) -- ([xshift=-.5cm]wire diag bottom left |- wire diag top right);
        \begin{scope}[shift={(wire diag bottom left)},y=0.5cm,x=2cm]
            \begin{scope}[yshift=2.2cm]
            \draw[alt=<1>{red},thick,rounded corners] (0, -1)-- (0.5, -1) -- (1, 0) -- (1.5, 0) -- (2, -1) -- (2.5, -1) -- (3, 0);
            \begin{visibleenv}<2>
                \draw[red,very thick] (0.6, -1.1) rectangle (0.9, 0.1);
                \draw[red,thick] (0.8, -1.1) -- ++(0.2, -0.2) node[below,align=center] {problem 1: how will ``add 1'' react to this value? \\ (not zero or one) \ldots};
            \end{visibleenv}
            \end{scope}
            \begin{scope}[yshift=1.1cm]
            \begin{visibleenv}<3->
            \draw[alt=<3>{red},thick,rounded corners] (0, -1)-- (0.6, -1) -- (1, -1) -- (1.6, 0) -- (2, 0) -- (2.3, 0) -- (2.7, 0) -- (3.0, 0);
            \begin{visibleenv}<4>
                \draw[red,very thick] (1.3, 1.8) rectangle (2.9, -1.1);
                \draw[red,thick] (1.7, -1.1) -- ++(0.2, -0.2) node[below] {problem 2: changes not in sync?};
            \end{visibleenv}
            \end{visibleenv}
            \end{scope}

        \end{scope}

    \end{tikzpicture}
\end{frame}

\begin{frame}{example: (broken) counter circuit (4)}
    \begin{tikzpicture}
    \counterCircuitBase
    \counterCircuitHCL
    \draw[thick] (pin point) -- ++(0.5, 0) node[right,align=left] (which) {
        {time 0: \tt 000} \\
        {time 1: \tt 001?} \\
        {time 2: \tt 010?} \\
        {time 3: \tt 011?} \\
    };
    \begin{visibleenv}<1->
        \node[cross out,fit=(which),draw,ultra thick,red] {};
        \node[align=left,anchor=north west,fill=white,draw=red,ultra thick] (explain box) at (6, -.5) { 
            circuit is \myemph{not stable} \\
            \myemph{transient values} during changes \\
            hard to predict behavior
        };
    \end{visibleenv}
    \end{tikzpicture}
\end{frame}


\begin{frame}{circuits: state}
    \begin{itemize}
    \item logic performs calculations all the time
    \item never stores values!
    \item need \myemph{extra elements} to store values
        \begin{itemize}
        \item registers, memory
        \end{itemize}
    \end{itemize}
\end{frame}

\begin{frame}{example: counter circuit (corrected)}
\begin{tikzpicture}
    \node[draw,minimum width=1cm,minimum height=1cm,fill=blue!20] (add1) at (3, 2) { add 1 };
    \node[hRegSmall={count},draw=red,thick,fill=red!10] (reg) at (5.5, 2) {};
    \draw[wire,->] (add1) -- (reg);
    \draw[wire,->] (reg) -- (7, 2) -- (7, 0) -- (0, 0) -- (0, 2) -- (add1);
    \coordinate (pinPoint) at (7, 1);
    \begin{visibleenv}<2->
    \draw[thick] (pinPoint) -- ++(0.5, 0) node[right,align=left] (which) {
        {time 0: \tt 000} \\
        {time 1: \tt 001} \\
        {time 2: \tt 010} \\
        {time 3: \tt 011} \\
    };
    \end{visibleenv}
    \begin{visibleenv}<3->
        \node[align=left] at (4, -3) {
            add \myemph{register} to store current count \\
            updates based on ``clock signal'' (not shown) \\
            avoids intermediate updates
        };
    \end{visibleenv}
\end{tikzpicture}
\end{frame}

% FIXME: regsiter timing picture here



\subsection{registers}

\begin{frame}[fragile,label=registerOperation]{registers}
    \begin{tikzpicture}
        \node[hReg={}] (pc) {};
        \draw[a,latex-] (pc.west) -- ++(-1cm, 0cm);
        \draw[a] (pc.east) -- ++(1cm, 0cm);
        \node[below=1.5cm of pc,visible on=<1>] (everyCC) {
            updates every \myemph{clock cycle}
        };
        \begin{scope}[shift={($(everyCC.east)+(1cm,-.5cm)$)}] 
            \draw[very thick] (0, 0) -- (1, 0) -- (1, 1) -- (2, 1) -- (2, 0) -- (3, 0) -- (3, 1) -- (4, 1);
            \draw[ultra thick,red!95!black] (1, 0) -- (1, 1);
            \draw[ultra thick,red!95!black] (3, 0) -- (3, 1);
        \end{scope}
        \begin{scope}[shift={($(everyCC.east)+(1cm,-.6cm)$)}] 
            \fill[color=blue!80!black] (0, 0) rectangle (0.99, -.5);
            \fill[color=green!40!white] (1.01, 0) rectangle (2.99, -.5);
            \fill[color=violet!60!black] (3.01, 0) rectangle (4., -.5);
            \node[anchor=east,font=\small] at (0,-.25) {register output};
            \fill[pattern=north west lines] (0.0, -0.6) rectangle (0.2, -1.1);
            \fill[color=green!40!white] (0.2, -0.6) rectangle (1.01, -1.1);
            \fill[pattern=north west lines] (1.01, -0.6) rectangle (1.7, -1.1);
            \fill[color=violet!60!black] (1.7, -0.6) rectangle (3.01, -1.1);
            \fill[pattern=north west lines] (3.01, -0.6) rectangle (3.9, -1.1);
            \fill[color=orange!40] (3.9, -0.6) rectangle (4., -1.1);
            \node[anchor=east,font=\small] at (0,-.85) {register input};
        \end{scope}
    \end{tikzpicture}
\end{frame}


\subsection{registers in HCLRS}

% FIXME
\tikzset{
    simple wire/.style={very thick,>=Latex},
    wire/.style={line width=1.5pt,>=Latex},
    wire small/.style={line width=1pt,>=Latex},
    binLabel/.style={font=\tt},
    hiBox/.style={red, very thick, draw}
}

\begin{frame}{example: counter circuit (real HCLRS)}
\begin{tikzpicture}
    \node[draw,minimum width=1cm,minimum height=1cm,fill=blue!20] (add1) at (3, 2) { add 1 };
    \node[hRegSmall={count},draw=red,thick,fill=red!10] (reg) at (5.5, 2) {};
    \draw[wire,->,alt=<6>{red,line width=3pt}] (add1) -- (reg);
    \draw[wire,->,alt=<7>{red,line width=3pt}] (reg) -- (7, 2) -- (7, 0) -- (0, 0) -- (0, 2) -- (add1);
    \coordinate (pinPoint) at (7, 1);
    \begin{visibleenv}<2->
        \node[align=left,font=\tt] (codeBox) at (4, -3) {
            register \alt<4>{%
                \setlength\fboxrule{2pt}%
                \fcolorbox{red}{white}{{\color{green!80!black}x}{\color{blue!80!black}Y}}%
            }{%
                \setlength\fboxrule{2pt}%
                \fcolorbox{white}{white}{{\color{green!80!black}x}{\color{blue!80!black}Y}}%
            } \{ \\ %
            \hspace{1cm} \alt<5>{%
                    \setlength{\fboxrule}{2pt}\fcolorbox{red}{white}{count}%
                }{%
                    \setlength{\fboxrule}{2pt}\fcolorbox{white}{white}{count}%
                } : 3 = %
                \setlength{\fboxrule}{2pt}%
                \alt<8>{%
                    \fcolorbox{red}{white}{0b000}%
                }{%
                    \fcolorbox{white}{white}{0b000}%
                }; \\ %
            \} \\ %
            \setlength\fboxrule{2pt}%
            \alt<6>{%
                \fcolorbox{red}{white}{{\color{green!80!black}x}\_count}%
            }{%
                \fcolorbox{white}{white}{{\color{green!80!black}x}\_count}%
            } %
            = %
            \alt<7>{%
                \fcolorbox{red}{white}{{\color{blue!80!black}Y}\_count}%
            }{%
                \fcolorbox{white}{white}{{\color{blue!80!black}Y}\_count}%
            } %
            + 0b001;%
        };
    \end{visibleenv}
    \begin{visibleenv}<2->
        \path[fill=green]  (reg.west) circle (.1cm);
        \path[fill=blue]  (reg.east) circle (.1cm);
    \end{visibleenv}
    \begin{visibleenv}<3>
        \draw[ultra thick,red] ([xshift=8cm,yshift=0em]codeBox.north west) -- ++ (0cm, -1.8cm) node[align=left,midway,right] (bank label) {
            register ``bank'' \\
            can have multiple (related) registers
        };
    \end{visibleenv}
    \begin{visibleenv}<4>
        \draw[ultra thick,red,yshift=-1cm] ([xshift=3.5cm,yshift=-.25em]codeBox.north west) -- ++ (3cm, .25cm) node[align=left,above right,yshift=-1cm] (prefix label) {
            label for left/right side of registers \\
            {\tt x}: label for input side (always lowercase) \\
            {\tt Y}: label for output side (always uppercase)
        };
        \draw[ultra thick,red] ([yshift=1mm]prefix label.south west) -- ([yshift=-1mm]prefix label.north west);
    \end{visibleenv}
    \begin{visibleenv}<5>
        \draw[ultra thick,red,yshift=-1cm] ([xshift=3cm,yshift=-1.75em]codeBox.north west) -- ++ (2cm, .1cm) node[align=left,above right] (regname label) {
            register ``name'' \\
            input/output = \texttt{\textit{prefix}\_name}
        };
        \draw[ultra thick,red] (regname label.south west) -- (regname label.north west);
    \end{visibleenv}
    \begin{visibleenv}<6>
        \draw[ultra thick,red,yshift=-1cm] ([xshift=1cm,yshift=-5.5em]codeBox.north west) -- ++ (1cm, -.5cm) node[align=left,below right] (input label) {
            input wire to register
        };
    \end{visibleenv}
    \begin{visibleenv}<7>
        \draw[ultra thick,red,yshift=-1cm] ([xshift=5cm,yshift=-5.5em]codeBox.north west) -- ++ (1cm, -.5cm) node[align=left,below right] (output label) {
            output wire of register
        };
    \end{visibleenv}
    \begin{visibleenv}<8>
        \draw[ultra thick,red,yshift=-1cm] ([xshift=6cm,yshift=-1.5em]codeBox.north west) -- ++ (1cm, .5cm) node[align=left,above right] (init label) {
            initial value of register \\
            first value for output wire (\texttt{Y\_count})
        };
    \end{visibleenv}
\end{tikzpicture}
\end{frame}

\begin{frame}{example: counter circuit}
\begin{tikzpicture}
    \node[draw,minimum width=1cm,minimum height=1cm,fill=blue!20] (add1) at (3, 2) { add 1 };
    \node[hRegSmall={count},draw=red,thick,fill=red!10] (reg) at (5.5, 2) {};
    \draw[wire,->,alt=<0>{red,line width=3pt}] (add1) -- (reg);
    \draw[wire,->,alt=<0>{red,line width=3pt}] (reg) -- (7, 2) -- (7, 0) -- (0, 0) -- (0, 2) -- (add1);
    \coordinate (pinPoint) at (7, 1);
    \begin{visibleenv}<1->
        \node[align=left,font=\small\tt,anchor=north west] (codeBox) at (8, 2) {
            register \alt<0>{%
                \setlength\fboxrule{2pt}%
                \fcolorbox{red}{white}{{\color{green!80!black}x}{\color{blue!80!black}Y}}%
            }{%
                \setlength\fboxrule{2pt}%
                \fcolorbox{white}{white}{{\color{green!80!black}x}{\color{blue!80!black}Y}}%
            } \{ \\%
            \hspace{1cm} \alt<0>{%
                    \setlength{\fboxrule}{2pt}\fcolorbox{red}{white}{count}%
                }{%
                    \setlength{\fboxrule}{2pt}\fcolorbox{white}{white}{count}%
                } : 3 = %
                \setlength{\fboxrule}{2pt}%
                \alt<0>{%
                    \fcolorbox{red}{white}{0b000}%
                }{%
                    \fcolorbox{white}{white}{0b000}%
                }; \\ %
            \} \\%
            \setlength\fboxrule{2pt}%
            \alt<0>{%
                \fcolorbox{red}{white}{{\color{green!80!black}x}\_count}%
            }{%
                \fcolorbox{white}{white}{{\color{green!80!black}x}\_count}%
            } %
            = %
            \alt<0>{%
                \fcolorbox{red}{white}{{\color{blue!80!black}Y}\_count}%
            }{%
                \fcolorbox{white}{white}{{\color{blue!80!black}Y}\_count}%
            } %
            + 0b001;%
        };
    \end{visibleenv}
    \begin{visibleenv}<2->
    \matrix[tight matrix no line,anchor=north west,
            nodes={text width=2.5cm,font=\small,align=center},
            column 1/.style={nodes={align=left,text width=4cm}},
            row 1/.style={row sep=1mm},
        ] (timeline) at (0, -.5) {
        time \& \tt Y\_count \& \tt x\_count \\
        start \& \tt 000 \& \tt 001  \\
        start + 1 rising edge \& \tt 001 \& \tt 010 \\
        start + 2 rising edges \& \tt 010 \& \tt 011 \\
        start + 3 rising edges \& \tt 011 \& \tt 100 \\
        \ldots \& \ldots \& \ldots \\
    };
    \draw[very thick] (timeline-1-1.south west) -- (timeline-1-3.south east);
    \node[fit=(timeline),draw,very thick,inner sep=0.1mm] {};
    \end{visibleenv}
    \begin{visibleenv}<3>
        \node[draw,red,very thick,fit=(timeline-2-3),inner xsep=-.25cm,inner ysep=0cm] (box 1 in) {};
        \node[draw,red,very thick,fit=(timeline-3-2),inner xsep=-.25cm,inner ysep=0cm] (box 1 out) {};
        \draw[red,-Latex,very thick] (box 1 in.west) -- (box 1 out.east);
        \node[draw,red!50!white,very thick,fit=(timeline-3-3),inner xsep=-.25cm,inner ysep=0cm] (box 2 in) {};
        \node[draw,red!50!white,very thick,fit=(timeline-4-2),inner xsep=-.25cm,inner ysep=0cm] (box 2 out) {};
        \draw[red!50!white,-Latex,very thick] (box 2 in.west) -- (box 2 out.east);
        \node[draw,red!25!white,very thick,fit=(timeline-4-3),inner xsep=-.25cm,inner ysep=0cm] (box 3 in) {};
        \node[draw,red!25!white,very thick,fit=(timeline-5-2),inner xsep=-.25cm,inner ysep=0cm] (box 3 out) {};
        \draw[red!25!white,-Latex,very thick] (box 3 in.west) -- (box 3 out.east);
    \end{visibleenv}
\end{tikzpicture}
\end{frame}


\subsection{exercise}
\usetikzlibrary{arrows.meta,calc,chains,matrix,patterns,positioning,shapes.callouts,shapes.geometric,shapes.misc}

\tikzset{
    simple wire/.style={very thick,>=Latex},
    wire/.style={line width=1.5pt,>=Latex},
    wire small/.style={line width=1pt,>=Latex},
    binLabel/.style={font=\tt},
    hiBox/.style={red, very thick, draw}
}

\begin{frame}[fragile,label=hclRegCircuitEx]{HCL circuit with registers}
\begin{Verbatim}
register xY {
    a : 4 = 1;  /* <-- initial Y_a */
    b : 4 = 1;  /* <-- initial Y_b */
}
x_b = x_a + Y_a;
x_a = Y_a + Y_b;
\end{Verbatim}
% zero: x_a = 2, x_b = 3
% one cycle: Y_a = 2, Y_b = 3; x_a = 5, x_b = 7
% two cycles Y_a = 5, Y_b = 7
\begin{itemize}
\item exercise: value of Y\_a, Y\_b after two rising edges of clock?
    \begin{itemize}
    \item A. Y\_a = 2, Y\_b = 3
    \item B. Y\_a = 2, Y\_b = 2
    \item C. Y\_a = 3, Y\_b = 5
    \item D. Y\_a = 3, Y\_b = 7
    \item E. Y\_a = 3, Y\_b = 11
    \item F. Y\_a = 5, Y\_b = 7
    \item G. Y\_a = 7, Y\_b = 11
    \item H. none of the above
    \end{itemize}
    \end{itemize}
\end{frame}


% FIXME: register operation exercise

\subsection{instruction memory}


\begin{frame}[fragile,label=imem]{instruction memory}
    \begin{tikzpicture}
        \node[mem] (imem) {Instr. \\ Mem.};
        \coordinate (imemData) at (imem.east);
        \coordinate (imemAddr) at (imem.west);
        \draw[thick,-latex] (imemData) -- +(.5cm,0cm) node [font=\small,right] { data (HCL: \texttt{i10bytes}) };
        \draw[thick,latex-] (imemAddr) -- +(-.3cm,0cm) node [font=\small,left] { address (HCL: \texttt{pc}) };
        \begin{scope}[shift={($(imem.south east) + (-1, -3)$)}] 
            \node[anchor=east,font=\small] at (0,-.25) {address input};
            \fill[color=red!80!black] (0.1, 0) rectangle (2.0, -.5);
            \fill[pattern=north west lines] (0.0, -.6) rectangle (0.9, -1.1);
            \fill[color=blue!80!black] (0.9, -.6) rectangle (2.0, -1.1);
            \node[anchor=east,font=\small] at (0,-.85) {data output};
            \draw[thin,-latex] (0.0, -1.3) -- (1.5, -1.3) node[right,font=\scriptsize] {time};
        \end{scope}

    \end{tikzpicture}
\end{frame}



% FIXME: value in example exercise?

\subsection{Stat signal}

\begin{frame}{Stat signal}
    \begin{itemize}
    \item how do we stop the simulated machine?
    \item hard-wired mechanism --- {\tt Stat} wire
    \item possible values:
        \begin{itemize}
        \item {\tt STAT\_AOK} --- keep going
        \item {\tt STAT\_HLT} --- stop, normal shtdown
        \item {\tt STAT\_INS} --- invalid instruction
        \item \ldots (and more errors)
        \end{itemize}
    \item (predefined 3-bit constants)
    \item \myemph{must be set}
    \item determines if \myemph{simulator} keeps going
    \end{itemize}
\end{frame}


\subsection{nop CPU}
\tikzset{
    pinOn/.style={
        pin edge={-,thick,visible on=#1},
        visible on=#1,
    },
}

\begin{frame}<1-7>[fragile,label=nopCPU]{nop CPU}
\begin{tikzpicture}
        \tikzset{
            hiOn/.style={alt=#1{draw=red,ultra thick}{}},
        }
        \node[hRegSmall=thePc,pcStyle] (pc) {};
        \node[mem,right=1.5cm of pc,font=\scriptsize] (imem) {Instr. \\ Mem.};
        \node[draw,fill=blue!20,hiOn=<4>] (add1) at ([yshift=-2cm,xshift=-.25cm]pc.north) { add 1 };

        \begin{visibleenv}<3>
            \fill[blue!80!black] ([xshift=-1mm]pc.west) circle (1.5mm);
            \fill[green!60!black] ([xshift=1mm]pc.east) circle (1.5mm);
        \end{visibleenv}

        \draw[a,hiOn=<5>] (pc.east) -- (imem.west) node[near end,pin={[pinOn=<5->]north:``{\tt pc}''}] {};
        \draw[a,hiOn=<5>] (imem.east) -- ++(1cm,0cm) node[pin={[pinOn=<5->]north:``{\tt i10bytes}''}] {} -- ++(1cm,0cm);
        \draw[a,hiOn=<4>] (pc.east) -- ++(.75cm,0cm) |- (add1.east);
        \draw[a,hiOn=<4>] (add1.west) -| ([xshift=-1cm]pc.west) -- (pc.west);
        \coordinate (textLoc)  at ([yshift=-.1cm,xshift=-1cm]add1.south west);
        
        \begin{visibleenv}<6->
        \node[hRegSmall=Stat,anchor=north] (Stat) at ([xshift=5cm,yshift=.5cm]add1) {};
        \draw[aR] (Stat) -- ++ (-1cm,0cm) node[left,font=\tt\small] {STAT\_AOK};
        \end{visibleenv}

\begin{scope}[every node/.style={font=\small\tt,align=left,anchor=north west,inner sep=.25mm}]
\begin{visibleenv}<2->
\node[at={(textLoc)}] (registerDecl) {
{\bfseries register} \textcolor{blue!80!black}{p}\textcolor{green!60!black}{F} \{ \\
\hspace{.5cm} thePc : 64 = 0; \\
\}
};
\end{visibleenv}
\begin{visibleenv}<4->
\node[at={(registerDecl.south west)}] (assignStatementA) {
p\_thePc = \myemph<4>{F\_thePc + 1};
};
\end{visibleenv}

\begin{visibleenv}<5->
\node[at={(assignStatementA.south west)}] (assignStatementB) {
pc = F\_thePc;
};
\end{visibleenv}
\begin{visibleenv}<5>
\node[my callout2=imem.south east,anchor=north west,font=\normalfont] at ([xshift=.5cm,yshift=-.5cm]imem.south east) {
    built-in component \\
    use is \myemph{mandatory}
};
\end{visibleenv}
\begin{visibleenv}<6->
\node[at={(assignStatementB.south west)}] (boilerplate) {
Stat = STAT\_AOK;
};
\end{visibleenv}
\begin{visibleenv}<6|8>
\node[my callout2=Stat.north east,anchor=south west,font=\normalfont] at ([xshift=.5cm,yshift=.5cm]Stat.north east) {
    built-in component: \\
    AOK: continue \\
    HLT: stop 
};
\end{visibleenv}
\end{scope}
\end{tikzpicture}
\end{frame}


\subsection{running nop CPU}
\begin{frame}{nop CPU: running}
    \begin{itemize}
    \item need a program in memory
        \begin{itemize}
        \item .yo file
        \end{itemize}
    \item {\tt tools/yas} --- convert {\tt .ys}  to {\tt .yo}
    \item {\tt tools/yis} --- reference interpreter for {\tt .yo} files
        \begin{itemize}
        \item if your processor doesn't do the same thing\ldots
        \end{itemize}
    \item can build tools by running {\tt make}
    \end{itemize}
\end{frame}

\begin{frame}[fragile,label=makeProgram]{nop CPU: creating a program}
    \begin{itemize}
    \item create assemby file: nops.ys:
\begin{Verbatim}
    nop
    nop
    nop
    nop
    nop
\end{Verbatim}
    \item assemble using {\tt tools/yas nops.ys} or {\tt make nops.yo}
    \end{itemize}
\end{frame}

\begin{frame}[fragile,label=nopYo]{nop.yo}
    \begin{itemize}
    \item more readable/simpler than normal executables:
\begin{Verbatim}
0x000: 10                   | nop
0x001: 10                   | nop
0x002: 10                   | nop
0x003: 10                   | nop
0x004: 10                   | nop
                            | 
\end{Verbatim}
    \item loaded into data and program memory
    \item parts left of {\tt |} just comments
    \end{itemize}
\end{frame}

\begin{frame}[fragile,label=runSim1]{running a simulator (1)}
    \begin{itemize}
    \vspace{.25cm}
\begin{Verbatim}[fontsize=\fontsize{8}{9}\selectfont]
Usage: ./hclrs [options] HCL-FILE [YO-FILE [TIMEOUT]]
Runs HCL_FILE on YO-FILE. If --check is specified, no YO-FILE may be supplied.
Default timeout is 9999 cycles.

Options:
    -c, --check         check syntax only
    -d, --debug         output wire values after each cycle and other debug
                        output
    -q, --quiet         only output state at the end
    -t, --testing       do not output custom register banks (for autograding)
    -h, --help          print this help menu
    -i, --interactive   prompt after each cycle
        --trace-assignments 
                        show assignments in the order they are simulated
        --version       print version number
\end{Verbatim}
    \end{itemize}
\end{frame}

\begin{frame}[fragile,label=runSim2]{running a simulator (2)}
\begin{Verbatim}[fontsize=\fontsize{8}{9}\selectfont,commandchars=\\\{\}]
$ ./hclrs nop_cpu.hcl nops.yo
+------------------- between cycles    0 and    1 ----------------------+
| RAX:                0   RCX:                0   RDX:                0 |
| RBX:                0   RSP:                0   RBP:                0 |
| RSI:                0   RDI:                0   R8:                 0 |
| R9:                 0   R10:                0   R11:                0 |
| R12:                0   R13:                0   R14:                0 |
| register \vemphA{pF(N) { thePc=0000000000000000 }}                             |
| used memory:   _0 _1 _2 _3  _4 _5 _6 _7   _8 _9 _a _b  _c _d _e _f    |
|  0x0000000_:   \vemphB{10 10 10 10  10}                                        |
+-----------------------------------------------------------------------+
\vemphC{pc = 0x0; loaded [10 : nop]}
....
+------------ timed out after  9999 cycles in state: -------------------+
| RAX:                0   RCX:                0   RDX:                0 |
| RBX:                0   RSP:                0   RBP:                0 |
| RSI:                0   RDI:                0   R8:                 0 |
| R9:                 0   R10:                0   R11:                0 |
| R12:                0   R13:                0   R14:                0 |
| register pF(N) { thePc=000000000000270f }                             |
| used memory:   _0 _1 _2 _3  _4 _5 _6 _7   _8 _9 _a _b  _c _d _e _f    |
|  0x0000000_:   10 10 10 10  10                                        |
+-----------------------------------------------------------------------+
\end{Verbatim}
\end{frame}






\section{aside: MUXes}
\subsection{motivation}
\againframe<8>{nopCPU}
\subsection{the HW component}
\begin{frame}[fragile,label=Muxes]{multiplexers}
    % FIXME: background colors for output/select 
    \newcommand{\cA}{\color{red!80!black}}
    \newcommand{\cB}{\color{blue!80!black}}
    \newcommand{\cC}{\color{green!60!black}}
    \newcommand{\cD}{\color{violet!90!black}}
    \begin{tikzpicture}[circuit logic]
        \node [draw,mux,inputs={nnnn},info={center:MUX},minimum height=3cm,minimum width=1.5cm] (theMux) {};
        \draw[thick,latex-] (theMux.input 1) -- +(-.5cm,0cm) node[left] (a) {\cA a};
        \draw[thick,latex-] (theMux.input 2) -- +(-.5cm,0cm) node[left] (b) {\cB b};
        \draw[thick,latex-] (theMux.input 3) -- +(-.5cm,0cm) node[left] (c) {\cC c};
        \draw[thick,latex-] (theMux.input 4) -- +(-.5cm,0cm) node[left] (d) {\cD d};
        \draw[thick,-latex] (theMux.output) -- +(.5cm,0cm) node[right] (output) {\strut output};
        \draw[thin,latex-] (theMux.select) -- +(0cm,-.5cm) node[below] (select) {\strut select};
        \onslide<2->{
            \node[right=1pt of select] {\strut= {\cA 0} or {\cB 1} or {\cC 2} or {\cD 3}};
            \node[right=1pt of output] {\strut= {\cA a} or {\cB b} or {\cC c} or {\cD d}};
        }
        \onslide<3->{
            \node[below right=.1cm and -2cm of select,align=left] {
                truth table: \\
                \begin{tabular}{ll|l}
                select bit 1 & select bit 0 & output (many bits) \\
                \tt 0 & \tt 0 & \cA a \\
                \tt 0 & \tt 1 & \cB b \\
                \tt 1 & \tt 0 & \cC c \\
                \tt 1 & \tt 1 & \cD d \\
                \end{tabular}
            };
        }
    \end{tikzpicture}
\end{frame}


\subsection{abstraction in HCL}
\begin{frame}{MUXes in HCLRS}
\begin{itemize}
\item book calls ``case expression''
\item conditions evaluated (as if) \myemph{in order}
\item first match is output:
{\tt\small
result = [ \\
\hspace{.5cm} x == 5: 1; \\
\hspace{.5cm} x in \{0, 6\}: 2; \\
\hspace{.5cm} x > 2: 3; \\
\hspace{.5cm} 1: 4; \\
]; \\
}
    \begin{itemize}
    \item x = 5: result is 1
    \item x = 6: result is 2
    \item x = 3: result is 3
    \item x = 4: result is 3
    \item x = 1: result is 4
    \end{itemize}
\end{itemize}
\end{frame}


\subsection{exercise}
\begin{frame}[fragile,label=muxExercise]{MUX exercise}
\begin{Verbatim}
foo = [
    bar > 10 : 100;
    (bar & 1) == 1 : 200;
    bar < 20 : 300;
    1 : 400;
]
\end{Verbatim}
exercise 1: if \texttt{bar} is 9, what is foo? \\
exercise 2: if \texttt{bar} is 10, what is foo? \\
exercise 3: if \texttt{bar} is 11, what is foo? \\
\end{frame}


\section{nop+halt CPU}
% FIXME: nop+halt ISA
% FIXME: nop+halt CPU picture
% FIXME: nop+halt HCL

\begin{frame}{Simple ISA: nop/halt CPU}
    \begin{itemize}
    \item nop
        \begin{itemize}
        \item encoding \texttt{10}
        \end{itemize}
    \item halt
        \begin{itemize}
        \item encoding \texttt{00}
        \end{itemize}
    \vspace{.5cm}
    \item<2-> our strategy: MUX to decide using opcode
    \end{itemize}
\end{frame}

% FIXME: example output

\tikzset{
    pinOn/.style={
        pin edge={-,thick,visible on=#1},
        visible on=#1,
    },
}

\begin{frame}<0>[fragile,label=nopHltCPU]{nop/halt CPU}
\begin{tikzpicture}[circuit logic US]
        \tikzset{
            hiOn/.style={alt=#1{draw=red,ultra thick}{}},
        }
        \node[hRegSmall=thePc,pcStyle] (pc) {};
        \node[mem,right=1.5cm of pc,font=\scriptsize] (imem) {Instr. \\ Mem.};
        \node[draw,fill=blue!20] (add1) at ([yshift=-2cm,xshift=-.25cm]pc.north) { add 1 };

        \draw[a] (pc.east) -- (imem.west);
        \draw[a] (pc.east) -- ++(.75cm,0cm) |- (add1.east);
        \draw[a] (add1.west) -| ([xshift=-1cm]pc.west) -- node[near end,pin={[pinOn=<0>]west:valP}] {} (pc.west);
        \coordinate (textLoc)  at ([yshift=-.1cm,xshift=-1cm]add1.south west);
        
        \node[hRegSmall=Stat,anchor=north] (Stat) at ([xshift=6cm,yshift=.5cm]add1) {};
        \node[draw,mux,inputs=nnn,minimum height=1cm,hiOn=<2>,label={[font=\scriptsize,align=center]center:M\\U\\X}] (statMux) at ([xshift=-1cm]Stat.west) {~};
        \draw[bR] (statMux.input 1) -- ++(-.5cm,0cm) node[left,font=\tt\scriptsize] {STAT\_AOK};
        \draw[bR] (statMux.input 2) -- ++(-.5cm,0cm) node[left,font=\tt\scriptsize] {STAT\_HLT};
        \draw[bR] (statMux.input 3) -- ++(-.5cm,0cm) node[left,font=\tt\scriptsize] {STAT\_INS};
        \draw[b] (statMux.output) -- (Stat.west);

        \node[draw,fill=blue!20,font=\small,hiOn=<3>] (extractOp) at (statMux |- imem.west) { extract opcode };
        \draw[a] (imem.east) -- (extractOp.west);
        \draw[b] (extractOp.south -| statMux.north) -- (statMux.north);

        \coordinate (textLoc)  at ([yshift=-.1cm,xshift=-1cm]add1.south west);
\begin{scope}[every node/.style={font=\fontsize{9}{10}\selectfont\tt,align=left,anchor=north west,inner sep=.25mm}]
\begin{visibleenv}<4->
\node[at={(textLoc)}] (registerDecl) {
{\bfseries register} \textcolor{blue!80!black}{p}\textcolor{green!60!black}{P} \{ \\
\hspace{.5cm} thePc : 64 = 0; \\
\} \\
p\_thePc = P\_thePc + 1; \\
pc = P\_thePc; \\
Stat = [ \\
\hspace{.5cm} \myemph<5>{i10bytes[4..8]} == NOP : STAT\_AOK; \\
\hspace{.5cm} \myemph<5>{i10bytes[4..8]} == HALT : STAT\_HLT; \\
\hspace{.5cm} \myemph<6>{1} : STAT\_INS; // {\normalfont\textit{(default case)}} \\
];
};
\end{visibleenv}
\end{scope}
\end{tikzpicture}
\end{frame}


\againframe<1-2>{nopHltCPU}

\subsection{using i10bytes}
\begin{frame}{what is i10bytes?}
\begin{tikzpicture}
\matrix[
    tight matrix,
    nodes={font=\fontsize{9}{10}\tt\selectfont},
    row 2 column 2/.style={alt=<3-6>{nodes={fill=blue!10}},alt=<4>{nodes={fill=red!25}}},
    row 3 column 2/.style={alt=<3-9>{nodes={fill=blue!10}},alt={<5,7>{nodes={fill=red!25}}}},
    row 4 column 2/.style={alt=<3-9>{nodes={fill=blue!10}},alt={<6,8>{nodes={fill=red!25}}}},
    row 5 column 2/.style={alt=<3-9>{nodes={fill=blue!10}},alt={<9>{nodes={fill=red!25}}}},
    row 6 column 2/.style={alt=<3-9>{nodes={fill=blue!10}}},
    row 7 column 2/.style={alt=<3-9>{nodes={fill=blue!10}}},
    row 8 column 2/.style={alt=<3-9>{nodes={fill=blue!10}}},
    row 9 column 2/.style={alt=<3-9>{nodes={fill=blue!10}}},
    row 10 column 2/.style={alt=<3-9>{nodes={fill=blue!10}}},
    row 11 column 2/.style={alt=<3-9>{nodes={fill=blue!10}}},
    row 12 column 2/.style={alt=<7-9>{nodes={fill=blue!10}}},
] (mem) {
\normalfont\small addr. \& \normalfont\small value \\
0x000 \& \myemph<4>{0x60} \\
0x001 \& \myemph<5,7>{0x12} \\
0x002 \& \myemph<6,8>{0x61} \\
0x003 \& \myemph<9>{0x21} \\
0x004 \& 0x00 \\
0x005 \& 0x00 \\
0x006 \& 0x00 \\
0x007 \& 0x00 \\
0x008 \& 0x00 \\
0x009 \& 0x00 \\
0x00a \& 0x01 \\
0x00b \& 0x00 \\
0x00c \& 0x00 \\
0x00d \& 0x00 \\
0x00e \& 0x00 \\
0x00f \& 0x00 \\
\ldots \& \ldots \\
};
\node[mem] (imem) at ([xshift=4cm]mem.east) {Instr.\\Mem.};
\coordinate (imemData) at (imem.east);
\coordinate (imemAddr) at (imem.west);
\draw[thick,-latex] (imemData) -- +(1cm,0cm) node [font=\small\tt,right] (i10b label) { i10bytes };
\draw[thick,latex-] (imemAddr) -- +(-1cm,0cm) node [font=\small\tt,left] (pc label) { pc };
\node[anchor=north,font=\small] at (i10b label.south) {(data)};
\node[anchor=north,font=\small] at (pc label.south) {(address)};
\begin{scope}
\path[clip] (mem.south east) -- ++ (10cm, 0cm) |- (mem.north east);
\draw[ultra thick,dotted] (imem.north east) -- (mem.north east);
\draw[ultra thick,dotted] (imem.south east) -- (mem.south east);
\draw[ultra thick,dotted] (imem.north west) -- (mem.north west);
\draw[ultra thick,dotted] (imem.south west) -- (mem.south west);
\end{scope}
\begin{visibleenv}<2->
\matrix[
    tight matrix,nodes={font=\fontsize{9}{10}\tt\selectfont},anchor=north west,
    column 2/.style={nodes={text width=4.25cm}},
    row 2/.style={alt=<3-6>{nodes={fill=blue!10}}},
    row 3/.style={alt=<7-9>{nodes={fill=blue!10}}},
    ] (mem result) 
    at ([xshift=7.5cm]mem.north east) {
pc \& i10bytes \\
0x000 \& 0x01000000000021\myemph<6>{61}\myemph<5>{12}\myemph<4>{60} \\
0x001 \& 0x00010000000000\myemph<9>{21}\myemph<8>{61}\myemph<7>{12} \\
0x002 \& 0x00000100000000002161 \\
0x003 \& 0x00000001000000000021 \\
\ldots \& \ldots \\
};
\end{visibleenv}
\end{tikzpicture}
\end{frame}



\subsection{extracting bits?}
\againframe<3>{nopHltCPU}

\begin{frame}{subsetting bits in HCLRS}
    \begin{itemize}
    \item extracting bits 2 (inclusive)--9 (exclusive): {\tt value[2..9]}
    \item \myemph{least significant bit} is bit 0
    \end{itemize}
\end{frame}



\subsection{instruction bit numbering}
\usetikzlibrary{chains,fit,positioning}

\begin{frame}[fragile,label=altView]{example}
\instrEncodingStyles
    \begin{itemize}
    \item {\tt pushq \%rbx} at memory address $x$: \opify{A}\literalify{F} \rnifyWide{2}\literalify{F}
    \item memory at $x+0$: \opifyWide{pushq} \literalify{F}; at $x+1$: \rnifyWide{rbx} \literalify{F}
    \item $x+0$: \opify{A}\literalify{F}; at $x+1$: \rnify{2}\literalify{F}
    \item as a little-endian 2-byte number in typical English order:
\begin{tikzpicture}
\begin{scope}[every node/.style={font=\large\tt,inner sep=1mm,on chain},start chain=going right,node distance=10mm]
\node[register] (n1) {2}; \node[literal] (n2) {F}; \node[opcode] (n3) {A}; \node[literal] (n4) {F};
\end{scope}
\begin{scope}[every node=/style={font=\tt}]
\node[below=2mm of n1] (n1b){ 0010 };
\node[below=2mm of n2] (n2b) { 1111 };
\node[below=2mm of n3] (n3b) { 1010 };
\node[below=2mm of n4] (n4b) { 1111 };
\end{scope}
\draw[Latex-,very thick] (n1b.south west) -- ++(-.5cm,-.5cm) node[below,font=\small,align=center] {most sig. bit\\(bit 15)};
\draw[Latex-,very thick] (n4b.south east) -- ++(.5cm,-.5cm) node[below,font=\small,align=center] {least sig. bit\\(bit 0)};
\end{tikzpicture}
    \end{itemize}
\end{frame}


\begin{frame}{Y86 encoding table}
\begin{tikzpicture}
\instrEncodingStyles
\instrEncodingTable
\begin{visibleenv}<2-4>
\node[draw=red,thick,fit=(table-1-2) (table-13-3)] (byte0Mark) {};
\end{visibleenv}
\begin{visibleenv}<2>
    \node[below=0cm of byte0Mark,text=red!60!black] {byte 0: bits 0--7};
\end{visibleenv}
\begin{visibleenv}<3>
\node[draw=blue,inner sep=2pt,very thick,fit=(table-2-3) (table-13-3)] (byte00Mark) {};
\end{visibleenv}
\begin{visibleenv}<3>
\node[below=0cm of byte00Mark,text=blue!60!black,align=center] {least sig. 4 bits of byte 0: bits 0--4};
\end{visibleenv}

\begin{visibleenv}<4>
\node[draw=blue,inner sep=2pt,very thick,fit=(table-2-2) (table-13-2)] (byte01Mark) {};
\end{visibleenv}
\begin{visibleenv}<4>
\node[below=0cm of byte01Mark,text=blue!60!black,align=center] {most sig. 4 bits of byte 0: bits 4--8};
\end{visibleenv}

\begin{visibleenv}<5>
\node[draw=blue,inner sep=2pt,very thick,fit=(table-2-4) (table-13-4)] (byte10Mark) {};
\end{visibleenv}
\begin{visibleenv}<5>
\node[below=0cm of byte10Mark,text=blue!60!black,align=center] {most sig. 4 bits of byte 1: bits 12--16};
\end{visibleenv}

\begin{visibleenv}<6>
\node[draw=blue,inner sep=2pt,very thick,fit=(table-2-5) (table-13-5)] (byte11Mark) {};
\end{visibleenv}
\begin{visibleenv}<6>
\node[below=0cm of byte11Mark,text=blue!60!black,align=center] {least sig. 4 bits of byte 1: bits 8--12};
\end{visibleenv}
\end{tikzpicture}
\end{frame}

\begin{frame}{Y86 encoding table (written differently)}
\begin{tikzpicture}
\instrEncodingStyles
\instrEncodingTableReversed
\begin{visibleenv}<2-4>
\node[draw=red,thick,fit=(table-1-20) (table-13-21)] (byte0Mark) {};
\end{visibleenv}
\begin{visibleenv}<2>
    \node[below=0cm of byte0Mark,text=red!60!black] {byte 0: bits 0--7};
\end{visibleenv}
\begin{visibleenv}<3>
\node[draw=blue,inner sep=2pt,very thick,fit=(table-2-21) (table-13-21)] (byte00Mark) {};
\end{visibleenv}
\begin{visibleenv}<3>
\node[below=0cm of byte00Mark,text=blue!60!black,align=center] {least sig. 4 bits of byte 0: bits 0--4};
\end{visibleenv}

\begin{visibleenv}<4>
\node[draw=blue,inner sep=2pt,very thick,fit=(table-2-20) (table-13-20)] (byte01Mark) {};
\end{visibleenv}
\begin{visibleenv}<4>
\node[below=0cm of byte01Mark,text=blue!60!black,align=center] {most sig. 4 bits of byte 0: bits 4--8};
\end{visibleenv}

\begin{visibleenv}<5>
\node[draw=blue,inner sep=2pt,very thick,fit=(table-2-18) (table-13-18)] (byte10Mark) {};
\end{visibleenv}
\begin{visibleenv}<5>
\node[below=0cm of byte10Mark,text=blue!60!black,align=center] {most sig. 4 bits of byte 1: bits 12--16};
\end{visibleenv}

\begin{visibleenv}<6>
\node[draw=blue,inner sep=2pt,very thick,fit=(table-2-19) (table-13-19)] (byte11Mark) {};
\end{visibleenv}
\begin{visibleenv}<6>
\node[below=0cm of byte11Mark,text=blue!60!black,align=center] {least sig. 4 bits of byte 1: bits 8--12};
\end{visibleenv}
\end{tikzpicture}
\end{frame}


    % FIXME: inverted instruction table

\subsection{finishing the CPU}

\againframe<4-6>{nopHltCPU}

\begin{frame}{demo}
\end{frame}

\section{nop+jmp(+halt) CPU}

\subsection{extending the nop+halt CPU}
\providecommand{\nopJmpSetup}{
    \tikzset{
        dmemNorm/.style={visible on=<0|handout:0>},
        ccsNorm/.style={visible on=<0|handout:0>},
        isStat/.style={visible on=<0|handout:0>},
        isStatReg/.style={visible on=<0|handout:0>},
        instrRegsPre/.style={visible on=<0|handout:0>},
        regNorm/.style={visible on=<0|handout:0>},
        regNormLabel/.style={visible on=<0|handout:0>},
        imemPcPre/.style={visible on=<1|handout:1>},
        regPre/.style={visible on=<0|handout:0>},
        hiOver/.style={opacity=0.2,fill=green},
        bookLabel/.style={color=red!60!black,font=\small\bfseries,outer sep=0pt,inner sep=1pt,fill=white},
        dest wire/.style={},
        plus one wire/.style={},
        pc mux/.style={},
    }
}
\providecommand{\nopJmpBase}{
    \node[hRegSmall=PC,pcStyle] (pc) {};
    \node[mem,right=2.cm of pc,font=\scriptsize] (imem) {Instr. \\ Mem.};
    \coordinate (imemData) at (imem.east);
    \coordinate (imemAddr) at (imem.west);
    \coordinate (beforePC) at ([xshift=-1cm]pc);
    \draw[a] (pc.east) -- (imemAddr);
    \node[logicBlock,right=.5cm of imemData] (split) {split};
    \draw[a] (imemData) -- (split);
}
\providecommand{\nopJmpStat}{
    \begin{pgfonlayer}{fg}
        \draw[b] ([yshift=-.1cm]split.east) -| ++ (0.2cm, -2cm) node[logicBlock,below,font=\small,align=center] {{\tt Stat} \\ logic} node[pos=0.7,fill=white,inner sep=0.2mm,font=\tt\scriptsize] {icode};
    \end{pgfonlayer}
}
\providecommand{\nopJmpStatSmall}{
    \draw[b] ([yshift=-.1cm]split.east) -| ++ (0.2cm, -1.5cm) node[logicBlock,below,font=\tiny,align=center] {{\tt Stat} \\ logic};
}
\providecommand{\nopJmpMux}{
        \node[mux,minimum height=1.2cm,minimum width=.65cm,inputs={nn},info={center:\scriptsize MUX},right=2cm of split,logicFill,pc mux] (nextPCMux) {~};
}
\providecommand{\nopJmpMuxSelect}{
        \node[logicBlock,below right=1.2cm and .6cm of split,anchor=west,align=left,font=\small] (testJump) {0 if nop \\ 1 if jmp};
        \draw[b] (testJump.east) -| (nextPCMux.select);
        \draw[b] ([yshift=-.1cm]split.east) -- ++ (0.2cm,0cm) |- (testJump)
            node[inner sep=0.5pt,outer sep=1pt,near start,below,fill=white,font=\scriptsize] (opcodeLabel) { \tt icode };
        \draw[a,dest wire] ([yshift=.1cm]split.east) -- ++(0.4cm,0cm) |- (nextPCMux.input 2)
            node[inner sep=0pt,outer sep=1pt,near end,fill=white,font=\scriptsize] (destLabel){ dest };
}
\providecommand{\nopJmpNopOnlyPlus}{
    \node[logicBlock,anchor=north,font=\small,align=center] (pcPlus) at ([yshift=-1cm]pc.south) {+ 1 \\ {\footnotesize (nop size)}};
    \draw[a] (pc.east) -- ++ (1cm, 0cm) |- (pcPlus.east);
    \draw[a] (pcPlus.west) -| (beforePC) -- (pc.west);
}
\providecommand{\nopJmpNopPlusPre}{
    \node[plus one wire,logicBlock,above right=.5cm and -.5cm of imem,font=\small] (pcPlus) {+ 1 {\footnotesize (nop size)}};
    \draw[plus one wire,a] (pc.east) -- ++ (1cm, 0cm) |- (pcPlus.west);
}
\providecommand{\nopJmpNopPlusPost}{
    \draw[plus one wire,a] (pcPlus.east) -- ++ (1cm, 0cm) |- (nextPCMux.input 1);
    \draw[a] (nextPCMux.output) -| ++(.5cm, 2cm) -| (beforePC) -- (pc.west);
}
\providecommand{\nopJmpNopPlusPostNoMux}{
    \draw[plus one wire,a] (pcPlus.east) -- ++(.5cm, 0cm) |- ([yshift=2cm]split.center) -| (beforePC) -- (pc.west);
}

\begin{frame}[fragile,label=nopJmpCPU]{nop/halt $\rightarrow$ nop/jmp CPU}
    \begin{tikzpicture}[circuit logic US]
        \nopJmpSetup
        \nopJmpBase
        \nopJmpStat
        \begin{visibleenv}<3->
            \nopJmpMux
        \end{visibleenv}
        \begin{visibleenv}<3>
            \draw[aR,red] (nextPCMux.input 2) -| ++(-.5cm,-1cm) node[below,font=\scriptsize] {jmp PC?};
        \end{visibleenv}
        \begin{visibleenv}<4->
            \nopJmpMuxSelect
        \end{visibleenv}
        \begin{visibleenv}<1>
            \nopJmpNopOnlyPlus
        \end{visibleenv}
        \begin{visibleenv}<2->
            \nopJmpNopPlusPre
        \end{visibleenv}
        \begin{visibleenv}<2>
            \nopJmpNopPlusPostNoMux
            \draw[a] (pcPlus.east) -- ++(.5cm, 0cm) |- ([yshift=2cm]split.center) -| (beforePC) -- (pc.west);
        \end{visibleenv}
        \begin{visibleenv}<3->
            \nopJmpNopPlusPost
        \end{visibleenv}
    \end{tikzpicture}
\end{frame}


\subsection{in HCLRS}
\providecommand{\nopJmpSetup}{
    \tikzset{
        dmemNorm/.style={visible on=<0|handout:0>},
        ccsNorm/.style={visible on=<0|handout:0>},
        isStat/.style={visible on=<0|handout:0>},
        isStatReg/.style={visible on=<0|handout:0>},
        instrRegsPre/.style={visible on=<0|handout:0>},
        regNorm/.style={visible on=<0|handout:0>},
        regNormLabel/.style={visible on=<0|handout:0>},
        imemPcPre/.style={visible on=<1|handout:1>},
        regPre/.style={visible on=<0|handout:0>},
        hiOver/.style={opacity=0.2,fill=green},
        bookLabel/.style={color=red!60!black,font=\small\bfseries,outer sep=0pt,inner sep=1pt,fill=white},
        dest wire/.style={},
        plus one wire/.style={},
        pc mux/.style={},
    }
}
\providecommand{\nopJmpBase}{
    \node[hRegSmall=PC,pcStyle] (pc) {};
    \node[mem,right=2.cm of pc,font=\scriptsize] (imem) {Instr. \\ Mem.};
    \coordinate (imemData) at (imem.east);
    \coordinate (imemAddr) at (imem.west);
    \coordinate (beforePC) at ([xshift=-1cm]pc);
    \draw[a] (pc.east) -- (imemAddr);
    \node[logicBlock,right=.5cm of imemData] (split) {split};
    \draw[a] (imemData) -- (split);
}
\providecommand{\nopJmpStat}{
    \begin{pgfonlayer}{fg}
        \draw[b] ([yshift=-.1cm]split.east) -| ++ (0.2cm, -2cm) node[logicBlock,below,font=\small,align=center] {{\tt Stat} \\ logic} node[pos=0.7,fill=white,inner sep=0.2mm,font=\tt\scriptsize] {icode};
    \end{pgfonlayer}
}
\providecommand{\nopJmpStatSmall}{
    \draw[b] ([yshift=-.1cm]split.east) -| ++ (0.2cm, -1.5cm) node[logicBlock,below,font=\tiny,align=center] {{\tt Stat} \\ logic};
}
\providecommand{\nopJmpMux}{
        \node[mux,minimum height=1.2cm,minimum width=.65cm,inputs={nn},info={center:\scriptsize MUX},right=2cm of split,logicFill,pc mux] (nextPCMux) {~};
}
\providecommand{\nopJmpMuxSelect}{
        \node[logicBlock,below right=1.2cm and .6cm of split,anchor=west,align=left,font=\small] (testJump) {0 if nop \\ 1 if jmp};
        \draw[b] (testJump.east) -| (nextPCMux.select);
        \draw[b] ([yshift=-.1cm]split.east) -- ++ (0.2cm,0cm) |- (testJump)
            node[inner sep=0.5pt,outer sep=1pt,near start,below,fill=white,font=\scriptsize] (opcodeLabel) { \tt icode };
        \draw[a,dest wire] ([yshift=.1cm]split.east) -- ++(0.4cm,0cm) |- (nextPCMux.input 2)
            node[inner sep=0pt,outer sep=1pt,near end,fill=white,font=\scriptsize] (destLabel){ dest };
}
\providecommand{\nopJmpNopOnlyPlus}{
    \node[logicBlock,anchor=north,font=\small,align=center] (pcPlus) at ([yshift=-1cm]pc.south) {+ 1 \\ {\footnotesize (nop size)}};
    \draw[a] (pc.east) -- ++ (1cm, 0cm) |- (pcPlus.east);
    \draw[a] (pcPlus.west) -| (beforePC) -- (pc.west);
}
\providecommand{\nopJmpNopPlusPre}{
    \node[plus one wire,logicBlock,above right=.5cm and -.5cm of imem,font=\small] (pcPlus) {+ 1 {\footnotesize (nop size)}};
    \draw[plus one wire,a] (pc.east) -- ++ (1cm, 0cm) |- (pcPlus.west);
}
\providecommand{\nopJmpNopPlusPost}{
    \draw[plus one wire,a] (pcPlus.east) -- ++ (1cm, 0cm) |- (nextPCMux.input 1);
    \draw[a] (nextPCMux.output) -| ++(.5cm, 2cm) -| (beforePC) -- (pc.west);
}
\providecommand{\nopJmpNopPlusPostNoMux}{
    \draw[plus one wire,a] (pcPlus.east) -- ++(.5cm, 0cm) |- ([yshift=2cm]split.center) -| (beforePC) -- (pc.west);
}

\begin{frame}[fragile,label=nopJmpCPUHCL]{nop/jmp CPU}
    \begin{tikzpicture}[circuit logic US]
        \nopJmpSetup
        \tikzset{
            dest wire/.style={alt=<3>{red}},
            plus one wire/.style={alt=<2>{red}},
            pc mux/.style={alt=<4>{draw=red,fill=red!10}},
        }
        \nopJmpBase
        %\nopJmpStat
        \nopJmpStatSmall
        \nopJmpMux
        \nopJmpNopPlusPre
        \nopJmpNopPlusPost
        \nopJmpMuxSelect
        \coordinate (textLoc)  at ([yshift=-1cm,xshift=-1cm]pc.south west);
\begin{scope}[every node/.style={font=\fontsize{9}{10}\selectfont\tt,align=left,anchor=north west,inner sep=.25mm}]
\node[at={(textLoc)}] (allDecl) {
{\bfseries wire} valP : 64; \\
{\bfseries wire} icode : 4, dest: 64; \\
{\bfseries register} \textcolor{blue!80!black}{p}\textcolor{green!60!black}{P} \{ \\
\hspace{.5cm} thePc : 64 = 0; \\
\} \\
icode = i10bytes[4..8]; \\
\myemph<3>{dest = i10bytes[8..72]}; \\
\myemph<4>{valP = [} \\
\hspace{.5cm} \myemph<4>{icode == NOP : \myemph<2>{P\_thePc + 1};} \\
\hspace{.5cm} \myemph<4>{icode == JXX : \myemph<3>{dest};} \\
\hspace{.5cm} \myemph<4>{1: \myemph<5>{0xBADBADBAD};} \\
\myemph<4>{]}; \\
p\_thePc = valP; \\
pc = P\_thePc; 
};
\node[at={([yshift=-2.5cm,xshift=.2cm]allDecl.north east)},anchor=north west] {
Stat = [ \\
\hspace{.5cm} (icode == NOP || \\
\hspace{.6cm}  icode == JXX) : STAT\_AOK; \\
\hspace{.5cm} icode == HALT : STAT\_HLT; \\
\hspace{.5cm} 1 : STAT\_INS; \\
];
};
\end{scope}
\end{tikzpicture}
\end{frame}


\subsection{nop+jmp CPU}
\begin{frame}{demo: running nop/jmp}
\end{frame}
\begin{frame}{demo: debug and interactive mode}
\end{frame}
\begin{frame}{demo: yis}
\end{frame}
\begin{frame}[fragile,label=hcl2DInteractiveA]{running nop/jmp/halt}
    \begin{itemize}
    \item {\tt nopjmp.ys}:
\begin{Verbatim}[fontsize=\small,commandchars=\\\{\}]
    nop
    jmp C
B:  jmp D
C:  jmp B
D:  nop
    nop
    halt
\end{Verbatim}
    \item \ldots assemble with {\tt yas}
\end{itemize}
\end{frame}

\begin{frame}[fragile,label=hcl2DInteractiveB]{nopjmp.yo}
\begin{itemize}
    \item {\tt nopjmp.yo}:
\begin{Verbatim}[fontsize=\small,commandchars=\\\{\}]
\vemphA{0x000: 10}                   |     nop
\vemphA{0x001: 701300000000000000}   |     jmp C
\vemphA{0x00a: 701c00000000000000}   | B:  jmp D
\vemphA{0x013: 700a00000000000000}   | C:  jmp B
\vemphA{0x01c: 10}                   | D:  nop
\vemphA{0x01d: 10}                   |     nop
\vemphA{0x01e: 00}                   |     halt
\end{Verbatim}
\end{itemize}
\end{frame}

\begin{frame}[fragile,label=runningNopJmp]{running nopjmp.yo}
\begin{Verbatim}[fontsize=\fontsize{8}{9}\selectfont]
$ ./hclrs nopjmp_cpu.hcl nopjmp.yo
...
...
+--------------------- (end of halted state) ---------------------------+
Cycles run: 7
\end{Verbatim}
\end{frame}


\section{addq CPU}

\subsection{new ISA: addq}

\begin{frame}[fragile,label=AddIntro]{simple ISA: addq}
    \instrEncodingStyles
    \begin{itemize}
        \item \lstinline|addq %rXX, %rYY|
        \item encoding: \opify{6} \literalify{0} \rnifyWide{\%rXX} \rnifyWide{\%rYY} (two 4-bit register \#s)
            \begin{itemize}
            \item 2 byte instructions, no opcode
            \end{itemize}
        \vspace{.5cm}
        \item for now: no other instructions
            \begin{itemize}
            \item later: adding support for nop+halt
            \end{itemize}
    \end{itemize}
\end{frame}


\subsection{the regsiter file}
\usetikzlibrary{decorations.pathreplacing,patterns}
\begin{frame}[fragile,label=Y86RegFile]{register file}
    \begin{tikzpicture}
        \tikzset{
            readReg/.style={blue!50!black},
            writeReg/.style={green!50!black}
        }
        \node[regFile,minimum height=5cm] (regs) {register file \\ \scriptsize \%rax, \%rdx, \ldots{}};
        \coordinate (regSelect1) at ($(regs.north west) - (0cm, .5cm)$);
        \coordinate (regSelect2) at ($(regs.north west) - (0cm, 1cm)$);
        \coordinate (regSelect3) at ($(regs.north west) - (0cm, 1.5cm)$);
        \coordinate (regSelect4) at ($(regs.north west) - (0cm, 2cm)$);
        \coordinate (regWriteIn1) at ($(regs.north west) - (0cm, 3cm)$);
        \coordinate (regWriteIn2) at ($(regs.north west) - (0cm, 3.5cm)$);
        \coordinate (regRead1) at ($(regs.north east) - (0cm, .5cm)$);
        \coordinate (regRead2) at ($(regs.north east) - (0cm, 1cm)$);
        \foreach \x in {regSelect1,regSelect2} {
            \draw[readReg,latex-] (\x) -- +(-.5cm,0cm);
        }
        \foreach \x in {regSelect3,regSelect4} {
            \draw[writeReg,latex-] (\x) -- +(-.5cm,0cm);
        }
        \foreach \x in {regRead1,regRead2} {
            \draw[readReg,thick,-latex] (\x) -- +(.5cm,0cm);
        }
        \node[readReg,ll,above right=2pt of regRead1,outer sep=1pt,inner sep=0pt] {reg values};
        \draw[writeReg,thick,latex-] (regWriteIn1) -- +(-.5cm,0cm);
        \draw[writeReg,thick,latex-] (regWriteIn2) -- +(-.5cm,0cm);
        \node[readReg,ll,above left=2pt of regSelect1,outer sep=1pt,inner sep=0pt] (regNumLabel) {read reg \#s};
        \node[writeReg,ll,above left=2pt of regSelect3,outer sep=1pt,inner sep=0pt] (regNumLabel) {write reg \#s};
        \node[writeReg,ll,above left=2pt of regWriteIn1,outer sep=1pt,inner sep=0pt,align=right] {data to write};

        \begin{visibleenv}<2>
            \draw[red,thick,latex-] (regSelect1) -- ++(-.5cm,0cm);
            \draw[red,thick,-latex] (regRead1) -- ++(.5cm,0cm);
        \end{visibleenv}
        \begin{visibleenv}<2->
            \begin{scope}[shift={($(regs.south east) + (-1, -0.5)$)}] 
                \node[anchor=east,font=\small,alt=<2>{red}] at (0,-.25) {register number input};
                \fill[color=red!80!black] (0.1, 0) rectangle (2.0, -.5);
                \fill[pattern=north west lines] (0.0, -.6) rectangle (0.9, -1.1);
                \fill[color=blue!80!black] (0.9, -.6) rectangle (2.0, -1.1);
                \node[anchor=east,font=\small,alt=<2>{red}] at (0,-.85) {register value output};
                \draw[thin,-latex] (0.0, -1.3) -- (1.5, -1.3) node[right,font=\scriptsize] {time};
            \end{scope}
        \end{visibleenv}

        \begin{visibleenv}<3>
            \draw[red,thick,latex-] (regSelect3) -- ++(-.5cm,0cm);
            \draw[red,thick,latex-] (regWriteIn1) -- ++(-.5cm,0cm);
        \end{visibleenv}

        \begin{visibleenv}<3->
        \begin{scope}[shift={($(regs.south east) + (6, -0.4)$)}] 
            \draw[very thick] (0, 0) -- (1, 0) -- (1, 1) -- (2, 1) -- (2, 0) -- (2.3, 0);
            \draw[very thick,opacity=0.2] (2.3,0) -- (3,0) -- (3,1) --  (4,1);
            \draw[ultra thick,red!95!black] (1, 0) -- (1, 1);
        \end{scope}
        \begin{scope}[shift={($(regs.south east) + (6, -0.5)$)}] 
            \node[anchor=east,font=\small,alt=<3>{red}] at (0,-.25) {register number input};
            \fill[color=orange!70!black] (0.1, 0) rectangle (1.03, -.5);
            \node[anchor=east,font=\small,alt=<3>{red}] at (0,-.85) {data input};
            \fill[pattern=north west lines] (0.0, -0.6) rectangle (0.2, -1.1);
            \fill[color=green!60!white] (0.2, -.6) rectangle (1.02, -1.1);
            \node[anchor=east,font=\small] at (0,-1.45) {value in register};
            \fill[color=blue!80!black] (0, -1.2) rectangle (0.99, -1.7);
            \fill[color=green!60!white] (1.0, -1.2) rectangle (2.3, -1.7);
        \end{scope}
        \end{visibleenv}
        
        \begin{visibleenv}<4>
            \node[align=left] at ([xshift=4cm]regs.east) {
                write register \#15 (REG\_NONE): \\ write is ignored \\ ~ \\
                read register \#15 (REG\_NONE): \\ value is always 0
            };
        \end{visibleenv}
        \begin{visibleenv}<5>
            \tikzset{
                hcl name/.style={font=\tt\fontsize{9}{10}\selectfont,orange!50!black},
            }
            \node[hcl name,anchor=east] at ([xshift=-1.4cm]regSelect1) {reg\_srcA};
            \node[hcl name,anchor=east] at ([xshift=-1.4cm]regSelect2) {reg\_srcB};
            \node[hcl name,anchor=east] at ([xshift=-1.4cm]regSelect3) {reg\_dstE};
            \node[hcl name,anchor=east] at ([xshift=-1.4cm]regSelect4) {reg\_dstM};

            \node[hcl name,anchor=east] at ([xshift=-1cm,yshift=-1mm]regWriteIn1) {reg\_inputE};
            \node[hcl name,anchor=east] at ([xshift=-1cm,yshift=-1mm]regWriteIn2) {reg\_inputM};

            \node[hcl name,anchor=west] at ([xshift=1cm]regRead1) {reg\_outputA};
            \node[hcl name,anchor=west] at ([xshift=1cm]regRead2) {reg\_outputB};

            \draw[decorate,decoration={brace,mirror},ultra thick,orange!50!black]
                ([xshift=-1cm,yshift=.5cm]regSelect1)
                --  ++(-2cm,0cm) node[midway,above,font=\small,align=center] (names label) {HCLRS\\names};
            \draw[decorate,decoration={brace},ultra thick,orange!50!black]
                ([xshift=1.2cm,yshift=.5cm]regRead1)
                --  ++(2cm,0cm) ;
            \node[orange!50!black,anchor=west,font=\fontsize{9}{10}\selectfont] at ([xshift=1cm]names label.east) {
                (also in HCLRS README on website)
            };

        \end{visibleenv}
    \end{tikzpicture}
\end{frame}


\subsection{addq processor}
\begin{frame}[fragile,label=AddCPU]{addq CPU}
    \instrEncodingStyles
    \begin{tikzpicture}
        \tikzset{
            dmemNorm/.style={visible on=<0|handout:0>},
            dmemPre/.style={visible on=<0|handout:0>},
            ccsNorm/.style={visible on=<0|handout:0>},
            isStat/.style={visible on=<0|handout:0>},
            isStatReg/.style={visible on=<0|handout:0>},
            regPre/.style={visible on=<0|handout:0>},
            regNormLabelM/.style={visible on=<0|handout:0>},
            regPreSingle/.style={visible on=<0|handout:0>},
            instrRegsPre/.style={visible on=<1-2|handout:0>},
            regPre/.style={visible on=<1-3|handout:0>},
            hiOver/.style={opacity=0.2,fill=green},
        }
        \circuitState
        \circuitStatePre
        \node[hiOver,fit=(pc.east) (imem.west),visible on=<2|handout:0>] {};
        \begin{visibleenv}<3->
            \node[above right=2cm and 0cm of imem.east,font=\small] (instrSplit) {\rnifyWide{\%rXX}~~\rnifyWide{\%rYY}};
            \draw[very thick,dashed,-Latex] (instrSplit) -- ([xshift=.15cm]imem.east);
            \draw (imem.east) -- ++(0.35cm, 0cm);
        \end{visibleenv}
        \begin{visibleenv}<2->
            \draw[a] (pc.east) -- (imemAddr);
            \node[logicBlock,right=0.5cm of imem] (split) {split};
            \draw[a] (imem) -- (split);
        \end{visibleenv}
        \begin{visibleenv}<4->
            \draw[b] ([yshift=.1cm]split.east) -- ([yshift=.1cm,xshift=.1cm]split.east) |- (regSelect1);
            \draw[b] ([yshift=-.1cm]split.east) -- ([yshift=-.1cm,xshift=.2cm]split.east) |- (regSelect2);
        \end{visibleenv}
        \begin{visibleenv}<2->
            \node[logicBlock,align=center,anchor=north,font=\small] (proc opcode) at ([yshift=-.5cm]split.south) {
                convert\\opcode
            };
            \draw[b] (split.south) -- (proc opcode.north);
            \draw[b] (proc opcode.south) -- ++(0cm, -.25cm) node[below,font=\fontsize{9}{10}\selectfont\tt] {Stat};
        \end{visibleenv}
        \begin{visibleenv}<5->
            \coordinate (regReadMid) at ($(regRead1)!0.5!(regRead2)$);
            \node[logicBlock,right=1.5cm of regReadMid,align=center] (add) {add};
            \draw[a] (regRead1) -- (add);
            \draw[a] (regRead2) -- (add);
        \end{visibleenv}
        \begin{visibleenv}<6->
            \coordinate (addBelowRight) at ($(add.south east) + (.5cm,-3.5cm)$);
            \coordinate (regWriteIn2Left)  at ($(regWriteIn2) + (-.8cm,0cm)$);
            \coordinate (addBelowLeft) at (addBelowRight -| regWriteIn2Left);
            \draw[a] (add.east) -| (addBelowRight) -- (addBelowLeft) -- (regWriteIn2Left) |- (regWriteIn2);
            \draw[b] ([yshift=-.1cm]split.east) -- ([yshift=-.1cm,xshift=.2cm]split.east) |- (regSelect4);
        \end{visibleenv}
        \begin{visibleenv}<7>
            \begin{scope}[every node/.style={font=\scriptsize,anchor=south east,inner sep=0mm}]
            \node at (regSelect1) {0 (=\%rax)};
            \node at (regSelect2) {2 (=\%rdx)};
            \node at (regSelect4) {2 (=\%rdx)};
            \end{scope}
            \draw ([xshift=4mm]regRead1) -- ++ (.5cm, .5cm)
                node[above,font=\scriptsize,align=center,inner sep=0mm] {10 \\ (value of RAX)};
            \draw ([xshift=4mm]regRead2) -- ++ (.5cm, -.5cm)
                node[below,font=\scriptsize,align=center,inner sep=0mm] {30 \\ (value of RDX)};
            \node[below right=1.5cm and -.5cm of pc,align=left,font=\small,fill=white, fill opacity=0.9] (traceBrokenPC) {
\begin{lstlisting}
/* 0x00: */ addq %rax, %rdx 
/* 0x02: */ addq %rbx, %rdx
\end{lstlisting}
                \\
                \begin{tabular}{l@{\hspace{.1cm}}llll}
                initially:&PC = 0x00,& rax = 10,& rbx = 20,&rdx = 30 \\
                after cycle 1:&PC = ????,& rax = 10,& rbx = 20,&\myemph{rdx = 40} \\
                \end{tabular}
            };
        \end{visibleenv}
        \begin{visibleenv}<8>
            \node[below right=1.5cm and -.5cm of pc,align=left,font=\small,fill=white,fill opacity=0.9] (traceBrokenPC) {
\begin{lstlisting}
/* 0x00: */ addq %rax, %rdx 
/* 0x02: */ addq %rbx, %rdx
\end{lstlisting}
                \\
                \begin{tabular}{l@{\hspace{.1cm}}llll}
                initially:&PC = 0x00,& rax = 10,& rbx = 20,&rdx = 30 \\
                after cycle 1:&PC = ????,& rax = 10,& rbx = 20,&\myemph{rdx = 40} \\
                after cycle 2:&PC = ????,& rax = ??,& rbx = ??,&rdx = ?? \\
                \end{tabular}
            };
        \end{visibleenv}
        \begin{visibleenv}<9->
            \node[logicBlock,above=.5cm of pc,font=\small,xshift=-.2cm] (addOne) {+2};
            \draw[a] (pc.east) -- +(.8cm, 0cm) |- (addOne.east);
            \draw[a] (addOne.west) -- +(-.5cm, 0cm) |- (pc.west);
        \end{visibleenv}
        \begin{visibleenv}<10>
            \node[below right=1.5cm and -.5cm of pc,align=left,font=\small] (traceUnbrokenPC) {
\begin{lstlisting}
/* 0x00: */ addq %rax, %rdx 
/* 0x02: */ addq %rbx, %rdx
\end{lstlisting}
                \\
                \begin{tabular}{l@{\hspace{.1cm}}llll}
                initially:&PC = 0x00,& rax = 10,& rbx = 20,&rdx = 30 \\
                after cycle 1:&PC = 0x02,&rax = 10,& rbx = 20,&\myemph{rdx = 40} \\
                after cycle 2:&PC = 0x04,&rax = 10,& rbx = 20,&\myemph{rdx = 60} \\
                \end{tabular}
            };
        \end{visibleenv}
    \end{tikzpicture}
\end{frame}

 % FIXME: w/ with halt base

\subsection{\ldots in HCLRS}
\begin{frame}[fragile,label=AddCPUHCL]{addq CPU: HCL}
    \instrEncodingStyles
    \begin{tikzpicture}
        \tikzset{
            dmemNorm/.style={visible on=<0|handout:0>},
            dmemPre/.style={visible on=<0|handout:0>},
            ccsNorm/.style={visible on=<0|handout:0>},
            isStat/.style={visible on=<0|handout:0>},
            isStatReg/.style={visible on=<0|handout:0>},
            regPre/.style={visible on=<0|handout:0>},
            regNormLabelM/.style={visible on=<0|handout:0>},
            regPreSingle/.style={visible on=<0|handout:0>},
            instrRegsPre/.style={visible on=<0|handout:0>},
            regPre/.style={visible on=<0|handout:0>},
            hiOver/.style={opacity=0.2,fill=green},
            hiOverS/.style={opacity=0.2,fill=green,inner sep=0.25mm},
        }
        \circuitState
        \circuitStatePre
        \begin{visibleenv}<1->
            \node[above right=2cm and 0cm of imem.east,font=\small] (instrSplit) {\rnifyWide{\%rXX}~~\rnifyWide{\%rYY}};
            \draw[very thick,dashed,-Latex] (instrSplit) -- ([xshift=.15cm]imem.east);
            \draw (imem.east) -- ++(0.35cm, 0cm);
        \end{visibleenv}
        \begin{visibleenv}<1->
            \draw[a] (pc.east) -- (imemAddr);
            \node[logicBlock,right=0.5cm of imem] (split) {split};
            \draw[a] (imem) -- (split);
        \end{visibleenv}
        \begin{visibleenv}<1->
            \draw[b] ([yshift=.1cm]split.east) -- ([yshift=.1cm,xshift=.1cm]split.east) |- (regSelect1);
            \draw[b] ([yshift=-.1cm]split.east) -- ([yshift=-.1cm,xshift=.2cm]split.east) |- (regSelect2);
        \end{visibleenv}
        \begin{visibleenv}<1->
            \node[logicBlock,align=center,anchor=north,font=\small] (proc opcode) at ([yshift=-.5cm]split.south) {
                convert\\opcode
            };
            \draw[b] (split.south) -- (proc opcode.north);
            \draw[b] (proc opcode.south) -- ++(0cm, -.25cm) node[below,font=\fontsize{9}{10}\selectfont\tt] {Stat};
        \end{visibleenv}
        \begin{visibleenv}<1->
            \coordinate (regReadMid) at ($(regRead1)!0.5!(regRead2)$);
            \node[logicBlock,right=1.5cm of regReadMid,align=center] (add) {add};
            \draw[a] (regRead1) -- (add);
            \draw[a] (regRead2) -- (add);
        \end{visibleenv}
        \begin{visibleenv}<1->
            \coordinate (addBelowRight) at ($(add.south east) + (.5cm,-3.5cm)$);
            \coordinate (regWriteIn2Left)  at ($(regWriteIn2) + (-.8cm,0cm)$);
            \coordinate (addBelowLeft) at (addBelowRight -| regWriteIn2Left);
            \draw[a] (add.east) -| (addBelowRight) -- (addBelowLeft) -- (regWriteIn2Left) |- (regWriteIn2);
            \draw[b] ([yshift=-.1cm]split.east) -- ([yshift=-.1cm,xshift=.2cm]split.east) |- (regSelect4);
        \end{visibleenv}
        \begin{visibleenv}<1->
            \node[logicBlock,above=.5cm of pc,font=\small,xshift=-.2cm] (addOne) {+2};
            \draw[a] (pc.east) -- +(.8cm, 0cm) |- (addOne.east);
            \draw[a] (addOne.west) -- +(-.5cm, 0cm) |- (pc.west);
        \end{visibleenv}
        \begin{visibleenv}<2->
        \tikzset{
            code block/.style={
                font=\tt\fontsize{9.5}{10.5}\selectfont,
                inner sep=0mm,
                text height=.7em,
                align=left
            },
        }
        \node[code block,anchor=north west] (pc decl 1) at ([yshift=-1.8cm,xshift=-1cm]pc.south) {
register pP \{
};
        \node[code block,anchor=north west] (pc decl 2) at (pc decl 1.south west) {
\hspace{1cm}pc : 64 = 0;
};
        \node[code block,anchor=north west] (pc decl 3) at (pc decl 2.south west) {
\}
};
        \node[code block,anchor=north west] (pc incr) at (pc decl 3.south west) {
p\_pc = P\_pc + 2;
};
        \node[code block,anchor=north west] (imem in) at (pc incr.south west) {
pc = P\_pc;
};
        \coordinate (col 2 base) at ([xshift=1.5cm]pc decl 1.north east);
        \node[code block,anchor=north west] (decl wires 1) at (col 2 base) {
wire opcode : 4;
        };
        \node[code block,anchor=north west] (decl wires 2) at (decl wires 1.south west) {
wire rA : 4, rB : 4;
        };
        \node[code block,anchor=north west] (split 1) at (decl wires 2.south west) {
opcode = i10bytes[4..8];
};
        \node[code block,anchor=north west] (split 2) at (split 1.south west) {
rA = i10bytes[12..16];
};
        \node[code block,anchor=north west] (split 3) at (split 2.south west) {
 rB = i10bytes[8..12];
};
        \coordinate (col 3 base) at ([xshift=2.5cm]decl wires 1.north east);
        \node[code block,anchor=north west] (reg srcA) at (col 3 base) {
reg\_srcA = rA;
        };
        \node[code block,anchor=north west] (reg srcB) at (reg srcA.south west) {
reg\_srcB = rB;
        };
        \node[code block,anchor=north west] (reg dstE) at (reg srcB.south west) {
reg\_dstE = rB;
        };
        \node[code block,anchor=north west] (reg inputE 1) at (reg dstE.south west) {
reg\_inputE =
        };
        \node[code block,anchor=north west] (reg inputE 2) at (reg inputE 1.south west) {
\hspace{1cm} reg\_outputA +
        };
        \node[code block,anchor=north west] (reg inputE 3) at (reg inputE 2.south west) {
\hspace{1cm} reg\_outputB;
        };
        \end{visibleenv}
        \begin{visibleenv}<3>
            \node[hiOver,fit=(pc)] {};
            \node[hiOverS,fit=(pc decl 1) (pc decl 2) (pc decl 3)] {};
        \end{visibleenv}
        \begin{visibleenv}<4>
            \node[hiOverS,fit=(pc incr)] {};
            \draw[line width=10pt,green,opacity=0.2] (pc.east) -- +(.8cm, 0cm) |- (addOne.east);
            \draw[line width=10pt,green,opacity=0.2] (addOne.west) -- +(-.5cm, 0cm) |- (pc.west);
        \end{visibleenv}
        \begin{visibleenv}<5>
            \node[hiOverS,fit=(imem in)] {};
            \draw[line width=10pt,green,opacity=0.2] (pc.east) -- (imem.west);
        \end{visibleenv}
        \begin{visibleenv}<6>
            \node[hiOverS,fit=(split 1) (split 2) (split 3)] {};
            \begin{pgfonlayer}{bg}
                \begin{visibleenv}<6>
                \node[hiOver,fit=(split)] {};
                \end{visibleenv}
            \end{pgfonlayer}
        \end{visibleenv}
        \begin{visibleenv}<7>
            \node[hiOverS,fit=(reg srcA) (reg srcB)] {};
            \draw[line width=10pt,green,opacity=0.2] ([yshift=.1cm]split.east) -- ([yshift=.1cm,xshift=.1cm]split.east) |- (regSelect1);
            \draw[line width=10pt,green,opacity=0.2] ([yshift=-.1cm]split.east) -- ([yshift=-.1cm,xshift=.2cm]split.east) |- (regSelect2);
        \end{visibleenv}
        \begin{visibleenv}<8>
            \node[hiOverS,fit=(reg dstE) (reg inputE 1) (reg inputE 2) (reg inputE 3)] {};
            \begin{scope}[line width=10pt,green,opacity=0.2]
                \draw (regRead1) -- (add);
                \draw (regRead2) -- (add);
                \draw (add.east) -| (addBelowRight) -- (addBelowLeft) -- (regWriteIn2Left) |- (regWriteIn2);
                \draw ([yshift=-.1cm]split.east) -- ([yshift=-.1cm,xshift=.2cm]split.east) |- (regSelect4);
            \end{scope}
        \end{visibleenv}
    \end{tikzpicture}
\end{frame}


\subsection{HCLRS versus book}
\begin{frame}{differences from book}
    \begin{itemize}
    \item {\textbf wire} not {\textbf bool} or {\textbf int}
    \item book uses names like {\tt valC} --- not required!
        \begin{itemize}
        \item author's environment limited adding new wires
        \end{itemize}
    \item \myemph<2>{MUXes must have default ({\tt 1 : something}) case}
    \item \myemph<3>{implement your own ALU}
    \end{itemize}
\end{frame}



\section{ALUs}
\subsection{abstractly}
\begin{frame}[fragile,label=ALUs]{ALUs}
    \tikzset{alu/.style={trapezium,
            trapezium angle=30,
            shape border rotate=270,
            minimum width=4cm,
            minimum height=3cm,
            trapezium stretches=true,
            append after command={%
                    \pgfextra
                        \draw (\tikzlastnode.top left corner) --
                           (\tikzlastnode.top right corner) -- 
                           (\tikzlastnode.bottom right corner) -- 
                           ($(\tikzlastnode.bottom right corner)!.666!(\tikzlastnode.bottom side)$)--
                           ([xshift=8mm]\tikzlastnode.bottom side)--
                           ($(\tikzlastnode.bottom side)!.334!(\tikzlastnode.bottom left corner)$)--
                           (\tikzlastnode.bottom left corner)--
                           (\tikzlastnode.top left corner);
                    \endpgfextra}}}
    \begin{tikzpicture}
        \node[alu] (alu) {ALU};
        \draw[a] (alu.east) -- ++(0:5mm) node[right] (outLabel) {{\color{blue}A} {\it OP} {\color{green}B}};
        \draw[a,latex-] (alu.130) -- ++(180:5mm) node[left] {\color{blue} A};
        \draw[a,latex-] (alu.230) -- ++(180:5mm) node[left] {\color{green} B};
        \draw[b,latex-] (alu.south) -- ++(-90:2cm) node[below] {operation select};
        \node[draw, rectangle,below right=1cm and .2cm of outLabel, align=left] {
            Operations needed: \\
            add --- \addq, addresses \\
            sub --- \subq \\
            xor --- \xorq \\
            and --- \andq \\
            more?
        };
    \end{tikzpicture}
\end{frame}

\begin{frame}{ALUs not for PC increment}
    \begin{itemize}
    \item our processor will have one ALU
    \vspace{.5cm}
    \item not used for PC increment (computing next instruction address)
        \begin{itemize}
        \item need to do other computation in same cycle
        \item don't need a general circuit for it
        \end{itemize}
    \end{itemize}
\end{frame}


\subsection{in HCLRS}
\begin{frame}{ALUs in HCLRS}
    \begin{itemize}
    \item HCLRS doesn't supply an ALU
        \begin{itemize}
        \item the HCL the textbook authors use does
        \end{itemize}
    \item \ldots but you can build one yourself
        \begin{itemize}
        \item not required --- we check functionality
        \end{itemize}
    \end{itemize}
\end{frame}



\section{review: nop+halt CPU}
\againframe<7>{nopHltCPU}

\section{exercise: nop+add CPU}
\begin{frame}{exercise: nop/add CPU}
\begin{itemize}
\item Let's say we wanted to make a \myemph{add+nop CPU}. Where would we need MUXes? Before\ldots
    \begin{itemize}
    \item \small (modify add CPU to also support the nop instruction)
    \end{itemize}
\vspace{.5cm}
\item A. one or both of the register file `register number to read' inputs {\small (reg\_src\ldots)}
\item B. the PC register's input {\small (p\_pc)}
\item C. one of the register file `register number to write' inputs {\small (reg\_dst\ldots)}
\item D. one of the register file `register value to write' inputs {\small (reg\_input\ldots)}
\item E. the instruction memory's address input {\small (pc)}
\end{itemize}
\end{frame}

\section{mov-to-register CPU}
\tikzset{controlPath/.style={}}

\providecommand{\movSplitBlock}{
    \node[logicBlock,right=.5cm of imemData,minimum height=1cm] (split) {split};
    \coordinate (splitEast1) at ([yshift=.4cm]split.east);
    \coordinate (splitEast2) at ([yshift=.2cm]split.east);
    \coordinate (splitEast3) at ([yshift=0cm]split.east);
    \coordinate (splitEast4) at ([yshift=.2cm]split.east);
    \coordinate (splitEast5) at ([yshift=.4cm]split.east);
    \coordinate (splitSouth1) at ([xshift=-.3cm]split.south);
    \coordinate (splitSouth2) at ([xshift=-.1cm]split.south);
    \coordinate (splitSouth3) at ([xshift=.1cm]split.south);
    \coordinate (splitSouth4) at ([xshift=.3cm]split.south);
    \draw[a] (imemData) -- (split);
}
\providecommand{\splitRegSelect}{
    \draw[b] (splitEast1) -- ++ (0.2cm,0cm) |- (regSelect1);
    \draw[b] (splitEast2) -- ++ (0.4cm,0cm) |- (regSelect2);
}
\providecommand{\splitRegSelectDest}{
    \draw[b] (splitEast2) -- ++ (0.4cm,0cm) |- (regSelect4);
}
\providecommand{\splitRegSelectMuxA}{
    \coordinate (regSelect4PreA) at ([xshift=-.4cm]regSelect4);
    \begin{scope}[circuit logic]
        \node[mux,inputs={nnn},minimum height=.75cm,logicFill] 
            (regSelect4MuxA) at (regSelect4PreA){};
    \end{scope}
    \draw[bN] (splitEast1) -- ++ (0.375cm,0cm);
    \draw[b] ([xshift=0.425cm]splitEast1) -- ++ (0.3cm, 0cm) |- (regSelect4MuxA.input 1);
    \draw[b] (splitEast2) -- ++ (0.4cm,0cm) |- (regSelect4MuxA.input 2);
    \draw[b] (regSelect4MuxA.output) -- (regSelect4);
    \draw[controlPath,b] (convertOp.east) -- ++(0.5cm, 0cm) |- ([yshift=-.5cm]regSelect4MuxA.south) -|
        (regSelect4MuxA.south);
}

\providecommand{\splitRegSelectMuxF}{
    \draw[bR] (regSelect4MuxA.input 3) -- ++(-.25cm, 0cm) node[left,font=\tt\tiny,inner sep=.1mm,outer sep=0mm] (noneInput) {0xF};
}
\providecommand{\movPcLogic}{
    \node[below right=2cm and -.2cm of pc,mux,inputs={nn},
          minimum height=1.2cm,
          rotate=180,logicFill] (pcSelectMux) {};
    \node[logicBlock,font=\scriptsize,right=.3cm of pcSelectMux.input 2] (pcAddTwo) {+2};
    \node[logicBlock,font=\scriptsize,right=.3cm of pcSelectMux.input 1] (pcAddTen) {+10};
    \draw[a] (pcAddTwo) -- (pcSelectMux.input 2);
    \draw[a] (pcAddTen) -- (pcSelectMux.input 1);
    \draw[b,controlPath] ([xshift=.1cm]convertOp.south west) -- ++ (0cm,-.6cm) -| (pcSelectMux.north);
    \draw[a] (pc.east) -| ([xshift=-2pt]pcAddTwo.north east);
    \draw[a] (pc.east) -| ([xshift=-2pt]pcAddTen.north east);
    \draw[a,overlay] (pcSelectMux.output) -- ++(-.3cm,0cm) |- (pc.west);
}


\begin{frame}[fragile,label=MovRegIntro]{simple ISA: mov-to-register}
    \begin{itemize}
        \item \lstinline|irmovq $constant, %rYY|
        \item \lstinline|rrmovq %rXX, %rYY|
        \item \lstinline|mrmovq 10(%rXX), %rYY|
    \end{itemize}
\end{frame}




\subsection{data memory}
\usetikzlibrary{patterns}

\begin{frame}[fragile,label=mem]{two memories}
    \begin{tikzpicture}
        \node[mem] (imem) {Instr. \\ Mem.};
        \coordinate (imemData) at (imem.east);
        \coordinate (imemAddr) at (imem.west);
        \draw[thick,-latex] (imemData) -- +(.5cm,0cm) node [font=\small,right] (i10bytes label) { data };
        \draw[thick,latex-] (imemAddr) -- +(-.3cm,0cm) node [font=\small,left] (pc label) { address };
        \onslide<2->{
            \node[mem,right=6cm of imem] (dmem) {Data \\ Mem.};
            \coordinate (dmemInHigh) at ($(dmem.west) + (0cm,-.5cm)$);
            \coordinate (dmemInLow) at ($(dmem.west) + (0cm,.5cm)$);
            \coordinate (dmemDataOut) at (dmem.east);
            \coordinate (dmemWE) at ([xshift=-.2cm]dmem.south);
            \coordinate (dmemRE) at ([xshift=.2cm]dmem.south);
            \draw[thick,-latex] (dmemDataOut) -- +(.3cm,0cm) node[right,font=\small] (mem out label) {data output};
            \draw[thick,latex-] (dmemInHigh) -- +(-.3cm,0cm) node[left,font=\small] (mem addr label) { address };
            \draw[thick,latex-] (dmemInLow) -- +(-.3cm,0cm) node[left,font=\small,align=right] (mem input label) { input \\ to write };

            \draw[thin,latex-] (dmemWE) |- +(-1cm,-1.5cm) node[left,inner sep=.05mm,font=\small] (write enable label) { write enable? };
            \draw[thin,latex-] (dmemRE) |- +(1cm,-1.5cm) node[right,inner sep=.05mm,font=\small] (read enable label) { read enable? };
        }
        \begin{scope}[shift={($(imem.south east) + (-1, -4)$)}] 
            \node[anchor=east,font=\small] at (0,-.25) {address input};
            \fill[color=red!80!black] (0.1, 0) rectangle (2.0, -.5);
            \fill[pattern=north west lines] (0.0, -.6) rectangle (0.9, -1.1);
            \fill[color=blue!80!black] (0.9, -.6) rectangle (2.0, -1.1);
            \node[anchor=east,font=\small] at (0,-.85) {data output};
            \draw[thin,-latex] (0.0, -1.3) -- (1.5, -1.3) node[right,font=\scriptsize] {time};
        \end{scope}

        \onslide<3->{
        \begin{scope}[shift={($(dmem.south east) + (-1, -3.9)$)}] 
            \draw[very thick] (0, 0) -- (1, 0) -- (1, 1) -- (2, 1) -- (2, 0) -- (2.3, 0);
            \draw[very thick,opacity=0.2] (2.3,0) -- (3,0) -- (3,1) --  (4,1);
            \draw[ultra thick,red!95!black] (1, 0) -- (1, 1);
        \end{scope}
        \begin{scope}[shift={($(dmem.south east) + (-1, -4)$)}] 
            \node[anchor=east,font=\small] at (0,-.25) {address input};
            \fill[color=orange!70!black] (0.1, 0) rectangle (1.03, -.5);
            \node[anchor=east,font=\small] at (0,-.85) {input to write};
            \fill[pattern=north west lines] (0.0, -0.6) rectangle (0.2, -1.1);
            \fill[color=green!60!white] (0.2, -.6) rectangle (1.02, -1.1);
            \node[anchor=east,font=\small] at (0,-1.45) {value in memory};
            \fill[color=blue!80!black] (0, -1.2) rectangle (0.99, -1.7);
            \fill[color=green!60!white] (1.0, -1.2) rectangle (2.3, -1.7);
        \end{scope}
        }

        \begin{visibleenv}<4>
            \tikzset{
                hcl name/.style={orange!50!black,font=\small\tt},
            }
            \node[hcl name,anchor=north] at (pc label.south) {pc};
            \node[hcl name,anchor=north] at (i10bytes label.south) {i10bytes};
            \node[hcl name,anchor=south west] at ([yshift=-2mm]mem out label.north west) {mem\_output};
            \node[hcl name,anchor=north east] at ([yshift=2mm]mem addr label.south east) {mem\_addr};
            \node[hcl name,anchor=south east] at ([yshift=-2mm]mem input label.north east) {mem\_input};
            \node[hcl name,anchor=north east] at ([yshift=-2mm]write enable label.north east) {mem\_writebit};
            \node[hcl name,anchor=north west] at ([yshift=-2mm]read enable label.north west) {mem\_readbit};
        \end{visibleenv}
    \end{tikzpicture}
\end{frame}

\begin{frame}{really two memories??}
    \begin{itemize}
    \item in Y86-64 {\small (and many real CPUs)}: \\
          writing to address $X$ in data memory: \\ changes address $X$ in instruction memory
    \vspace{.5cm}
    \item<2-> so really just one memory??
    \item<2-> we'll explain when we talk about \textit{caches}
    \end{itemize}
\end{frame}


\subsection{exercise: MUXes in mov-to-register}
\begin{frame}[fragile,label=MovRegIntro]{exercise: mov-to-register}
    \begin{itemize}
        \item \lstinline|irmovq $constant, %rYY|
        \item \lstinline|rrmovq %rXX, %rYY|
        \item \lstinline|mrmovq 10(%rXX), %rYY|
        \vspace{.5cm}
        \item for which are these are we going to need MUXes? before\ldots
            \begin{itemize}
            \item A. register file's register number (index) inputs ({\small \texttt{reg\_srcA, reg\_srcB, reg\_dstE,} \ldots})
            \item B. register file's value inputs ({\small \texttt{reg\_inputE/M}})
            \item C. PC register's input
            \item D. instruction memory's address input ({\small \texttt{pc}})
            \end{itemize}
    \end{itemize}
\end{frame}




\subsection{building mov-to-register}

\usetikzlibrary{patterns}
\tikzset{controlPath/.style={}}

\providecommand{\movSplitBlock}{
    \node[logicBlock,right=.5cm of imemData,minimum height=1cm] (split) {split};
    \coordinate (splitEast1) at ([yshift=.4cm]split.east);
    \coordinate (splitEast2) at ([yshift=.2cm]split.east);
    \coordinate (splitEast3) at ([yshift=0cm]split.east);
    \coordinate (splitEast4) at ([yshift=.2cm]split.east);
    \coordinate (splitEast5) at ([yshift=.4cm]split.east);
    \coordinate (splitSouth1) at ([xshift=-.3cm]split.south);
    \coordinate (splitSouth2) at ([xshift=-.1cm]split.south);
    \coordinate (splitSouth3) at ([xshift=.1cm]split.south);
    \coordinate (splitSouth4) at ([xshift=.3cm]split.south);
    \draw[a] (imemData) -- (split);
}
\providecommand{\splitRegSelect}{
    \draw[b] (splitEast1) -- ++ (0.2cm,0cm) |- (regSelect1);
    \draw[b] (splitEast2) -- ++ (0.4cm,0cm) |- (regSelect2);
}
\providecommand{\splitRegSelectDest}{
    \draw[b] (splitEast2) -- ++ (0.4cm,0cm) |- (regSelect4);
}
\providecommand{\splitRegSelectMuxA}{
    \coordinate (regSelect4PreA) at ([xshift=-.4cm]regSelect4);
    \begin{scope}[circuit logic]
        \node[mux,inputs={nnn},minimum height=.75cm,logicFill] 
            (regSelect4MuxA) at (regSelect4PreA){};
    \end{scope}
    \draw[bN] (splitEast1) -- ++ (0.375cm,0cm);
    \draw[b] ([xshift=0.425cm]splitEast1) -- ++ (0.3cm, 0cm) |- (regSelect4MuxA.input 1);
    \draw[b] (splitEast2) -- ++ (0.4cm,0cm) |- (regSelect4MuxA.input 2);
    \draw[b] (regSelect4MuxA.output) -- (regSelect4);
    \draw[controlPath,b] (convertOp.east) -- ++(0.5cm, 0cm) |- ([yshift=-.5cm]regSelect4MuxA.south) -|
        (regSelect4MuxA.south);
}

\providecommand{\splitRegSelectMuxF}{
    \draw[bR] (regSelect4MuxA.input 3) -- ++(-.25cm, 0cm) node[left,font=\tt\tiny,inner sep=.1mm,outer sep=0mm] (noneInput) {0xF};
}
\providecommand{\movPcLogic}{
    \node[below right=2cm and -.2cm of pc,mux,inputs={nn},
          minimum height=1.2cm,
          rotate=180,logicFill] (pcSelectMux) {};
    \node[logicBlock,font=\scriptsize,right=.3cm of pcSelectMux.input 2] (pcAddTwo) {+2};
    \node[logicBlock,font=\scriptsize,right=.3cm of pcSelectMux.input 1] (pcAddTen) {+10};
    \draw[a] (pcAddTwo) -- (pcSelectMux.input 2);
    \draw[a] (pcAddTen) -- (pcSelectMux.input 1);
    \draw[b,controlPath] ([xshift=.1cm]convertOp.south west) -- ++ (0cm,-.6cm) -| (pcSelectMux.north);
    \draw[a] (pc.east) -| ([xshift=-2pt]pcAddTwo.north east);
    \draw[a] (pc.east) -| ([xshift=-2pt]pcAddTen.north east);
    \draw[a,overlay] (pcSelectMux.output) -- ++(-.3cm,0cm) |- (pc.west);
}


\begin{frame}[fragile,label=MovRegCPU]{mov-to-register CPU}
    \begin{tikzpicture}[circuit logic]
        % FIXME: missing pre
        \instrEncodingStyles
        \tikzset{
            dmemNorm/.style={},
            dmemInputLabel/.style={},
            dmemLabel/.style={visible on=<0|handout:0>},
            dmemPre/.style={visible on=<0|handout:0>},
            dmemPreSingle/.style={visible on=<1|handout:1>},
            ccsNorm/.style={visible on=<0|handout:0>},
            isStat/.style={visible on=<0|handout:0>},
            isStatReg/.style={visible on=<0|handout:0>},
            instrRegsPre/.style={visible on=<0|handout:0>},
            instrRegsPreSingle/.style={visible on=<1|handout:1>},
            regPreSingle/.style={visible on=<1|handout:1>},
            regPre/.style={visible on=<0|handout:0>},
            %regNorm/.style={visible on=<0|handout:0>},
            imemPcPre/.style={visible on=<1|handout:1>},
            hiOver/.style={opacity=0.2,fill=green},
            controlPath/.style={alt=<0>{}},
            regNormLabelM/.style={visible on=<0|handout:0>},
        }
        \circuitState
        \dmemInput
        \circuitStatePre
        \instrEncodingStyles
        \begin{visibleenv}<2-|handout:1>
            \draw[a] (pc.east) -- (imemAddr);
            \movSplitBlock
        \end{visibleenv}
        \begin{visibleenv}<3-|handout:1>
            \splitRegSelect
        \end{visibleenv}
        \begin{visibleenv}<3-6|handout:1>
            \splitRegSelectDest
        \end{visibleenv}
        \coordinate (regReadOffset) at ([xshift=.3cm]regRead1);
        \coordinate (regReadOffset2) at ([xshift=.6cm]regRead1);
        \begin{visibleenv}<4|handout:1>
            \draw[a] (regRead1) -- (regReadOffset) |- ([xshift=-.5cm,yshift=-.5cm]regs.south west) |- (regWriteIn2);
        \end{visibleenv}
        \begin{visibleenv}<5-|handout:1>
            \node[alt=<5>{draw=red},mux,inputs={nnn},info={center:\scriptsize MUX},minimum height=1.2cm,minimum width=.8cm,below left=1.8cm and 1cm of regWriteIn1,
                  rotate=180,logicFill] (writeInMux) {};
            \draw[a] (writeInMux.output) -- ++ (-.2cm,0cm) |- (regWriteIn2);
            \node[logicBlock,alt=<5>{draw=red},below=2cm of split,font=\tiny] (convertOp) {convert\\opcode};
            \draw[b,alt=<5>{draw=red}] (splitSouth2) -- (convertOp.north -| splitSouth2);
            \draw[b,alt=<5>{draw=red}] ([xshift=-.1cm]convertOp.south east) -- ++ (0cm,-.2cm) -| (writeInMux.north);
            \draw[a] (regRead1) -- (regReadOffset) |- (writeInMux.input 3);
            \begin{visibleenv}<5>
            \node[font=\tt\fontsize{9}{10}\selectfont,draw=red,very thick,anchor=west,align=left] at ([xshift=6cm,yshift=.5cm]convertOp.west) {
                reg\_inputE = [ \\
                \hspace{.25cm} icode == RRMOVQ : \\
                \hspace{1cm}reg\_outputA; \\
                \hspace{.25cm} \ldots \\
                ]
            };
            \end{visibleenv}
        \end{visibleenv}
        \begin{visibleenv}<6-|handout:1>
            \draw[a] (split.north) -- ++(0cm,2.0cm) node[above right,font=\scriptsize] (immediateLabel) {immediate} -| (regReadOffset2) |- (writeInMux.input 2);
        \end{visibleenv}
        \begin{visibleenv}<7-|handout:1>
            \splitRegSelectMuxA
            \node[logicBlock,font=\scriptsize,left=.3cm of dmemInLow](plusAddr){+ (ALU)};
            \coordinate (plusHigh) at ([yshift=.1cm]plusAddr.west);
            \coordinate (plusLow) at ([yshift=-.1cm]plusAddr.west);
            \draw[aN] (regRead2) -- ([xshift=.15cm]regRead2) |- ([xshift=-1pt]plusHigh -| regReadOffset);
            \draw[aN] ([xshift=1pt]plusHigh -| regReadOffset) -- ([xshift=-1pt]plusHigh -| regReadOffset2);
            \draw[a] ([xshift=1pt]plusHigh -| regReadOffset2) -- (plusHigh);
            \draw[a] (plusLow -| regReadOffset2) -- (plusLow);
            \draw[a] (plusAddr.east) -- (dmemInLow);
            \draw[a] (dmemDataOut) -- ++(.2cm,0cm) |- (writeInMux.input 1);
            \begin{visibleenv}<8>
                \node[hiOver,fit=(plusAddr)] {};
                \node[font=\tt\fontsize{9}{10}\selectfont,draw=red,very thick,anchor=south west,align=left,inner sep=.5mm] at ([xshift=-3cm,yshift=.25cm]dmem.north) {
                    mem\_addr = reg\_outputB + \ldots;\\
                    mem\_readbit = 1; 
                };
            \end{visibleenv}
        \end{visibleenv}

        \begin{visibleenv}<9-|handout:1>
            \movPcLogic
        \end{visibleenv}

        \instrEncodingSubTable{below right=2.5cm and -1cm of pc}{encodings}{
            \rrmovq \rA, \rB \& |[opcode]| 2 \& |[literal]| 0 \& |[register]| \rA \& |[register]| \rB
                \& ~ \& ~ \& ~ \& ~ 
                \& ~ \& ~ \& ~ \& ~ 
                \& ~ \& ~ \& ~ \& ~
                \& ~ \& ~ \& ~ \& ~
            \\
            \irmovq \V, \rB \& |[opcode]| 3 \& |[literal]| 0 \& |[literal]| F \& |[register]| \rB 
                \& ~ \& ~ \& ~ \& ~ 
                \& ~ \& ~ \& ~ \& ~ 
                \& ~ \& ~ \& ~ \& ~
                \& ~ \& ~ \& ~ \& ~
            \\
            \mrmovq \D(\rB), \rA \& |[opcode]| 5 \& |[literal]| 0 \& |[register]| \rA \& |[register]| \rB
                \& ~ \& ~ \& ~ \& ~ 
                \& ~ \& ~ \& ~ \& ~ 
                \& ~ \& ~ \& ~ \& ~
                \& ~ \& ~ \& ~ \& ~
            \\
        };
        \node[hiOver,fit=(encodings-1-1),visible on=<1-5|handout:0>]{};
        \node[hiOver,fit=(writeInMux.input 3),visible on=<5|handout:0>]{};
        \node[hiOver,fit=(encodings-2-1),visible on=<6|handout:0>]{};
        \node[hiOver,fit=(writeInMux.input 2),visible on=<6|handout:0>]{};
        \node[hiOver,fit=(encodings-3-1),visible on=<7-8|handout:0>]{};
        \node[hiOver,fit=(writeInMux.input 1),visible on=<8|handout:0>]{};
        \node[fake,outer sep=0pt,fit=(encodings-1-6) (encodings-1-21)] (rrEmpty) {};
        \node[immediateLabel=\V,inner sep=0pt,outer sep=0pt,fit=(encodings-2-6) (encodings-2-21)] (irV) {};
        \node[immediateLabel=\D,inner sep=0pt,outer sep=0pt,fit=(encodings-3-6) (encodings-3-21)] (mrD) {};
    \end{tikzpicture}
\end{frame}



\section{mov CPU}
\usetikzlibrary{patterns,shapes.callouts,shapes.misc}

\begin{frame}[fragile,label=MovIntro]{simple ISA: mov (all cases)}
    \begin{itemize}
        \item \lstinline|irmovq $constant, %rYY|
        \item \lstinline|rrmovq %rXX, %rYY|
        \item \lstinline|mrmovq 10(%rXX), %rYY|
        \item \lstinline|rmmovq %rXX, 10(%rYY)|
    \end{itemize}
\end{frame}

\tikzset{controlPath/.style={}}

\providecommand{\movSplitBlock}{
    \node[logicBlock,right=.5cm of imemData,minimum height=1cm] (split) {split};
    \coordinate (splitEast1) at ([yshift=.4cm]split.east);
    \coordinate (splitEast2) at ([yshift=.2cm]split.east);
    \coordinate (splitEast3) at ([yshift=0cm]split.east);
    \coordinate (splitEast4) at ([yshift=.2cm]split.east);
    \coordinate (splitEast5) at ([yshift=.4cm]split.east);
    \coordinate (splitSouth1) at ([xshift=-.3cm]split.south);
    \coordinate (splitSouth2) at ([xshift=-.1cm]split.south);
    \coordinate (splitSouth3) at ([xshift=.1cm]split.south);
    \coordinate (splitSouth4) at ([xshift=.3cm]split.south);
    \draw[a] (imemData) -- (split);
}
\providecommand{\splitRegSelect}{
    \draw[b] (splitEast1) -- ++ (0.2cm,0cm) |- (regSelect1);
    \draw[b] (splitEast2) -- ++ (0.4cm,0cm) |- (regSelect2);
}
\providecommand{\splitRegSelectDest}{
    \draw[b] (splitEast2) -- ++ (0.4cm,0cm) |- (regSelect4);
}
\providecommand{\splitRegSelectMuxA}{
    \coordinate (regSelect4PreA) at ([xshift=-.4cm]regSelect4);
    \begin{scope}[circuit logic]
        \node[mux,inputs={nnn},minimum height=.75cm,logicFill] 
            (regSelect4MuxA) at (regSelect4PreA){};
    \end{scope}
    \draw[bN] (splitEast1) -- ++ (0.375cm,0cm);
    \draw[b] ([xshift=0.425cm]splitEast1) -- ++ (0.3cm, 0cm) |- (regSelect4MuxA.input 1);
    \draw[b] (splitEast2) -- ++ (0.4cm,0cm) |- (regSelect4MuxA.input 2);
    \draw[b] (regSelect4MuxA.output) -- (regSelect4);
    \draw[controlPath,b] (convertOp.east) -- ++(0.5cm, 0cm) |- ([yshift=-.5cm]regSelect4MuxA.south) -|
        (regSelect4MuxA.south);
}

\providecommand{\splitRegSelectMuxF}{
    \draw[bR] (regSelect4MuxA.input 3) -- ++(-.25cm, 0cm) node[left,font=\tt\tiny,inner sep=.1mm,outer sep=0mm] (noneInput) {0xF};
}
\providecommand{\movPcLogic}{
    \node[below right=2cm and -.2cm of pc,mux,inputs={nn},
          minimum height=1.2cm,
          rotate=180,logicFill] (pcSelectMux) {};
    \node[logicBlock,font=\scriptsize,right=.3cm of pcSelectMux.input 2] (pcAddTwo) {+2};
    \node[logicBlock,font=\scriptsize,right=.3cm of pcSelectMux.input 1] (pcAddTen) {+10};
    \draw[a] (pcAddTwo) -- (pcSelectMux.input 2);
    \draw[a] (pcAddTen) -- (pcSelectMux.input 1);
    \draw[b,controlPath] ([xshift=.1cm]convertOp.south west) -- ++ (0cm,-.6cm) -| (pcSelectMux.north);
    \draw[a] (pc.east) -| ([xshift=-2pt]pcAddTwo.north east);
    \draw[a] (pc.east) -| ([xshift=-2pt]pcAddTen.north east);
    \draw[a,overlay] (pcSelectMux.output) -- ++(-.3cm,0cm) |- (pc.west);
}


\begin{frame}<1-3|handout:1>[fragile,label=MovCPU]{mov CPU}
\vspace{-.5cm}
    \begin{tikzpicture}[circuit logic]
        \instrEncodingStyles
        \tikzset{
            dmemNorm/.style={},
            dmemInputLabel/.style={},
            dmemLabel/.style={visible on=<0|handout:0>},
            ccsNorm/.style={visible on=<0|handout:0>},
            isStat/.style={visible on=<0|handout:0>},
            isStatReg/.style={visible on=<0|handout:0>},
            %regNorm/.style={visible on=<0|handout:0>},
            hiOver/.style={opacity=0.2,fill=green},
            bookLabel/.style={color=red!60!black,font=\small\bfseries,outer sep=0pt,inner sep=1pt,fill=white},
            controlPath/.style={alt=<6|handout:3>{draw=red,very thick}{}},
            regNormLabelM/.style={visible on=<0|handout:0>},
        }
        \circuitState
        \dmemInput
        \instrEncodingStyles
        \begin{visibleenv}<1-|handout:1->
            \draw[a] (pc.east) -- (imemAddr);
            \movSplitBlock
        \end{visibleenv}
        \begin{visibleenv}<1-|handout:1>
            \splitRegSelect
        \end{visibleenv}
        \begin{visibleenv}<1-|handout:1>
            \node[mux,inputs={nnn},info={center:\scriptsize MUX},minimum height=1.2cm,minimum width=.8cm,below left=1.7cm and 1cm of regWriteIn1,
                  rotate=180,logicFill] (writeInMux) {};
            \draw[a] (writeInMux.output) -- ++ (-.2cm,0cm) |- (regWriteIn2);
            \node[logicBlock,below=1.5cm of split,font=\tiny,controlPath] (convertOp) {convert\\opcode};
            \draw[b,controlPath] (splitSouth2) -- (convertOp.north -| splitSouth2);
            \draw[b,controlPath] ([xshift=.3cm]convertOp.south) |- ([yshift=-.2cm]writeInMux.north) -- (writeInMux.north);
        \end{visibleenv}
        \begin{visibleenv}<1-|handout:1>
            \coordinate (regReadOffsetB) at ([xshift=.15cm]regRead2);
            \coordinate (regReadOffset) at ([xshift=.3cm]regRead1);
            \coordinate (regReadOffset2) at ([xshift=.6cm]regRead1);
            \draw[a] (regRead1) -- (regReadOffset) |- (writeInMux.input 3);
            \draw[a,overlay] (split.north) -- ++(0cm,2.0cm) node[above right,font=\scriptsize] (immediateLabel) {immediate} -| (regReadOffset2) |- (writeInMux.input 2);
        \end{visibleenv}
        \begin{visibleenv}<1-|handout:1>
            \draw[a,overlay] (split.north) -- ++(0cm,2cm) node[above right,font=\scriptsize] {immediate} -| (regReadOffset2) |- (writeInMux.input 2);
        \end{visibleenv}
        \begin{visibleenv}<1-|handout:0>]
            \node[logicBlock,font=\scriptsize,left=.3cm of dmemInLow](plusAddr){+};
            \coordinate (plusHigh) at ([yshift=.1cm]plusAddr.west);
            \coordinate (plusLow) at ([yshift=-.1cm]plusAddr.west);
            \draw[thick] (regRead2) -- (regReadOffsetB) |- ([xshift=-1.2pt]plusHigh -| regReadOffset);
            \draw[thick] ([xshift=1.2pt]plusHigh -| regReadOffset) -- ([xshift=-1.2pt]plusHigh -| regReadOffset2);
            \draw[a] ([xshift=1.2pt]plusHigh -| regReadOffset2) -- (plusHigh);
            \draw[a] (plusLow -| regReadOffset2) -- (plusLow);
            \draw[a] (plusAddr.east) -- (dmemInLow);
        \end{visibleenv}
        \begin{visibleenv}<1-|handout:1->
            \draw[a] (dmemDataOut) -- ++(.2cm,0cm) |- (writeInMux.input 1);
        \end{visibleenv}
        \begin{visibleenv}<1-|handout:1->
            \movPcLogic
        \end{visibleenv}
        
        \begin{visibleenv}<1-|handout:1->
            \splitRegSelectMuxA
        \end{visibleenv}
        \begin{visibleenv}<2-|handout:1->
            \splitRegSelectMuxF
        \end{visibleenv}
        \begin{visibleenv}<2|handout:0>
            \node[hiOver,fit=(noneInput)]{};
        \end{visibleenv}
        % FIXME: addition of F as destination option
        \begin{visibleenv}<3-|handout:1->
            \draw[thick] (regReadOffset |- dmemInHigh) -- ([xshift=-1.2pt]regReadOffset2 |- dmemInHigh);
            \draw[a] ([xshift=1.2pt]regReadOffset2 |- dmemInHigh) -- (dmemInHigh);
            \coordinate (dmemWE) at ([xshift=-.5cm]dmem.south);
            \draw[controlPath,b,latex-] (dmemWE) -- ++(0cm,-.5cm) node[inner sep=0pt,outer sep=1pt,below,font=\scriptsize,blue!70,
                      align=center] {write enable \\ \tiny from convert opcode};
            \node[hiOver,visible on=<3|handout:0>,fit=(dmemWE)] {};
        \end{visibleenv}

        \instrEncodingSubTable{below right=2cm and -1cm of pc}{encodings}{
            \rrmovq \rA, \rB \& |[opcode]| 2 \& |[literal]| 0 \& |[register]| \rA \& |[register]| \rB
                \& ~ \& ~ \& ~ \& ~ 
                \& ~ \& ~ \& ~ \& ~ 
                \& ~ \& ~ \& ~ \& ~
                \& ~ \& ~ \& ~ \& ~
            \\
            \irmovq \V, \rB \& |[opcode]| 3 \& |[literal]| 0 \& |[literal]| F \& |[register]| \rB 
                \& ~ \& ~ \& ~ \& ~ 
                \& ~ \& ~ \& ~ \& ~ 
                \& ~ \& ~ \& ~ \& ~
                \& ~ \& ~ \& ~ \& ~
            \\
            \mrmovq \D(\rB), \rA \& |[opcode]| 5 \& |[literal]| 0 \& |[register]| \rA \& |[register]| \rB
                \& ~ \& ~ \& ~ \& ~ 
                \& ~ \& ~ \& ~ \& ~ 
                \& ~ \& ~ \& ~ \& ~
                \& ~ \& ~ \& ~ \& ~
            \\
            \rmmovq \rA, \D(\rB) \& |[opcode]| 4 \& |[literal]| 0 \& |[register]| \rA \& |[register]| \rB
                \& ~ \& ~ \& ~ \& ~ 
                \& ~ \& ~ \& ~ \& ~ 
                \& ~ \& ~ \& ~ \& ~
                \& ~ \& ~ \& ~ \& ~
            \\
        };
        \node[fake,outer sep=0pt,fit=(encodings-1-6) (encodings-1-21)] (rrEmpty) {};
        \node[immediateLabel=\V,inner sep=0pt,outer sep=0pt,fit=(encodings-2-6) (encodings-2-21)] (irV) {};
        \node[immediateLabel=\D,inner sep=0pt,outer sep=0pt,fit=(encodings-3-6) (encodings-3-21)] (mrD) {};
        \node[immediateLabel=\D,inner sep=0pt,outer sep=0pt,fit=(encodings-4-6) (encodings-4-21)] (rmD) {};
        \node[hiOver,fit=(encodings-4-1),visible on=<1-|handout:0>] {};

        \begin{scope}[visible on=<4|handout:2>,overlay]
            \node[bookLabel,above left=1cm and .5cm of pcSelectMux,xshift=.3cm] {valP};
            \node[draw,cross out,fit=(immediateLabel),inner sep=0pt, outer sep=0pt] {};
            \node[bookLabel,right=1pt of immediateLabel] {valC};
            \node[bookLabel,above right=.5cm and .75cm of regRead2] (valBLabel) {valB};
            \draw[color=red!60!black,thick,dashed,-latex] (valBLabel) -- (regRead2);
            \node[bookLabel,above right=.5cm and .75cm of regRead1] (valALabel) {valA};
            \draw[color=red!60!black,thick,dashed,-latex] (valALabel) -- (regRead1);
            \node[bookLabel,above=.2cm of plusAddr.east] (valELabel) {valE};
            \draw[color=red!60!black,thick,dashed] (valELabel) -- ([xshift=5pt]plusAddr.east);
            \node[bookLabel,right=1cm of writeInMux.input 1] {valM};
        \end{scope}
    \end{tikzpicture}
\end{frame}




\section{aside: data paths and control paths}
\begin{frame}{data path versus control path}
\begin{itemize}
    \item data path --- signals carrying ``actual data''
    \item control path --- signals that control MUXes, etc.
        \begin{itemize}
        \item fuzzy line: e.g. are condition codes part of control path?
        \end{itemize}
    \vspace{.5cm}
    \item we will often omit parts of the control path in drawings, etc.
\end{itemize}
\end{frame}


\againframe<6>{MovCPU}

\section{the textbook's stages}

\subsection{mov stages}

\begin{frame}<1-2|handout:1>[fragile,label=MovCPUStages]{mov CPU}
\vspace{-.5cm}
    \begin{tikzpicture}[circuit logic]
        \instrEncodingStyles
        \tikzset{
            dmemNorm/.style={},
            dmemInputLabel/.style={},
            dmemLabel/.style={visible on=<0|handout:0>},
            ccsNorm/.style={visible on=<0|handout:0>},
            isStat/.style={visible on=<0|handout:0>},
            isStatReg/.style={visible on=<0|handout:0>},
            %regNorm/.style={visible on=<0|handout:0>},
            hiOver/.style={opacity=0.2,fill=green},
            bookLabel/.style={color=red!60!black,font=\small\bfseries,outer sep=0pt,inner sep=1pt,fill=white},
            regNormLabelM/.style={visible on=<0|handout:0>},
        }
        \circuitState
        \dmemInput
        \instrEncodingStyles
        \begin{visibleenv}<1-|handout:1->
            \draw[a] (pc.east) -- (imemAddr);
            \movSplitBlock
        \end{visibleenv}
        \begin{visibleenv}<1-|handout:1>
            \splitRegSelect
        \end{visibleenv}
        \begin{visibleenv}<1-|handout:1>
            \node[mux,inputs={nnn},info={center:\scriptsize MUX},minimum height=1.2cm,minimum width=.8cm,below left=1.7cm and 1cm of regWriteIn1,
                  rotate=180,logicFill] (writeInMux) {};
            \draw[a] (writeInMux.output) -- ++ (-.2cm,0cm) |- (regWriteIn2);
            \node[logicBlock,below=1.5cm of split,font=\tiny,controlPath] (convertOp) {convert\\opcode};
            \draw[b,controlPath] (splitSouth2) -- (convertOp.north -| splitSouth2);
            \draw[b,controlPath] ([xshift=.3cm]convertOp.south) |- ([yshift=-.2cm]writeInMux.north) -- (writeInMux.north);
        \end{visibleenv}
        \begin{visibleenv}<1-|handout:1>
            \coordinate (regReadOffsetB) at ([xshift=.15cm]regRead2);
            \coordinate (regReadOffset) at ([xshift=.3cm]regRead1);
            \coordinate (regReadOffset2) at ([xshift=.6cm]regRead1);
            \draw[a] (regRead1) -- (regReadOffset) |- (writeInMux.input 3);
            \draw[a,overlay] (split.north) -- ++(0cm,2.0cm) node[above right,font=\scriptsize] (immediateLabel) {immediate} -| (regReadOffset2) |- (writeInMux.input 2);
        \end{visibleenv}
        \begin{visibleenv}<1-|handout:1>
            \draw[a,overlay] (split.north) -- ++(0cm,2cm) node[above right,font=\scriptsize] {immediate} -| (regReadOffset2) |- (writeInMux.input 2);
        \end{visibleenv}
        \begin{visibleenv}<1-|handout:0>]
            \node[logicBlock,font=\scriptsize,left=.3cm of dmemInLow](plusAddr){+ (ALU)};
            \coordinate (plusHigh) at ([yshift=.1cm]plusAddr.west);
            \coordinate (plusLow) at ([yshift=-.1cm]plusAddr.west);
            \draw[thick] (regRead2) -- (regReadOffsetB) |- ([xshift=-1.2pt]plusHigh -| regReadOffset);
            \draw[thick] ([xshift=1.2pt]plusHigh -| regReadOffset) -- ([xshift=-1.2pt]plusHigh -| regReadOffset2);
            \draw[a] ([xshift=1.2pt]plusHigh -| regReadOffset2) -- (plusHigh);
            \draw[a] (plusLow -| regReadOffset2) -- (plusLow);
            \draw[a] (plusAddr.east) -- (dmemInLow);
        \end{visibleenv}
        \begin{visibleenv}<1-|handout:1->
            \draw[a] (dmemDataOut) -- ++(.2cm,0cm) |- (writeInMux.input 1);
        \end{visibleenv}
        \begin{visibleenv}<1-|handout:1->
            \movPcLogic
        \end{visibleenv}
        
        \begin{visibleenv}<1-|handout:1->
            \splitRegSelectMuxA
        \end{visibleenv}
        \begin{visibleenv}<1-|handout:1->
            \splitRegSelectMuxF
        \end{visibleenv}
        \begin{visibleenv}<1-|handout:1->
            \draw[thick] (regReadOffset |- dmemInHigh) -- ([xshift=-1.2pt]regReadOffset2 |- dmemInHigh);
            \draw[a] ([xshift=1.2pt]regReadOffset2 |- dmemInHigh) -- (dmemInHigh);
            \coordinate (dmemWE) at ([xshift=-.5cm]dmem.south);
            \draw[b,latex-] (dmemWE) -- ++(0cm,-.5cm) node[inner sep=0pt,outer sep=1pt,below,font=\scriptsize,blue!70,
                      align=center] {write enable \\ \tiny from convert opcode};
        \end{visibleenv}

        \tikzset{
            stageLabel/.style={font=\bfseries,fill=white,fill opacity=0.7},stageLine/.style={blue!40!black,line width=2pt},
        }
        \begin{visibleenv}<2-|handout:1->
        \draw[stageLine,red!60!black] ([xshift=-.25cm,yshift=.25cm]pcSelectMux.south east) -| ([xshift=-.25cm,yshift=.25cm]imem.north west)
              -| ([xshift=.25cm,yshift=-.5cm]split.south east) -| ([xshift=.25cm,yshift=-2cm]imem.south east)
              |- ([xshift=-1cm,yshift=-.25cm]pcSelectMux.north west) |- cycle;
        \node[stageLabel,text=red!60!black] at ([xshift=2.5cm]pcSelectMux) { fetch };
        \draw[stageLine,orange!60!black] ([xshift=1cm,yshift=.25cm]split.north east |- regSelect1) rectangle ([yshift=-.2cm,xshift=2cm]regSelect2);
        \node[above=.25cm of regSelect1,stageLabel,draw=none,text=orange!60!black] { decode };
        \node[draw,stageLine,fit=(plusAddr),green!60!black] {};
        \node[above=.25cm of plusAddr,stageLabel,xshift=-1cm,text=green!60!black] { execute};
              
        \node[draw,stageLine,fit=(dmem),blue!60!black] {};
        \node[above=.25cm of dmem,stageLabel,text=blue!60!black] { memory};
        \coordinate (writeInMuxPost) at ([xshift=-.5cm]writeInMux.east);
        \coordinate (regSelect4Post) at ([xshift=1cm]regSelect4);
        \node[draw,stageLine,violet!60!black,inner sep=.1mm,fit=(writeInMux) (writeInMuxPost) (regSelect4) (regSelect4MuxA) (regSelect4Post)] {};
        \node[below=1cm of writeInMux,stageLabel,text=violet!60!black] { writeback};
        \draw[stageLine,magenta!60!black] ([xshift=-1cm,yshift=.25cm]pc.west) rectangle ([yshift=-.25cm]pc.west);
        \node[above=1cm of pc.west,stageLabel=text=magenta!60!black] { PC update };
        \end{visibleenv}
    \end{tikzpicture}
\end{frame}


\subsection{describing stages}


\begin{frame}{Stages}
    \begin{itemize}
        \item conceptual division of instruction:
        \vspace{.5cm}
        \item \myemph{fetch} --- read instruction memory, split instruction, compute length
        \item \myemph{decode} --- read register file
        \item \myemph{execute} --- arithmetic (including of addresses)
        \item \myemph{memory} --- read or write data memory
        \item \myemph{write back} --- write to register file
        \item \myemph{PC update} --- compute next value of PC
    \end{itemize}
\end{frame}


\subsection{exercise: stages versus time}

\begin{frame}[fragile,label=StagesAndTime]{stages and time}
    \begin{itemize}
        \item {\small fetch / decode / execute / memory / write back / PC update}
        \item \myemph{Order} when these events happen \lstinline|pushq %rax| instruction:
            \begin{enumerate}
                \item instruction read
                \item memory changes
                \item \lstinline|%rsp| changes
                \item PC changes
            \end{enumerate}
        \item Hint: recall how registers, register files, memory works
        \vspace{-.1cm}
        \item\begin{tabular}{ll}
            \textbf{a.} & 1; then 2, 3, and 4 in any order \\
            \textbf{b.} & 1; then 2, 3, and 4 at almost the same time \\
            \textbf{c.} & 1; then 2; then 3; then 4 \\
            \textbf{d.} & 1; then 3; then 2; then 4 \\
            \textbf{e.} & 1; then 2; then 3 and 4 at almost the same time \\
            \textbf{f.} & something else \\
        \end{tabular}
    \end{itemize} 
\end{frame}



\subsection{walking through the stages}

\usetikzlibrary{shapes.misc,shapes.callouts}

\begin{frame}[fragile,label=SEQIFetch]{SEQ: instruction fetch}
\begin{itemize}
\item read instruction memory at PC
\item split into seperate wires:
\begin{itemize}
    \item \myemph{icode:ifun} --- opcode
    \item \myemph{rA}, \myemph{rB} --- register numbers
    \item \myemph{valC} --- call target or mov displacement % FIXME: MUX based on instruction mark
\end{itemize}
\item compute next instruction address:
\begin{itemize}
    \item \myemph{valP} --- PC + (instr length)
\end{itemize}
\end{itemize}
\end{frame}

\begin{frame}[fragile,label=ifetchCPU]{instruction fetch}
\begin{tikzpicture}[circuit logic]
    \tikzset{
        dmemLabel/.style={visible on=<0|handout:0>},
        dmemInputLabel/.style={visible on=<1|handout:1>},
        instrRegs/.style={visible on=<0|handout:0>},
        regsLogic/.style={visible on=<0|handout:0>},
        logicDmem/.style={visible on=<0|handout:0>},
        dmemPC/.style={visible on=<1|handout:1>},
        dmemPCMux/.style={visible on=<0|handout:0>},
        dmemPCNoMux/.style={visible on=<0|handout:0>},
        dmemWB/.style={visible on=<0|handout:0>},
        bookLabel/.style={color=red!60!black,font=\small\bfseries,outer sep=0pt,inner sep=1pt,fill=white},
        instrRegs/.style={visible on=<1-|handout:1>},
        instrRegsSplitOut/.style={visible on=<1-|handout:1>},
        instrRegsRS1/.style={visible on=<0|handout:0>},
        instrRegsMux/.style={visible on=<0|handout:0>},
    }
    \circuitState
    \circuitConnectDetail
    \draw[a] (iLenPlus) -- ++ (-1cm, 0cm) node[left,bookLabel] {valP};
\end{tikzpicture}
\end{frame}

\begin{frame}[fragile,label=SEQIDecode]{SEQ: instruction ``decode''}
\begin{itemize}
\item read registers
\begin{itemize}
    \item \myemph{valA}, \myemph{valB} --- register values
\end{itemize}
\end{itemize}
\end{frame}

\tikzset{overlayQuestion/.style={overlay,anchor=south,at={([yshift=.5cm]current page.south)},align=left,fill=white}}

\begin{frame}[fragile,label=idecodeCPUBroken]{instruction decode (1)}
\begin{tikzpicture}[circuit logic]
    \tikzset{
        dmemLabel/.style={visible on=<0|handout:0>},
        dmemInputLabel/.style={visible on=<1-|handout:1>},
        instrRegs/.style={visible on=<0|handout:0>},
        regsLogic/.style={visible on=<0|handout:0>},
        logicDmem/.style={visible on=<0|handout:0>},
        dmemPC/.style={visible on=<1-|handout:1>},
        dmemPCMux/.style={visible on=<0|handout:0>},
        dmemPCNoMux/.style={visible on=<0|handout:0>},
        dmemWB/.style={visible on=<0|handout:0>},
        bookLabel/.style={color=red!60!black,font=\small\bfseries,outer sep=0pt,inner sep=1pt,fill=white},
        instrRegs/.style={visible on=<1-|handout:1>},
        instrRegsSplitOut/.style={visible on=<1-|handout:1>},
        instrRegsRS1/.style={visible on=<1-|handout:1>},
        instrRegsMux/.style={visible on=<0|handout:0>},
        instrRegsNoMuxRS2/.style={visible on=<1-|handout:1>},
    }
    \circuitState
    \circuitConnectDetail
    \draw[a] (iLenPlus) -- ++ (-1cm, 0cm) node[left,bookLabel] {valP};
\end{tikzpicture}
\begin{tikzpicture}[remember picture]
    \begin{visibleenv}<2>
        \node[overlayQuestion] {
            exercise: for which instructions would there be a problem ? \\
            {\tt nop}, {\tt addq}, {\tt mrmovq}, {\tt rmmovq}, {\tt jmp}, {\tt pushq} \iftoggle{heldback}{}{\only<3->{\color{red} of these: only pushq} }
        };
    \end{visibleenv}
\end{tikzpicture}
\end{frame}

\begin{frame}[fragile,label=SEQIDecodeSrc]{SEQ: srcA, srcB}
\begin{itemize}
\item always read rA, rB?
\item Problems: 
\begin{itemize}
    \item push rA
    \item pop
    \item call
    \item ret
\end{itemize}
\item book: extra signals: srcA, srcB --- computed input register
\item MUX controlled by icode
\end{itemize}
\end{frame}

\newcommand{\pushq}{{\keywordstyle pushq}}
\newcommand{\popq}{{\keywordstyle popq}}


\begin{frame}[fragile,label=SEQIDecodeSrcRegs]{SEQ: possible registers to read}
\small
\begin{tabular}{l|ll}
instruction & srcA & srcB \\ \hline
\halt, \nop, {\keywordstyle j{\it CC}}, \irmovq & none & none \\
{\keywordstyle cmovCC}, {\keywordstyle rrmovq} & rA & none \\
\mrmovq & none & rB \\
\rmmovq, {\keywordstyle {\it OP}q} & rA & rB \\
{\keywordstyle call}, {\keywordstyle ret} & none? & \myemph<2>{\tt \%rsp} \\
\pushq, \popq & rA & \myemph<2>{\tt \%rsp} \\
\end{tabular}
\begin{tikzpicture}[circuit logic]
    \instrEncodingStyles
    \node[draw,mux,inputs={nnn},info={center:MUX},minimum height=3cm,minimum width=1.5cm] (rBMux) {};
    \draw[thick,-latex] (rBMux.output) -- +(0:5mm) node[right,font=\small] {srcB};
    \draw[thick,latex-] (rBMux.input 1) -- +(180:5mm) node[left] {\rnify{rB}};
    \draw[thick,latex-] (rBMux.input 2) -- +(180:5mm) node[left] {\rnifyWide{\%rsp}};
    \draw[thick,latex-] (rBMux.input 3) -- +(180:5mm) node[left] (noneLabel) {(none) \literalify{F}};
    \node[logicBlock,below=.5cm of rBMux.select] (logic) {logic function};
    \draw[b] (logic) -- (rBMux.select);
    \draw[b,latex-] (logic.west) -- +(180:5mm) node[left] {\icode};
    \onslide<2->{
        \node[draw,red,cross out,fit=(noneLabel)] {};
    }
\end{tikzpicture}
\end{frame}

\begin{frame}[fragile,label=idecodeCPUWorking]{instruction decode (2)}
\begin{tikzpicture}[circuit logic]
    \tikzset{
        dmemLabel/.style={visible on=<0|handout:0>},
        dmemInputLabel/.style={visible on=<1|handout:1>},
        instrRegs/.style={visible on=<0|handout:0>},
        regsLogic/.style={visible on=<0|handout:0>},
        logicDmem/.style={visible on=<0|handout:0>},
        dmemPC/.style={visible on=<1|handout:1>},
        dmemPCMux/.style={visible on=<0|handout:0>},
        dmemPCNoMux/.style={visible on=<0|handout:0>},
        dmemWB/.style={visible on=<0|handout:0>},
        bookLabel/.style={color=red!60!black,font=\small\bfseries,outer sep=0pt,inner sep=1pt,fill=white},
        instrRegs/.style={visible on=<1|handout:1>},
        instrRegsRS1/.style={visible on=<1|handout:1>},
        instrRegsMux/.style={visible on=<1|handout:1>},
        instrRegsMuxRS3/.style={visible on=<0|handout:0>},
        instrRegsMuxRS4/.style={visible on=<0|handout:0>},
    }
    \circuitState
    \circuitConnectDetail
    \draw[a] (iLenPlus) -- ++ (-1cm, 0cm) node[left,bookLabel] {valP};
\end{tikzpicture}
\end{frame}

\begin{frame}[fragile,label=SEQExecute]{SEQ: execute}
\begin{itemize}
\item perform ALU operation (add, sub, xor, and)
\begin{itemize}
    \item \myemph{valE} --- ALU output
\end{itemize}
\item read prior condition codes
\begin{itemize}
    \item \myemph{Cnd} --- condition codes based on ifun (instruction type for jCC/cmovCC)
\end{itemize}
\item write new condition codes
\end{itemize}
\end{frame}

\begin{frame}[fragile,label=UsingCCodes]{using condition codes: cmov\*}
\begin{tikzpicture}[circuit logic US]
\instrEncodingStyles
\node[draw,mux,inputs={nnnnnnn},minimum height=5cm] (condMux) {};
\begin{scope}[thick,latex-]
\draw (condMux.input 1) -- ++(-.5cm, 0cm) node[left,font=\small] {(\textit{always}) {\tt 1}};
\draw (condMux.input 2) -- ++(-.5cm, 0cm) node[left,font=\small] {(\textit{le}) {\tt SF | ZF}};
\draw (condMux.input 3) -- ++(-.5cm, 0cm) node[left,font=\small] {(\textit{l}) {\tt SF}};
\draw (condMux.input 4) -- ++(-.5cm, 0cm);
\draw (condMux.input 5) -- ++(-.5cm, 0cm);
\draw (condMux.input 6) -- ++(-.5cm, 0cm);
\draw (condMux.input 7) -- ++(-.5cm, 0cm);
\end{scope}
\node[below=.5cm of condMux.select,font=\small,align=center] (cc) {\ccify{cc}~\\(from instr)};
\draw[thick,-latex] (cc) -- (condMux.select);
\node[draw,mux,minimum height=2cm,inputs={nnn},right=5cm of condMux] (dstMuxE) {};
\begin{scope}[thick,latex-]
\draw (dstMuxE.input 1) -- ++(-.5cm, 0cm) node[left,font=\small] {\vrB};
\draw (dstMuxE.input 2) -- ++(-.5cm, 0cm) node[left,font=\small] {\tt 0xF};
\end{scope}
\begin{scope}[thick,-latex]
\draw (dstMuxE.output) -- ++(.5cm, 0cm) node[right,font=\small] {dstE};
\end{scope}
\node [not gate,minimum width=1.2cm] (notGate) at ([xshift=1cm]condMux.output) {\scriptsize NOT};
\draw[dashed,thick,-latex] (condMux.output) -- (notGate.input);
\draw[dashed,thick,-latex] (notGate.output) -- ++(.25cm, 0cm) |- ([xshift=-.25cm,yshift=-.25cm]dstMuxE.south west)
        -| (dstMuxE.select);
\end{tikzpicture}
\end{frame}

\begin{frame}[fragile,label=executeCPUBroken]{execute (1)}
\begin{tikzpicture}[circuit logic]
    \tikzset{
        dmemLabel/.style={visible on=<0|handout:0>},
        dmemInputLabel/.style={visible on=<1-|handout:1>},
        instrRegs/.style={visible on=<0|handout:0>},
        regsLogic/.style={visible on=<1-|handout:1>},
        regsLogicNoMux/.style={visible on=<1-|handout:1>},
        regsLogicMux/.style={visible on=<0|handout:0>},
        logicDmem/.style={visible on=<0|handout:0>},
        dmemPC/.style={visible on=<1-|handout:1>},
        dmemPCMux/.style={visible on=<0|handout:0>},
        dmemPCNoMux/.style={visible on=<0|handout:0>},
        dmemWB/.style={visible on=<0|handout:0>},
        instrRegs/.style={visible on=<1-|handout:1>},
        instrRegsSplitImmed/.style={visible on=<1-|handout:1>},
        instrRegsRS1/.style={visible on=<1-|handout:1>},
        instrRegsMux/.style={visible on=<1-|handout:1>},
        instrRegsMuxRS3/.style={visible on=<0|handout:0>},
        instrRegsMuxRS4/.style={visible on=<0|handout:0>},
    }
    \circuitState
    \circuitConnectDetail
    \draw[a] (iLenPlus) -- ++ (-1cm, 0cm) node[left,bookLabel] {valP};
\end{tikzpicture}
\begin{tikzpicture}[remember picture]
    \begin{visibleenv}<2>
        \begin{scope}[overlay]
            \node[overlayQuestion] {
                exercise: which of these instructions would there be a problem ? \\
                {\tt nop}, {\tt addq}, {\tt mrmovq}, {\tt popq}, {\tt call}, 
            };
        \end{scope}
    \end{visibleenv}
\end{tikzpicture}
\end{frame}

\begin{frame}[fragile,label=SEQExecuteALU]{SEQ: ALU operations?}
\begin{itemize}
% FIXME: ALU picture
\item ALU inputs always \myemph{valA}, \myemph{valB} (register values)?
\item no, inputs from instruction: (Displacement + rB)
\begin{tikzpicture}[overlay,circuit logic,global scale=0.5,yshift=-1.2cm]
    \node[draw,mux,inputs={nn},info={center:MUX},minimum height=2cm,minimum width=1cm] (BMux) {};
    \draw[thick,-latex] (BMux.output) -- +(0:5mm) node[right,font=\small] {aluA};
    \draw[thick,latex-] (BMux.input 1) -- +(180:5mm) node[left] {valA};
    \draw[thick,latex-] (BMux.input 2) -- +(180:5mm) node[left] {valC};
\end{tikzpicture}
\begin{itemize}
    \item \mrmovq
    \item \rmmovq
\end{itemize}
\item no, constants: (rsp +/- 8)
\begin{itemize}
    \item \pushq
    \item \popq
    \item {\keywordstyle call}
    \item {\keywordstyle ret}
\end{itemize}
\item extra signals: \myemph{aluA}, \myemph{aluB}
\begin{itemize}
    \item computed ALU input values
\end{itemize}
\end{itemize}
\end{frame}

\begin{frame}[fragile,label=executeCPUWorkingNoBMux]{execute (2)}
    \begin{tikzpicture}[circuit logic]
    \tikzset{
        dmemLabel/.style={visible on=<0|handout:0>},
        dmemInputLabel/.style={visible on=<1-|handout:1>},
        instrRegs/.style={visible on=<0|handout:0>},
        regsLogic/.style={visible on=<1-|handout:1>},
        regsLogicMux/.style={visible on=<0|handout:0>},
        regsLogicMuxA/.style={visible on=<1-|handout:1>},
        regsLogicNoMuxB/.style={visible on=<1-|handout:1>},
        logicDmem/.style={visible on=<0|handout:0>},
        dmemPC/.style={visible on=<1-|handout:1>},
        dmemPCMux/.style={visible on=<0|handout:0>},
        dmemPCNoMux/.style={visible on=<0|handout:0>},
        dmemWB/.style={visible on=<0|handout:0>},
        instrRegs/.style={visible on=<1-|handout:1>},
        %instrRegsSplitImmed/.style={visible on=<1-|handout:1>},
        instrRegsRS1/.style={visible on=<1-|handout:1>},
        instrRegsMux/.style={visible on=<1-|handout:1>},
        instrRegsMuxRS3/.style={visible on=<0|handout:0>},
        instrRegsMuxRS4/.style={visible on=<0|handout:0>},
    }
    \circuitState
    \circuitConnectDetail
    \draw[a] (iLenPlus) -- ++ (-1cm, 0cm) node[left,bookLabel] {valP};
    \end{tikzpicture}
\end{frame}

% FIXME: computing ALU input values

\begin{frame}[fragile,label=SEQMemory]{SEQ: Memory}
\begin{itemize}
\item read or write data memory
\begin{itemize}
    \item \myemph{valM} --- value read from memory (if any)
\end{itemize}
\end{itemize}
\end{frame}

\begin{frame}[fragile,label=memoryNoMux]{memory (1)}
    \begin{tikzpicture}[circuit logic]
    \tikzset{
        dmemLabel/.style={visible on=<0|handout:0>},
        dmemInputLabel/.style={visible on=<1-|handout:1>},
        instrRegs/.style={visible on=<0|handout:0>},
        regsLogic/.style={visible on=<1-|handout:1>},
        regsLogicMux/.style={visible on=<0|handout:0>},
        regsLogicMuxA/.style={visible on=<1-|handout:1>},
        %regsLogicMuxB/.style={visible on=<1-|handout:1>},
        regsLogicNoMuxB/.style={visible on=<1-|handout:1>},
        logicDmem/.style={visible on=<1-|handout:1>},
        logicDmemMux/.style={visible on=<0|handout:0>},
        logicDmemNoMux/.style={visible on=<1-|handout:1>},
        dmemPC/.style={visible on=<1-|handout:1>},
        dmemPCMux/.style={visible on=<0|handout:0>},
        dmemPCNoMux/.style={visible on=<0|handout:0>},
        dmemWB/.style={visible on=<0|handout:0>},
        instrRegs/.style={visible on=<1-|handout:1>},
        %instrRegsSplitImmed/.style={visible on=<1-|handout:1>},
        instrRegsRS1/.style={visible on=<1-|handout:1>},
        instrRegsMux/.style={visible on=<1-|handout:1>},
        instrRegsMuxRS3/.style={visible on=<0|handout:0>},
        instrRegsMuxRS4/.style={visible on=<0|handout:0>},
    }
    \circuitState
    \circuitConnectDetail
    \draw[a] (iLenPlus) -- ++ (-1cm, 0cm) node[left,bookLabel] {valP};
\end{tikzpicture}
\begin{tikzpicture}[remember picture]
    \begin{visibleenv}<2>
        \node[overlayQuestion] {
            exercise: which of these instructions would there be a problem ? \\
            {\tt nop}, {\tt rmmovq}, {\tt mrmovq}, {\tt popq}, {\tt call}, 
        };
    \end{visibleenv}
\end{tikzpicture}
\end{frame}

\begin{frame}[fragile,label=SEQMemoryControl]{SEQ: control signals for memory}
\begin{itemize}
\item read/write --- \myemph{read enable}? \myemph{write enable}?
\item \myemph{Addr} --- address 
    \begin{itemize}
        \item mostly ALU output
        \item special cases (need extra MUX): \popq, {\keywordstyle ret}
    \end{itemize}
\item \myemph{Data} --- value to write
    \begin{itemize}
        \item mostly valA
        \item special cases (need extra MUX): {\keywordstyle call}
    \end{itemize}
\end{itemize}
\end{frame}

% FIXME: Addr == ALUOutput?
% FIXME: Data == valA?

\begin{frame}[fragile,label=memoryMux]{memory (2)}
    \begin{tikzpicture}[circuit logic]
    \tikzset{
        dmemLabel/.style={visible on=<0|handout:0>},
        dmemInputLabel/.style={visible on=<1-|handout:1>},
        instrRegs/.style={visible on=<0|handout:0>},
        regsLogic/.style={visible on=<1-|handout:1>},
        regsLogicMux/.style={visible on=<0|handout:0>},
        regsLogicMuxA/.style={visible on=<1-|handout:1>},
        %regsLogicMuxB/.style={visible on=<1-|handout:1>},
        regsLogicNoMuxB/.style={visible on=<1-|handout:1>},
        logicDmem/.style={visible on=<1-|handout:1>},
        dmemPC/.style={visible on=<1-|handout:1>},
        dmemPCMux/.style={visible on=<0|handout:0>},
        dmemPCNoMux/.style={visible on=<0|handout:0>},
        dmemWB/.style={visible on=<0|handout:0>},
        instrRegs/.style={visible on=<1-|handout:1>},
        %instrRegsSplitImmed/.style={visible on=<1-|handout:1>},
        instrRegsRS1/.style={visible on=<1-|handout:1>},
        instrRegsMux/.style={visible on=<1-|handout:1>},
        instrRegsMuxRS3/.style={visible on=<0|handout:0>},
        instrRegsMuxRS4/.style={visible on=<0|handout:0>},
    }
    \circuitState
    \circuitConnectDetail
    \draw[a] (iLenPlus) -- ++ (-1cm, 0cm) node[left,bookLabel] {valP};
    \end{tikzpicture}
\end{frame}

\begin{frame}[fragile,label=SEQWriteBack]{SEQ: write back}
\begin{itemize}
\item write registers
\end{itemize}
\end{frame}

\begin{frame}[fragile,label=writeBackNoMux]{write back (1)}
    \begin{tikzpicture}[circuit logic]
    \tikzset{
        dmemLabel/.style={visible on=<0|handout:0>},
        dmemInputLabel/.style={visible on=<1-|handout:1>},
        instrRegs/.style={visible on=<0|handout:0>},
        regsLogic/.style={visible on=<1-|handout:1>},
        regsLogicMux/.style={visible on=<0|handout:0>},
        regsLogicMuxA/.style={visible on=<1-|handout:1>},
        logicDmem/.style={visible on=<1-|handout:1>},
        dmemPC/.style={visible on=<1-|handout:1>},
        dmemPCMux/.style={visible on=<0|handout:0>},
        dmemPCNoMux/.style={visible on=<0|handout:0>},
        dmemWB/.style={visible on=<1-|handout:1->},
        dmemOutToPC/.style={visible on=<0|handout:0>},
        instrRegs/.style={visible on=<1-|handout:1>},
        %instrRegsSplitImmed/.style={visible on=<1-|handout:1>},
        instrRegsRS1/.style={visible on=<1-|handout:1>},
        instrRegsMux/.style={visible on=<1-|handout:1>},
        instrRegsNoMuxRS3/.style={visible on=<1-|handout:1>},
        instrRegsNoMuxRS4/.style={visible on=<1-|handout:1>},
        instrRegsMuxRS3/.style={visible on=<0|handout:0>},
        instrRegsMuxRS4/.style={visible on=<0|handout:0>},
        regsLogicNoMuxB/.style={visible on=<1-|handout:1>},
    }
    \circuitState
    \circuitConnectDetail
    \draw[a] (iLenPlus) -- ++ (-1cm, 0cm) node[left,bookLabel] {valP};
    \begin{visibleenv}<2>
        \node[overlay,my callout2=regSelect3,align=left,anchor=south east,font=\small] at ([yshift=1cm,xshift=4cm]regSelect3) {
            textbook convention: \\
            E used for storing ALU results (e.g. add) \\
            M used for storing memory results (e.g. rmmovq) \\
            (you don't have to do this\ldots)
        };
    \end{visibleenv}
    \end{tikzpicture}
    \begin{tikzpicture}[remember picture]
    \begin{visibleenv}<3>
            \node[overlayQuestion] {
                exercise: which of these instructions would there be a problem ? \\
                {\tt nop}, {\tt irmovq}, {\tt mrmovq}, {\tt rmmovq}, {\tt addq}, {\tt popq}
            };
    \end{visibleenv}
    \end{tikzpicture}
\end{frame}


\begin{frame}[fragile,label=SEQWriteBackControl]{SEQ: control signals for WB}
\begin{itemize}
\item \myemph{two} write inputs --- two needed by \lstinline|popq|
\begin{itemize}
\item valM (memory output), valE (ALU output)
\end{itemize}
\item \myemph{two} register numbers
\begin{itemize}
\item dstM, dstE
\end{itemize}
\item write disable --- use dummy register number {\tt 0xF}
\end{itemize}
\begin{tikzpicture}[circuit logic]
    \instrEncodingStyles
    \node[draw,mux,inputs={nnn},info={center:MUX},minimum height=2cm,minimum width=1.2cm] (Mux) {};
    \draw[thick,-latex] (Mux.output) -- +(0:5mm) node[right,font=\small] {dstE};
    \draw[thick,latex-] (Mux.input 1) -- +(180:5mm) node[left] {\rnify{rB}};
    \draw[thick,latex-] (Mux.input 2) -- +(180:5mm) node[left] {\literalify{F}};
    \draw[thick,latex-] (Mux.input 3) -- +(180:5mm) node[left] {\rnifyWide{\%rsp}};
\end{tikzpicture}
\end{frame}

\begin{frame}[fragile,label=writeBackMuxA]{write back (2a)}
    \begin{tikzpicture}[circuit logic]
    \tikzset{
        dmemLabel/.style={visible on=<0|handout:0>},
        dmemInputLabel/.style={visible on=<1-|handout:1>},
        instrRegs/.style={visible on=<0|handout:0>},
        regsLogic/.style={visible on=<1-|handout:1>},
        regsLogicMux/.style={visible on=<0|handout:0>},
        regsLogicMuxA/.style={visible on=<1-|handout:1>},
        regsLogicNoMuxB/.style={visible on=<1-|handout:1>},
        logicDmem/.style={visible on=<1-|handout:1>},
        dmemPC/.style={visible on=<1-|handout:1>},
        dmemPCMux/.style={visible on=<0|handout:0>},
        dmemPCNoMux/.style={visible on=<0|handout:0>},
        dmemWB/.style={visible on=<1-|handout:1->},
        dmemWBvalEMux/.style={visible on=<1-|handout:1->,red},
        dmemWBvalENoMux/.style={visible on=<0|handout:0>},
        dmemOutToPC/.style={visible on=<0|handout:0>},
        instrRegs/.style={visible on=<1-|handout:1>},
        %instrRegsSplitImmed/.style={visible on=<1-|handout:1>},
        instrRegsRS1/.style={visible on=<1-|handout:1>},
        instrRegsMux/.style={visible on=<1-|handout:1>},
    }
    \circuitState
    \circuitConnectDetail
    \draw[a] (iLenPlus) -- ++ (-1cm, 0cm) node[left,bookLabel] {valP};
    \end{tikzpicture}
\end{frame}

\begin{frame}[fragile,label=writeBackMuxB]{write back (2b)}
    \begin{tikzpicture}[circuit logic]
    \tikzset{
        dmemLabel/.style={visible on=<0|handout:0>},
        dmemInputLabel/.style={visible on=<1-|handout:1>},
        instrRegs/.style={visible on=<0|handout:0>},
        regsLogic/.style={visible on=<1-|handout:1>},
        regsLogicMux/.style={visible on=<1-|handout:1>},
        regsLogicMuxB/.style={red},
        logicDmem/.style={visible on=<1-|handout:1>},
        dmemPC/.style={visible on=<1-|handout:1>},
        dmemPCMux/.style={visible on=<0|handout:0>},
        dmemPCNoMux/.style={visible on=<0|handout:0>},
        dmemWB/.style={visible on=<1-|handout:1->},
        dmemOutToPC/.style={visible on=<0|handout:0>},
        instrRegs/.style={visible on=<1-|handout:1>},
        %instrRegsSplitImmed/.style={visible on=<1-|handout:1>},
        instrRegsRS1/.style={visible on=<1-|handout:1>},
        instrRegsMux/.style={visible on=<1-|handout:1>},
    }
    \circuitState
    \circuitConnectDetail
    \draw[a] (iLenPlus) -- ++ (-1cm, 0cm) node[left,bookLabel] {valP};
    \end{tikzpicture}
\end{frame}


\begin{frame}[fragile,label=SEQUpdatePC]{SEQ: Update PC}
\begin{itemize}
\item choose value for PC next cycle (input to PC register)
\begin{itemize}
\item usually valP (following instruction)
\item exceptions: {\keywordstyle call}, {\keywordstyle j{\it CC}}, {\keywordstyle ret}
\end{itemize}
\end{itemize}
\end{frame}

\begin{frame}[fragile,label=pcUpdateCircuit]{PC update}
    \begin{tikzpicture}[circuit logic]
    \tikzset{
        dmemLabel/.style={visible on=<0|handout:0>},
        dmemInputLabel/.style={visible on=<1-|handout:1>},
        instrRegs/.style={visible on=<0|handout:0>},
        regsLogic/.style={visible on=<1-|handout:1>},
        regsLogicMux/.style={visible on=<0|handout:0>},
        regsLogicMuxA/.style={visible on=<1-|handout:1>},
        regsLogicMuxB/.style={visible on=<1-|handout:1>},
        logicDmem/.style={visible on=<1-|handout:1>},
        dmemPC/.style={visible on=<1-|handout:1>},
        dmemWB/.style={visible on=<1-|handout:1->},
        dmemOutToPC/.style={visible on=<1-|handout:1->},
        instrRegs/.style={visible on=<1-|handout:1>},
        %instrRegsSplitImmed/.style={visible on=<1-|handout:1>},
        instrRegsRS1/.style={visible on=<1-|handout:1>},
        instrRegsMux/.style={visible on=<1-|handout:1>},
    }
    \circuitState
    \circuitConnectDetail
    \draw[a] (iLenPlus) -- ++ (-1cm, 0cm) node[left,bookLabel] {valP};
    \end{tikzpicture}
\end{frame}



\subsection{exercises: setting muxes}

\begin{frame}<1-3|handout:1>[fragile,label=SettingMuxes]{circuit: setting MUXes}
\begin{tikzpicture}[circuit logic]
    \tikzset{
        dmemLabel/.style={visible on=<0|handout:0>},
        isStatReg/.style={draw=none},
        isStat/.style={draw=none},
        ccsNorm/.style={visible on=<0|handout:0>},
        hiBox/.style={fill=green,opacity=0.3},
        overText/.style={red,fill=white,fill opacity=0.7},
    }
    \circuitState
    \circuitConnectDetail
    \begin{visibleenv}<2-4>
    \begin{scope}[overlay]
        \node[overText,right=3pt of vrALabel,inner sep=0pt, outer sep=0pt, font=\tiny]{\tt 8}; 
        \node[overText,right=3pt of vrBLabel,inner sep=0pt, outer sep=0pt,font=\tiny]{\tt 9};
        \node[overText,below left=1cm and -.5cm of muxPc,font=\scriptsize,fill=white] {PC+2};
        \node[overText,above left=0.7cm and -1cm of muxPc,font=\scriptsize] {M[PC+1]};
        \node[overText,left=1pt of muxDstM.input 1,inner sep=0pt, outer sep=0pt,font=\tiny,yshift=.4ex]{rA=\tt 8};
        \node[overText,left=1pt of muxDstE.input 1,inner sep=0pt, outer sep=0pt,font=\tiny,yshift=.4ex]{rB=\tt 9};
        \node[overText,above right=0pt of regRead1,inner sep=0pt, outer sep=0pt,font=\tiny]{\tt R[8]};
        \node[overText,left=0.0cm of muxAluB.input 1,inner sep=0pt, outer sep=0pt,font=\tiny,yshift=.3ex]{\tt R[9]};
        \coordinate (eLabelPt) at ([yshift=-.25cm,xshift=1.25cm]regs.south east);
        \node[overText,inner sep=0pt, outer sep=0pt,font=\tiny,yshift=.4ex] at (eLabelPt) {\tt aluA + aluB};
        \coordinate (cLabel) at ([xshift=-2.5mm,yshift=2.15cm]muxAluA.input 1);
        \node[overText,inner sep=0pt, outer sep=0pt, font=\tiny,anchor=east] at (cLabel) {M[PC+2]};
        \draw (alu.south) -- ++(0,-2.5mm) node[below,inner sep=3pt,align=center,font=\scriptsize,fill=white,line width=2pt, draw=red,rectangle] (aluOverride) {add};
    \end{scope}
    \end{visibleenv}
    \begin{visibleenv}<1-4|handout:1>
    \begin{scope}[overlay]
        \node[draw,rectangle,line width=2pt,below right=1.8cm and -1.45cm of pc,text width=12cm,font=\small] {
            MUXes --- PC, dstM, dstE, aluA, aluB, dmemIn, dmemAddr, \ldots \\
            Exercise: what do they select when running \lstinline|addq %r8, %r9|?
        };
    \end{scope}
    \end{visibleenv}
    \begin{visibleenv}<3|handout:1>
        \draw[red,b] (muxSrcB.input 1) -- (muxSrcB.output);
        \draw[red,b] (muxDstM.input 2) -- (muxDstM.output);
        \draw[red,b] (muxDstE.input 1) -- (muxDstE.output);
        \draw[red,aa] (muxPc.input 3) -- (muxPc.output);
        \draw[red,aa] (muxAluA.input 2) -- (muxAluA.output);
        \draw[red,aa] (muxAluB.input 1) -- (muxAluB.output);
    \end{visibleenv}

    \begin{visibleenv}<5-6|handout:2>
    \begin{scope}[overlay]
        \node[draw,rectangle,line width=2pt,below right=1.8cm and -1.45cm of pc,text width=12cm,font=\small] {
            MUXes --- PC, dstM, dstE, aluA, aluB, dmemIn, dmemAddr, \ldots \\
            Exercise: what do they select for \rmmovq?
        };
    \end{scope}
    \end{visibleenv}
    \begin{visibleenv}<6|handout:2>
        \draw[red,b] (muxSrcB.input 1) -- (muxSrcB.output);
        \draw[red,b] (muxDstM.input 2) -- (muxDstM.output);
        \draw[red,b] (muxDstE.input 2) -- (muxDstE.output);
        \draw[red,aa] (muxPc.input 3) -- (muxPc.output);
        \draw[red,aa] (muxAluA.input 1) -- (muxAluA.output);
        \draw[red,aa] (muxAluB.input 1) -- (muxAluB.output);
    \end{visibleenv}
    \begin{visibleenv}<7|handout:3>
    \begin{scope}[overlay]
        \node[draw,rectangle,line width=2pt,below right=1.8cm and -1.45cm of pc,text width=12cm,font=\small] {
            MUXes --- PC, dstM, dstE, aluA, aluB, dmemIn, dmemAddr, \ldots \\
            Exercise: what do they select for {\keywordstyle irmovq}?
        };
    \end{scope}
    \end{visibleenv}

    \begin{visibleenv}<8|handout:4>
    \begin{scope}[overlay]
        \node[draw,rectangle,line width=2pt,below right=1.8cm and -1.45cm of pc,text width=12cm,font=\small] {
            MUXes --- PC, dstM, dstE, aluA, aluB, dmemIn, dmemAddr, \ldots \\
            Exercise: what do they select for {\keywordstyle mrmovq}?
        };
    \end{scope}
    \end{visibleenv}

    \begin{visibleenv}<9|handout:5>
    \begin{scope}[overlay]
        \node[draw,rectangle,line width=2pt,below right=1.8cm and -1.45cm of pc,text width=12cm,font=\small] {
            MUXes --- PC, dstM, dstE, aluA, aluB, dmemIn, dmemAddr, \ldots \\
            Exercise: what do they select for {\keywordstyle jle}?
        };
    \end{scope}
    \end{visibleenv}

    \begin{visibleenv}<10|handout:6>
    \begin{scope}[overlay]
        \node[draw,rectangle,line width=2pt,below right=1.8cm and -1.45cm of pc,text width=12cm,font=\small] {
            MUXes --- PC, dstM, dstE, aluA, aluB, dmemIn, dmemAddr, \ldots \\
            Exercise: what do they select for {\keywordstyle cmovle}?
        };
    \end{scope}
    \end{visibleenv}

    \begin{visibleenv}<11|handout:7>
    \begin{scope}[overlay]
        \node[draw,rectangle,line width=2pt,below right=1.8cm and -1.45cm of pc,text width=12cm,font=\small] {
            MUXes --- PC, dstM, dstE, aluA, aluB, dmemIn, dmemAddr, \ldots \\
            Exercise: what do they select for {\keywordstyle ret}?
        };
    \end{scope}
    \end{visibleenv}

    \begin{visibleenv}<12|handout:8>
    \begin{scope}[overlay]
        \node[draw,rectangle,line width=2pt,below right=1.8cm and -1.45cm of pc,text width=12cm,font=\small] {
            MUXes --- PC, dstM, dstE, aluA, aluB, dmemIn, dmemAddr, \ldots \\
            Exercise: what do they select for {\keywordstyle popq}?
        };
    \end{scope}
    \end{visibleenv}

    \begin{visibleenv}<13|handout:9>
    \begin{scope}[overlay]
        \node[draw,rectangle,line width=2pt,below right=1.8cm and -1.45cm of pc,text width=12cm,font=\small] {
            MUXes --- PC, dstM, dstE, aluA, aluB, dmemIn, dmemAddr, \ldots \\
            Exercise: what do they select for {\keywordstyle call}?
        };
    \end{scope}
    \end{visibleenv}
\end{tikzpicture}
\end{frame}

\againframe<5-6|handout:2>{SettingMuxes}

\againframe<7|handout:3>{SettingMuxes}

\againframe<8|handout:4>{SettingMuxes}

\againframe<9|handout:5>{SettingMuxes}

\againframe<10|handout:6>{SettingMuxes}

\againframe<11|handout:7>{SettingMuxes}

\againframe<12|handout:8>{SettingMuxes}

\againframe<13|handout:8>{SettingMuxes}
 % FIXME: remove push/pop examples

\section{backup slides}
\begin{frame}{backup slides}
\end{frame}

\subsection{reference implementation: yis}

\begin{frame}[fragile,label=compareNopJmp]{comparing to yis}
\begin{Verbatim}[fontsize=\fontsize{8}{9}\selectfont]
$ ./hclrs nopjmp_cpu.hcl nopjmp.yo
...
...
+--------------------- (end of halted state) ---------------------------+
Cycles run: 7
\end{Verbatim}
\begin{Verbatim}[fontsize=\fontsize{8}{9}\selectfont]
$ ./tools/yis nopjmp.yo
Stopped in 7 steps at PC = 0x1e.  Status 'HLT', CC Z=1 S=0 O=0
Changes to registers:

Changes to memory:
\end{Verbatim}
\end{frame}


\subsection{HCL summary}
\begin{frame}{HCLRS summary}
    \begin{itemize}
    \item declare/assign values to \myemph{wires}
    \item \myemph{MUXes} with
        \begin{itemize}
        \item {\tt [ test1: value1; test2: value2; 1: default; ]}
        \end{itemize}
    \item register banks with {\tt \textbf{register} iO}:
        \begin{itemize}
        \item next value on {\tt i\_name}; current value on {\tt O\_name}
        \end{itemize}
    \item fixed functionality
        \begin{itemize}
        \item register file (15 registers; 2 read + 2 write)
        \item memories (data + instruction)
        \item Stat register (start/stop/error)
        \end{itemize}
    \end{itemize}
\end{frame}


\end{document}
