% vectorizing matmul

\begin{frame}[fragile,label=sqSerial]{recall: matmul}
\lstset{language=C,style=small}
\begin{lstlisting}
void matmul(unsigned int *A, unsigned int *B) {
    for (int k = 0; k < N; ++k)
        for (int i = 0; i < N; ++i)
            for (int j = 0; j < N; ++j)
                B[i * N + j] += A[i * N + k] * A[k * N + j];
}
\end{lstlisting}
\end{frame}

\begin{frame}[fragile,label=sqSerialUnrolled]{matmul unrolled}
\lstset{language=C,style=smaller}
\begin{lstlisting}
void matmul(unsigned int *A, unsigned int *B) {
  for (int k = 0; k < N; ++k) {
    for (int i = 0; i < N; ++i)
      for (int j = 0; j < N; j += 4) {
        /* goal: vectorize this */
        B[i * N + j + 0] += A[i * N + k] * A[k * N + j + 0];
        B[i * N + j + 1] += A[i * N + k] * A[k * N + j + 1];
        B[i * N + j + 2] += A[i * N + k] * A[k * N + j + 2];
        B[i * N + j + 3] += A[i * N + k] * A[k * N + j + 3];
      }
}
\end{lstlisting}
\end{frame}

\begin{frame}[fragile,label=sqVectInstr]{handy intrinsic functions for matmul}
    \begin{itemize}
        \item {\tt \_mm\_set1\_epi32} --- load four copies of a 32-bit value into a 128-bit value
            \begin{itemize}
            \item instructions generated vary; one example: {\tt movq} + {\tt pshufd}
            \end{itemize}
        \item {\tt \_mm\_mullo\_epi32} --- multiply four pairs of 32-bit values, give lowest 32-bits of results
            \begin{itemize}
            \item generates {\tt pmulld}
            \end{itemize}
    \end{itemize}
\end{frame}

\begin{frame}[fragile,label=sqSerialUnrolledExpand]{vectorizing matmul}
\lstset{language=C,style=smaller,
    moredelim=**[is][\btHL<all:2>]{~2~}{~end~},
    moredelim=**[is][\btHL<all:3>]{~3~}{~end~},
    moredelim=**[is][\btHL<all:4>]{~4~}{~end~},
    moredelim=**[is][\btHL<all:5>]{~5~}{~end~},
}
\begin{lstlisting}
/* goal: vectorize this */
~2~B[i * N + j + 0]~end~ ~5~+=~end~ ~4~A[i * N + k] *~end~ ~3~A[k * N + j + 0]~end~;
~2~B[i * N + j + 1]~end~ ~5~+=~end~ ~4~A[i * N + k] *~end~ ~3~A[k * N + j + 1]~end~;
~2~B[i * N + j + 2]~end~ ~5~+=~end~ ~4~A[i * N + k] *~end~ ~3~A[k * N + j + 2]~end~;
~2~B[i * N + j + 3]~end~ ~5~+=~end~ ~4~A[i * N + k] *~end~ ~3~A[k * N + j + 3]~end~;
\end{lstlisting}
\hrulefill
\begin{onlyenv}<all:2>
\begin{lstlisting}
// load four elements from B
Bij = _mm_loadu_si128(&B[i * N + j + 0]);
... // manipulate vector here
// store four elements into B
_mm_storeu_si128((__m128i*) &B[i * N + j + 0], Bij);
\end{lstlisting}
\end{onlyenv}
\begin{onlyenv}<all:3>
\begin{lstlisting}
// load four elements from A
Akj = _mm_loadu_si128(&A[k * N + j + 0]);
... // multiply each by A[i * N + k] here
\end{lstlisting}
\end{onlyenv}
\begin{onlyenv}<all:4>
\begin{lstlisting}
// load four elements starting with A[k * n + j]
Akj = _mm_loadu_si128(&A[k * N + j + 0]);
// load four copies of A[i * N + k]
Aik = _mm_set1_epi32(A[i * N + k]);
// multiply each pair
multiply_results = _mm_mullo_epi32(Aik, Akj);
\end{lstlisting}
\end{onlyenv}
\begin{onlyenv}<all:5>
\begin{lstlisting}
Bij = _mm_add_epi32(Bij, multiply_results);
// store back results
_mm_storeu_si128(..., Bij);
\end{lstlisting}
\end{onlyenv}
\end{frame}



\begin{frame}[fragile,label=sqVectCode]{matmul vectorized}
\lstset{language=C,style=smaller,morekeywords={\_\_m128i},mathescape=true}
\begin{lstlisting}
__m128i Cij, Bkj, Aik, Aik_times_Akj;

// Cij = {$C_{i,j}$, $C_{i,j+1}$, $C_{i,j+2}$, $C_{i,j+3}$}
Cij = _mm_loadu_si128((__m128i*) &B[i * N + j]);
// Akj = {$A_{k,j}$, $A_{k,j+1}$, $A_{k,j+2}$, $A_{k,j+3}$}
Bkj = _mm_loadu_si128((__m128i*) &A[k * N + j]);

// Aik = {$A_{i,k}$, $A_{i,k}$, $A_{i,k}$, $A_{i,k}$}
Aik = _mm_set1_epi32(A[i * N + k]);

// Aik_times_Akj = {$A_{i,k}\times A_{k,j}$, $A_{i,k}\times A_{k,j+1}$, $A_{i,k}\times A_{k,j+2}$, $A_{i,k}\times A_{k,j+3}$}
Aik_times_Bkj = _mm_mullo_epi32(Aij, Bkj);

// Cij= {$C_{i,j} + A_{i,k}\times B_{k,j}$, $C_{i,j+1} + A_{i,k}\times B_{k,j+1}$, ...}
Cij = _mm_add_epi32(Bij, Aik_times_Bkj);

// store Bij into B
_mm_storeu_si128((__m128i*) &B[i * N + j], Bij);
\end{lstlisting}
\end{frame}

