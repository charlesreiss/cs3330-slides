\usetikzlibrary{shapes.misc,shapes.callouts}

\begin{frame}[fragile,label=SEQIFetch]{SEQ: instruction fetch}
\begin{itemize}
\item read instruction memory at PC
\item split into seperate wires:
\begin{itemize}
    \item \myemph{icode:ifun} --- opcode
    \item \myemph{rA}, \myemph{rB} --- register numbers
    \item \myemph{valC} --- call target or mov displacement % FIXME: MUX based on instruction mark
\end{itemize}
\item compute next instruction address:
\begin{itemize}
    \item \myemph{valP} --- PC + (instr length)
\end{itemize}
\end{itemize}
\end{frame}

\begin{frame}[fragile,label=ifetchCPU]{instruction fetch}
\begin{tikzpicture}[circuit logic]
    \tikzset{
        dmemLabel/.style={visible on=<0|handout:0>},
        dmemInputLabel/.style={visible on=<1|handout:1>},
        instrRegs/.style={visible on=<0|handout:0>},
        regsLogic/.style={visible on=<0|handout:0>},
        logicDmem/.style={visible on=<0|handout:0>},
        dmemPC/.style={visible on=<1|handout:1>},
        dmemPCMux/.style={visible on=<0|handout:0>},
        dmemPCNoMux/.style={visible on=<0|handout:0>},
        dmemWB/.style={visible on=<0|handout:0>},
        bookLabel/.style={color=red!60!black,font=\small\bfseries,outer sep=0pt,inner sep=1pt,fill=white},
        instrRegs/.style={visible on=<1-|handout:1>},
        instrRegsSplitOut/.style={visible on=<1-|handout:1>},
        instrRegsRS1/.style={visible on=<0|handout:0>},
        instrRegsMux/.style={visible on=<0|handout:0>},
    }
    \circuitState
    \circuitConnectDetail
    \draw[a] (iLenPlus) -- ++ (-1cm, 0cm) node[left,bookLabel] {valP};
\end{tikzpicture}
\end{frame}

\begin{frame}[fragile,label=SEQIDecode]{SEQ: instruction ``decode''}
\begin{itemize}
\item read registers
\begin{itemize}
    \item \myemph{valA}, \myemph{valB} --- register values
\end{itemize}
\end{itemize}
\end{frame}

\tikzset{overlayQuestion/.style={overlay,anchor=south,at={([yshift=.5cm]current page.south)},align=left,fill=white}}

\begin{frame}[fragile,label=idecodeCPUBroken]{instruction decode (1)}
\begin{tikzpicture}[circuit logic]
    \tikzset{
        dmemLabel/.style={visible on=<0|handout:0>},
        dmemInputLabel/.style={visible on=<1-|handout:1>},
        instrRegs/.style={visible on=<0|handout:0>},
        regsLogic/.style={visible on=<0|handout:0>},
        logicDmem/.style={visible on=<0|handout:0>},
        dmemPC/.style={visible on=<1-|handout:1>},
        dmemPCMux/.style={visible on=<0|handout:0>},
        dmemPCNoMux/.style={visible on=<0|handout:0>},
        dmemWB/.style={visible on=<0|handout:0>},
        bookLabel/.style={color=red!60!black,font=\small\bfseries,outer sep=0pt,inner sep=1pt,fill=white},
        instrRegs/.style={visible on=<1-|handout:1>},
        instrRegsSplitOut/.style={visible on=<1-|handout:1>},
        instrRegsRS1/.style={visible on=<1-|handout:1>},
        instrRegsMux/.style={visible on=<0|handout:0>},
        instrRegsNoMuxRS2/.style={visible on=<1-|handout:1>},
    }
    \circuitState
    \circuitConnectDetail
    \draw[a] (iLenPlus) -- ++ (-1cm, 0cm) node[left,bookLabel] {valP};
\end{tikzpicture}
\begin{tikzpicture}[remember picture]
    \begin{visibleenv}<2>
        \node[overlayQuestion] {
            exercise: for which instructions would there be a problem ? \\
            {\tt nop}, {\tt addq}, {\tt mrmovq}, {\tt rmmovq}, {\tt jmp}, {\tt pushq} \iftoggle{heldback}{}{\only<3->{\color{red} of these: only pushq} }
        };
    \end{visibleenv}
\end{tikzpicture}
\end{frame}

\begin{frame}[fragile,label=SEQIDecodeSrc]{SEQ: srcA, srcB}
\begin{itemize}
\item always read rA, rB?
\item Problems: 
\begin{itemize}
    \item push rA
    \item pop
    \item call
    \item ret
\end{itemize}
\item book: extra signals: srcA, srcB --- computed input register
\item MUX controlled by icode
\end{itemize}
\end{frame}

\newcommand{\pushq}{{\keywordstyle pushq}}
\newcommand{\popq}{{\keywordstyle popq}}


\begin{frame}[fragile,label=SEQIDecodeSrcRegs]{SEQ: possible registers to read}
\small
\begin{tabular}{l|ll}
instruction & srcA & srcB \\ \hline
\halt, \nop, {\keywordstyle j{\it CC}}, \irmovq & none & none \\
{\keywordstyle cmovCC}, {\keywordstyle rrmovq} & rA & none \\
\mrmovq & none & rB \\
\rmmovq, {\keywordstyle {\it OP}q} & rA & rB \\
{\keywordstyle call}, {\keywordstyle ret} & none? & \myemph<2>{\tt \%rsp} \\
\pushq, \popq & rA & \myemph<2>{\tt \%rsp} \\
\end{tabular}
\begin{tikzpicture}[circuit logic]
    \instrEncodingStyles
    \node[draw,mux,inputs={nnn},info={center:MUX},minimum height=3cm,minimum width=1.5cm] (rBMux) {};
    \draw[thick,-latex] (rBMux.output) -- +(0:5mm) node[right,font=\small] {srcB};
    \draw[thick,latex-] (rBMux.input 1) -- +(180:5mm) node[left] {\rnify{rB}};
    \draw[thick,latex-] (rBMux.input 2) -- +(180:5mm) node[left] {\rnifyWide{\%rsp}};
    \draw[thick,latex-] (rBMux.input 3) -- +(180:5mm) node[left] (noneLabel) {(none) \literalify{F}};
    \node[logicBlock,below=.5cm of rBMux.select] (logic) {logic function};
    \draw[b] (logic) -- (rBMux.select);
    \draw[b,latex-] (logic.west) -- +(180:5mm) node[left] {\icode};
    \onslide<2->{
        \node[draw,red,cross out,fit=(noneLabel)] {};
    }
\end{tikzpicture}
\end{frame}

\begin{frame}[fragile,label=idecodeCPUWorking]{instruction decode (2)}
\begin{tikzpicture}[circuit logic]
    \tikzset{
        dmemLabel/.style={visible on=<0|handout:0>},
        dmemInputLabel/.style={visible on=<1|handout:1>},
        instrRegs/.style={visible on=<0|handout:0>},
        regsLogic/.style={visible on=<0|handout:0>},
        logicDmem/.style={visible on=<0|handout:0>},
        dmemPC/.style={visible on=<1|handout:1>},
        dmemPCMux/.style={visible on=<0|handout:0>},
        dmemPCNoMux/.style={visible on=<0|handout:0>},
        dmemWB/.style={visible on=<0|handout:0>},
        bookLabel/.style={color=red!60!black,font=\small\bfseries,outer sep=0pt,inner sep=1pt,fill=white},
        instrRegs/.style={visible on=<1|handout:1>},
        instrRegsRS1/.style={visible on=<1|handout:1>},
        instrRegsMux/.style={visible on=<1|handout:1>},
        instrRegsMuxRS3/.style={visible on=<0|handout:0>},
        instrRegsMuxRS4/.style={visible on=<0|handout:0>},
    }
    \circuitState
    \circuitConnectDetail
    \draw[a] (iLenPlus) -- ++ (-1cm, 0cm) node[left,bookLabel] {valP};
\end{tikzpicture}
\end{frame}

\begin{frame}[fragile,label=SEQExecute]{SEQ: execute}
\begin{itemize}
\item perform ALU operation (add, sub, xor, and)
\begin{itemize}
    \item \myemph{valE} --- ALU output
\end{itemize}
\item read prior condition codes
\begin{itemize}
    \item \myemph{Cnd} --- condition codes based on ifun (instruction type for jCC/cmovCC)
\end{itemize}
\item write new condition codes
\end{itemize}
\end{frame}

\begin{frame}[fragile,label=UsingCCodes]{using condition codes: cmov\*}
\begin{tikzpicture}[circuit logic US]
\instrEncodingStyles
\node[draw,mux,inputs={nnnnnnn},minimum height=5cm] (condMux) {};
\begin{scope}[thick,latex-]
\draw (condMux.input 1) -- ++(-.5cm, 0cm) node[left,font=\small] {(\textit{always}) {\tt 1}};
\draw (condMux.input 2) -- ++(-.5cm, 0cm) node[left,font=\small] {(\textit{le}) {\tt SF | ZF}};
\draw (condMux.input 3) -- ++(-.5cm, 0cm) node[left,font=\small] {(\textit{l}) {\tt SF}};
\draw (condMux.input 4) -- ++(-.5cm, 0cm);
\draw (condMux.input 5) -- ++(-.5cm, 0cm);
\draw (condMux.input 6) -- ++(-.5cm, 0cm);
\draw (condMux.input 7) -- ++(-.5cm, 0cm);
\end{scope}
\node[below=.5cm of condMux.select,font=\small,align=center] (cc) {\ccify{cc}~\\(from instr)};
\draw[thick,-latex] (cc) -- (condMux.select);
\node[draw,mux,minimum height=2cm,inputs={nnn},right=5cm of condMux] (dstMuxE) {};
\begin{scope}[thick,latex-]
\draw (dstMuxE.input 1) -- ++(-.5cm, 0cm) node[left,font=\small] {\vrB};
\draw (dstMuxE.input 2) -- ++(-.5cm, 0cm) node[left,font=\small] {\tt 0xF};
\end{scope}
\begin{scope}[thick,-latex]
\draw (dstMuxE.output) -- ++(.5cm, 0cm) node[right,font=\small] {dstE};
\end{scope}
\node [not gate,minimum width=1.2cm] (notGate) at ([xshift=1cm]condMux.output) {\scriptsize NOT};
\draw[dashed,thick,-latex] (condMux.output) -- (notGate.input);
\draw[dashed,thick,-latex] (notGate.output) -- ++(.25cm, 0cm) |- ([xshift=-.25cm,yshift=-.25cm]dstMuxE.south west)
        -| (dstMuxE.select);
\end{tikzpicture}
\end{frame}

\begin{frame}[fragile,label=executeCPUBroken]{execute (1)}
\begin{tikzpicture}[circuit logic]
    \tikzset{
        dmemLabel/.style={visible on=<0|handout:0>},
        dmemInputLabel/.style={visible on=<1-|handout:1>},
        instrRegs/.style={visible on=<0|handout:0>},
        regsLogic/.style={visible on=<1-|handout:1>},
        regsLogicNoMux/.style={visible on=<1-|handout:1>},
        regsLogicMux/.style={visible on=<0|handout:0>},
        logicDmem/.style={visible on=<0|handout:0>},
        dmemPC/.style={visible on=<1-|handout:1>},
        dmemPCMux/.style={visible on=<0|handout:0>},
        dmemPCNoMux/.style={visible on=<0|handout:0>},
        dmemWB/.style={visible on=<0|handout:0>},
        instrRegs/.style={visible on=<1-|handout:1>},
        instrRegsSplitImmed/.style={visible on=<1-|handout:1>},
        instrRegsRS1/.style={visible on=<1-|handout:1>},
        instrRegsMux/.style={visible on=<1-|handout:1>},
        instrRegsMuxRS3/.style={visible on=<0|handout:0>},
        instrRegsMuxRS4/.style={visible on=<0|handout:0>},
    }
    \circuitState
    \circuitConnectDetail
    \draw[a] (iLenPlus) -- ++ (-1cm, 0cm) node[left,bookLabel] {valP};
\end{tikzpicture}
\begin{tikzpicture}[remember picture]
    \begin{visibleenv}<2>
        \begin{scope}[overlay]
            \node[overlayQuestion] {
                exercise: which of these instructions would there be a problem ? \\
                {\tt nop}, {\tt addq}, {\tt mrmovq}, {\tt popq}, {\tt call}, 
            };
        \end{scope}
    \end{visibleenv}
\end{tikzpicture}
\end{frame}

\begin{frame}[fragile,label=SEQExecuteALU]{SEQ: ALU operations?}
\begin{itemize}
% FIXME: ALU picture
\item ALU inputs always \myemph{valA}, \myemph{valB} (register values)?
\item no, inputs from instruction: (Displacement + rB)
\begin{tikzpicture}[overlay,circuit logic,global scale=0.5,yshift=-1.2cm]
    \node[draw,mux,inputs={nn},info={center:MUX},minimum height=2cm,minimum width=1cm] (BMux) {};
    \draw[thick,-latex] (BMux.output) -- +(0:5mm) node[right,font=\small] {aluA};
    \draw[thick,latex-] (BMux.input 1) -- +(180:5mm) node[left] {valA};
    \draw[thick,latex-] (BMux.input 2) -- +(180:5mm) node[left] {valC};
\end{tikzpicture}
\begin{itemize}
    \item \mrmovq
    \item \rmmovq
\end{itemize}
\item no, constants: (rsp +/- 8)
\begin{itemize}
    \item \pushq
    \item \popq
    \item {\keywordstyle call}
    \item {\keywordstyle ret}
\end{itemize}
\item extra signals: \myemph{aluA}, \myemph{aluB}
\begin{itemize}
    \item computed ALU input values
\end{itemize}
\end{itemize}
\end{frame}

\begin{frame}[fragile,label=executeCPUWorkingNoBMux]{execute (2)}
    \begin{tikzpicture}[circuit logic]
    \tikzset{
        dmemLabel/.style={visible on=<0|handout:0>},
        dmemInputLabel/.style={visible on=<1-|handout:1>},
        instrRegs/.style={visible on=<0|handout:0>},
        regsLogic/.style={visible on=<1-|handout:1>},
        regsLogicMux/.style={visible on=<0|handout:0>},
        regsLogicMuxA/.style={visible on=<1-|handout:1>},
        regsLogicNoMuxB/.style={visible on=<1-|handout:1>},
        logicDmem/.style={visible on=<0|handout:0>},
        dmemPC/.style={visible on=<1-|handout:1>},
        dmemPCMux/.style={visible on=<0|handout:0>},
        dmemPCNoMux/.style={visible on=<0|handout:0>},
        dmemWB/.style={visible on=<0|handout:0>},
        instrRegs/.style={visible on=<1-|handout:1>},
        %instrRegsSplitImmed/.style={visible on=<1-|handout:1>},
        instrRegsRS1/.style={visible on=<1-|handout:1>},
        instrRegsMux/.style={visible on=<1-|handout:1>},
        instrRegsMuxRS3/.style={visible on=<0|handout:0>},
        instrRegsMuxRS4/.style={visible on=<0|handout:0>},
    }
    \circuitState
    \circuitConnectDetail
    \draw[a] (iLenPlus) -- ++ (-1cm, 0cm) node[left,bookLabel] {valP};
    \end{tikzpicture}
\end{frame}

% FIXME: computing ALU input values

\begin{frame}[fragile,label=SEQMemory]{SEQ: Memory}
\begin{itemize}
\item read or write data memory
\begin{itemize}
    \item \myemph{valM} --- value read from memory (if any)
\end{itemize}
\end{itemize}
\end{frame}

\begin{frame}[fragile,label=memoryNoMux]{memory (1)}
    \begin{tikzpicture}[circuit logic]
    \tikzset{
        dmemLabel/.style={visible on=<0|handout:0>},
        dmemInputLabel/.style={visible on=<1-|handout:1>},
        instrRegs/.style={visible on=<0|handout:0>},
        regsLogic/.style={visible on=<1-|handout:1>},
        regsLogicMux/.style={visible on=<0|handout:0>},
        regsLogicMuxA/.style={visible on=<1-|handout:1>},
        %regsLogicMuxB/.style={visible on=<1-|handout:1>},
        regsLogicNoMuxB/.style={visible on=<1-|handout:1>},
        logicDmem/.style={visible on=<1-|handout:1>},
        logicDmemMux/.style={visible on=<0|handout:0>},
        logicDmemNoMux/.style={visible on=<1-|handout:1>},
        dmemPC/.style={visible on=<1-|handout:1>},
        dmemPCMux/.style={visible on=<0|handout:0>},
        dmemPCNoMux/.style={visible on=<0|handout:0>},
        dmemWB/.style={visible on=<0|handout:0>},
        instrRegs/.style={visible on=<1-|handout:1>},
        %instrRegsSplitImmed/.style={visible on=<1-|handout:1>},
        instrRegsRS1/.style={visible on=<1-|handout:1>},
        instrRegsMux/.style={visible on=<1-|handout:1>},
        instrRegsMuxRS3/.style={visible on=<0|handout:0>},
        instrRegsMuxRS4/.style={visible on=<0|handout:0>},
    }
    \circuitState
    \circuitConnectDetail
    \draw[a] (iLenPlus) -- ++ (-1cm, 0cm) node[left,bookLabel] {valP};
\end{tikzpicture}
\begin{tikzpicture}[remember picture]
    \begin{visibleenv}<2>
        \node[overlayQuestion] {
            exercise: which of these instructions would there be a problem ? \\
            {\tt nop}, {\tt rmmovq}, {\tt mrmovq}, {\tt popq}, {\tt call}, 
        };
    \end{visibleenv}
\end{tikzpicture}
\end{frame}

\begin{frame}[fragile,label=SEQMemoryControl]{SEQ: control signals for memory}
\begin{itemize}
\item read/write --- \myemph{read enable}? \myemph{write enable}?
\item \myemph{Addr} --- address 
    \begin{itemize}
        \item mostly ALU output
        \item special cases (need extra MUX): \popq, {\keywordstyle ret}
    \end{itemize}
\item \myemph{Data} --- value to write
    \begin{itemize}
        \item mostly valA
        \item special cases (need extra MUX): {\keywordstyle call}
    \end{itemize}
\end{itemize}
\end{frame}

% FIXME: Addr == ALUOutput?
% FIXME: Data == valA?

\begin{frame}[fragile,label=memoryMux]{memory (2)}
    \begin{tikzpicture}[circuit logic]
    \tikzset{
        dmemLabel/.style={visible on=<0|handout:0>},
        dmemInputLabel/.style={visible on=<1-|handout:1>},
        instrRegs/.style={visible on=<0|handout:0>},
        regsLogic/.style={visible on=<1-|handout:1>},
        regsLogicMux/.style={visible on=<0|handout:0>},
        regsLogicMuxA/.style={visible on=<1-|handout:1>},
        %regsLogicMuxB/.style={visible on=<1-|handout:1>},
        regsLogicNoMuxB/.style={visible on=<1-|handout:1>},
        logicDmem/.style={visible on=<1-|handout:1>},
        dmemPC/.style={visible on=<1-|handout:1>},
        dmemPCMux/.style={visible on=<0|handout:0>},
        dmemPCNoMux/.style={visible on=<0|handout:0>},
        dmemWB/.style={visible on=<0|handout:0>},
        instrRegs/.style={visible on=<1-|handout:1>},
        %instrRegsSplitImmed/.style={visible on=<1-|handout:1>},
        instrRegsRS1/.style={visible on=<1-|handout:1>},
        instrRegsMux/.style={visible on=<1-|handout:1>},
        instrRegsMuxRS3/.style={visible on=<0|handout:0>},
        instrRegsMuxRS4/.style={visible on=<0|handout:0>},
    }
    \circuitState
    \circuitConnectDetail
    \draw[a] (iLenPlus) -- ++ (-1cm, 0cm) node[left,bookLabel] {valP};
    \end{tikzpicture}
\end{frame}

\begin{frame}[fragile,label=SEQWriteBack]{SEQ: write back}
\begin{itemize}
\item write registers
\end{itemize}
\end{frame}

\begin{frame}[fragile,label=writeBackNoMux]{write back (1)}
    \begin{tikzpicture}[circuit logic]
    \tikzset{
        dmemLabel/.style={visible on=<0|handout:0>},
        dmemInputLabel/.style={visible on=<1-|handout:1>},
        instrRegs/.style={visible on=<0|handout:0>},
        regsLogic/.style={visible on=<1-|handout:1>},
        regsLogicMux/.style={visible on=<0|handout:0>},
        regsLogicMuxA/.style={visible on=<1-|handout:1>},
        logicDmem/.style={visible on=<1-|handout:1>},
        dmemPC/.style={visible on=<1-|handout:1>},
        dmemPCMux/.style={visible on=<0|handout:0>},
        dmemPCNoMux/.style={visible on=<0|handout:0>},
        dmemWB/.style={visible on=<1-|handout:1->},
        dmemOutToPC/.style={visible on=<0|handout:0>},
        instrRegs/.style={visible on=<1-|handout:1>},
        %instrRegsSplitImmed/.style={visible on=<1-|handout:1>},
        instrRegsRS1/.style={visible on=<1-|handout:1>},
        instrRegsMux/.style={visible on=<1-|handout:1>},
        instrRegsNoMuxRS3/.style={visible on=<1-|handout:1>},
        instrRegsNoMuxRS4/.style={visible on=<1-|handout:1>},
        instrRegsMuxRS3/.style={visible on=<0|handout:0>},
        instrRegsMuxRS4/.style={visible on=<0|handout:0>},
        regsLogicNoMuxB/.style={visible on=<1-|handout:1>},
    }
    \circuitState
    \circuitConnectDetail
    \draw[a] (iLenPlus) -- ++ (-1cm, 0cm) node[left,bookLabel] {valP};
    \begin{visibleenv}<2>
        \node[overlay,my callout2=regSelect3,align=left,anchor=south east,font=\small] at ([yshift=1cm,xshift=4cm]regSelect3) {
            textbook convention: \\
            E used for storing ALU results (e.g. add) \\
            M used for storing memory results (e.g. rmmovq) \\
            (you don't have to do this\ldots)
        };
    \end{visibleenv}
    \end{tikzpicture}
    \begin{tikzpicture}[remember picture]
    \begin{visibleenv}<3>
            \node[overlayQuestion] {
                exercise: which of these instructions would there be a problem ? \\
                {\tt nop}, {\tt irmovq}, {\tt mrmovq}, {\tt rmmovq}, {\tt addq}, {\tt popq}
            };
    \end{visibleenv}
    \end{tikzpicture}
\end{frame}


\begin{frame}[fragile,label=SEQWriteBackControl]{SEQ: control signals for WB}
\begin{itemize}
\item \myemph{two} write inputs --- two needed by \lstinline|popq|
\begin{itemize}
\item valM (memory output), valE (ALU output)
\end{itemize}
\item \myemph{two} register numbers
\begin{itemize}
\item dstM, dstE
\end{itemize}
\item write disable --- use dummy register number {\tt 0xF}
\end{itemize}
\begin{tikzpicture}[circuit logic]
    \instrEncodingStyles
    \node[draw,mux,inputs={nnn},info={center:MUX},minimum height=2cm,minimum width=1.2cm] (Mux) {};
    \draw[thick,-latex] (Mux.output) -- +(0:5mm) node[right,font=\small] {dstE};
    \draw[thick,latex-] (Mux.input 1) -- +(180:5mm) node[left] {\rnify{rB}};
    \draw[thick,latex-] (Mux.input 2) -- +(180:5mm) node[left] {\literalify{F}};
    \draw[thick,latex-] (Mux.input 3) -- +(180:5mm) node[left] {\rnifyWide{\%rsp}};
\end{tikzpicture}
\end{frame}

\begin{frame}[fragile,label=writeBackMuxA]{write back (2a)}
    \begin{tikzpicture}[circuit logic]
    \tikzset{
        dmemLabel/.style={visible on=<0|handout:0>},
        dmemInputLabel/.style={visible on=<1-|handout:1>},
        instrRegs/.style={visible on=<0|handout:0>},
        regsLogic/.style={visible on=<1-|handout:1>},
        regsLogicMux/.style={visible on=<0|handout:0>},
        regsLogicMuxA/.style={visible on=<1-|handout:1>},
        regsLogicNoMuxB/.style={visible on=<1-|handout:1>},
        logicDmem/.style={visible on=<1-|handout:1>},
        dmemPC/.style={visible on=<1-|handout:1>},
        dmemPCMux/.style={visible on=<0|handout:0>},
        dmemPCNoMux/.style={visible on=<0|handout:0>},
        dmemWB/.style={visible on=<1-|handout:1->},
        dmemWBvalEMux/.style={visible on=<1-|handout:1->,red},
        dmemWBvalENoMux/.style={visible on=<0|handout:0>},
        dmemOutToPC/.style={visible on=<0|handout:0>},
        instrRegs/.style={visible on=<1-|handout:1>},
        %instrRegsSplitImmed/.style={visible on=<1-|handout:1>},
        instrRegsRS1/.style={visible on=<1-|handout:1>},
        instrRegsMux/.style={visible on=<1-|handout:1>},
    }
    \circuitState
    \circuitConnectDetail
    \draw[a] (iLenPlus) -- ++ (-1cm, 0cm) node[left,bookLabel] {valP};
    \end{tikzpicture}
\end{frame}

\begin{frame}[fragile,label=writeBackMuxB]{write back (2b)}
    \begin{tikzpicture}[circuit logic]
    \tikzset{
        dmemLabel/.style={visible on=<0|handout:0>},
        dmemInputLabel/.style={visible on=<1-|handout:1>},
        instrRegs/.style={visible on=<0|handout:0>},
        regsLogic/.style={visible on=<1-|handout:1>},
        regsLogicMux/.style={visible on=<1-|handout:1>},
        regsLogicMuxB/.style={red},
        logicDmem/.style={visible on=<1-|handout:1>},
        dmemPC/.style={visible on=<1-|handout:1>},
        dmemPCMux/.style={visible on=<0|handout:0>},
        dmemPCNoMux/.style={visible on=<0|handout:0>},
        dmemWB/.style={visible on=<1-|handout:1->},
        dmemOutToPC/.style={visible on=<0|handout:0>},
        instrRegs/.style={visible on=<1-|handout:1>},
        %instrRegsSplitImmed/.style={visible on=<1-|handout:1>},
        instrRegsRS1/.style={visible on=<1-|handout:1>},
        instrRegsMux/.style={visible on=<1-|handout:1>},
    }
    \circuitState
    \circuitConnectDetail
    \draw[a] (iLenPlus) -- ++ (-1cm, 0cm) node[left,bookLabel] {valP};
    \end{tikzpicture}
\end{frame}


\begin{frame}[fragile,label=SEQUpdatePC]{SEQ: Update PC}
\begin{itemize}
\item choose value for PC next cycle (input to PC register)
\begin{itemize}
\item usually valP (following instruction)
\item exceptions: {\keywordstyle call}, {\keywordstyle j{\it CC}}, {\keywordstyle ret}
\end{itemize}
\end{itemize}
\end{frame}

\begin{frame}[fragile,label=pcUpdateCircuit]{PC update}
    \begin{tikzpicture}[circuit logic]
    \tikzset{
        dmemLabel/.style={visible on=<0|handout:0>},
        dmemInputLabel/.style={visible on=<1-|handout:1>},
        instrRegs/.style={visible on=<0|handout:0>},
        regsLogic/.style={visible on=<1-|handout:1>},
        regsLogicMux/.style={visible on=<0|handout:0>},
        regsLogicMuxA/.style={visible on=<1-|handout:1>},
        regsLogicMuxB/.style={visible on=<1-|handout:1>},
        logicDmem/.style={visible on=<1-|handout:1>},
        dmemPC/.style={visible on=<1-|handout:1>},
        dmemWB/.style={visible on=<1-|handout:1->},
        dmemOutToPC/.style={visible on=<1-|handout:1->},
        instrRegs/.style={visible on=<1-|handout:1>},
        %instrRegsSplitImmed/.style={visible on=<1-|handout:1>},
        instrRegsRS1/.style={visible on=<1-|handout:1>},
        instrRegsMux/.style={visible on=<1-|handout:1>},
    }
    \circuitState
    \circuitConnectDetail
    \draw[a] (iLenPlus) -- ++ (-1cm, 0cm) node[left,bookLabel] {valP};
    \end{tikzpicture}
\end{frame}

