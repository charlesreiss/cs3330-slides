\tikzset{wire/.style={draw,thick,>=Latex}}

\begin{frame}{describing hardware}
    \begin{itemize}
    \item how do we describe hardware?
    \item pictures?
    \end{itemize}
    \begin{tikzpicture}
        \node[draw,minimum width=1cm,minimum height=1cm,fill=blue!20] (add1) at (3, 2) { add 1 };
        \node[hRegSmall={count}] (reg) at (4.5, 2) {};
        \draw[wire,->] (add1) -- (reg);
        \draw[wire,->] (reg) -- (6, 2) -- (6, 0) -- (0, 0) -- (0, 2) -- (add1);
        \coordinate (pinPoint) at (6, 1);
    \end{tikzpicture}
\end{frame}

\begin{frame}{circuits with pictures?}
    \begin{itemize}
    \item yes, something you can do
    \item such commercial tools exist, but\ldots
    \vspace{.5cm}
    \item not commonly used for processors
    \end{itemize}
\end{frame}

\begin{frame}{hardware description language}
    \begin{itemize}
    \item \myemph{programming language for hardware}
    \item (typically) text-based representation of circuit
    \vspace{.5cm}
    \item often abstracts away details like:
        \begin{itemize}
        \item how to build arithmetic operations from gates
        \item how to build registers from transistors
        \item how to build memories from transistors
        \item how to build MUXes from gates
        \item \ldots
        \end{itemize}
    \item those details also not a topic in this course
    \end{itemize}
\end{frame}

\begin{frame}{our tool: HCLRS}
    \begin{itemize}
    \item built for this course
    \item assumes you're making a processor
    \vspace{.5cm}
    \item somewhat different from textbook's HCL
    \end{itemize}
\end{frame}

